\documentclass[14pt]{extarticle}
\usepackage{../preamble}
\newcommand{\vect}[2]{\left(
\begin{array}{c}
    #1\\
    #2
\end{array}\right)}
\begin{document}
%=============================================================================!
\begin{MathCl}[об ортогональном базисе в гильбертовом пространстве]{Воспоминание}
    Пусть $H$~--- гильбертово пространств и $\dim H = +\infty$, тогда 
    $\exists \{e_k\}_{k = 1}^\infty \subset H$, что $\forall N\; \{e_1,\dots,e_M\}$~--- линейно
    независимый набор.
\end{MathCl}
\begin{Opr}
    $\{e_k\}_{k = 1}^\infty$ называют базисом в $H$, если $\forall f \in H\;\exists!\{\alpha_
    k\}_{k = 1}^\infty \in \mathbb C\colon$
    $$
    \|f - \sum_{k = 1}^N \alpha_k e_k\| \to 0 \quad N \to \infty 
    $$
    Что по определению равносильно
    $$
    f = \sum_{k = 1}^\infty \alpha_k e_k
    $$
\end{Opr}
\begin{Opr}
    $\{e_k\}_{k = 1}^\infty$ называют полной в  $H$, если $\forall f \in H\; \forall \varepsilon
    > 0\;\exists N = N(\varepsilon, g)\; \exists \{\alpha_k = \alpha_k(\varepsilon, f)\}_{k = 1}^N
    \subset \mathbb C\colon \|f - \sum_{n = 1}^N \alpha_n e_n\| \le \varepsilon$.
    Это равносильно тому, что линейная оболочка $\Lin\{e_k\}_{k = 1}^\infty = \{\sum_{k = 
    1}^\infty \alpha_k e_k \mid \alpha_1 \dots \alpha_N \in \mathbb C\; \forall N \in \mathbb 
    N\}$ всюду плотна в $H$, что равносильно $\rho(f, L_N) \to 0\;N \to 0$, где $L_N = 
    \Lin\{e_k\}_{k = 1}^N$ и $L_N \subset L_{N + 1}$.
    $\rho(f, L_{N + 1}) \le \rho(f, L_N) \le \rho(f, L_{N(\varepsilon, f)}) \le \varepsilon$, поэтому
    $\rho(f, L_N)$ уменьшается при увеличении $N$ и $\lim \limits_{N \to \infty} \rho(f, 
    L_N) = 0$ по теореме Больцано~--- Вейерштрасса.
\end{Opr}
\begin{Prim}[есть полнота, но нет базиса]
    $\{e_k(x) = x^k\}_{k = 0}^\infty, e_0(x) = 1, x\in [0,1]$ в $H = L_2[0,1]$ является полной
    по теореме Вейерштрасса о равномерно приближении многочленом, но это не
    базис (доказать в качестве упражнения)
\end{Prim}
\begin{MathCl}{Утверждение}
    Пусть $\e{e}$~--- ортогональная полная система в $H$, тогда $\e{e}$~--- базис в $H$
\end{MathCl}
\begin{Proof}
    Пусть $L_N = \Line{e}{N}$.
    Тогда по минимальному свойству коэффициентов фурье по ортогональной системе
    $$
    P_{L_N}(f) = S_N(f) = \sum\limits_{n = 1}^N\dfrac{(f, e_n)}{(e_n, e_n)}e_n
    $$
    Причём
    $$
    P_n(f) = \dfrac{(f, e_n)}{(e_n, e_n)}e_n
    $$
    это ортопроекция $f$  на $\Lin\{e_n\}$ и $\|P_n\| = 1$, тогда
    $$
    P_{L_N} = \sum\limits_{n = 1}^N P_n
    $$
    Так как $\rho(f, L_N) \to 0$ и $\rho(f, L_N) = \|f - P_{L_N}(f)\|$, то
    $$
    f = \sump P_nf = \sump \dfrac{(f, e_n)}{(e_n, e_n)}e_n
    $$
    сходимость по норме в $H$.
    $\exists \alpha_n = \dfrac{(f, e_n)}{(e_n, e_n)}$~--- коэффициент фурье $f$ в системе
    $\e{e}$.
    Если существует другое разложение
    $$
    f = \sump \beta_n e_n
    $$
    тогда
    $$
    (f, e_n) = (\lim_{N \to \infty} \sump[n][1][N] \beta_n e_n, e_m) = \lim_{N \to \infty}
    ( \sump[n][1][N] \beta_n e_n, e_m)
    $$
    тогда при $N > m$ это равно $\beta_m (e_m, e_m)$, а значит $\beta_m = \alpha_m =
    \dfrac{(f, e_m)}{(e_m, e_m)}$
\end{Proof}
\begin{Opr}
    $\e{e}$ называют замкнутой, если из $(f, e_k) = 0\; \forall k$ следует, что $f = 0$, что
    равносильно по определению $(\Lin\{e_k\})^\perp = 0$, что равносильно 
    $\overline{\Lin\{e_k\}} = H \Leftrightarrow \{e_k\}$~--- полная система.
\end{Opr}
\begin{MathCl}{Наблюдение}
    $\{e_k\}$~--- полная система, тогда ортогонализуем её по Гильберту~--- Шмитду.
    \begin{gather*}
        0 \ne g_1 = e_1\\
        0 \ne g_2 = e_2 + \alpha_{21}g_1 \perp g_1\\
        \vdots\\
        0 \ne g_n = e_n + \alpha_{n1}g_1 + \dots + \alpha_{n n-1}g_{nn - 1} \perp g_1 \dots g_
        {n - 1}
    \end{gather*}
    Тогда $\Line{g}{N} = \Line{e}{N}$, $g_n = L_n$ и $e_n \in \Line{g}{N}$.
    $\rho(f, L_N) \to 0\;\forall f \in H$ в силу полноты, тогда $\e{g}$ осталась полной, но 
    стала ортогональной, поэтому это базис.
\end{MathCl}
\section*{Ортогональные базисы в декартовом и тензорном произведении двух 
гильбертовых пространств}
Пусть $H_1$ и $H_2$~--- два гильбертова пространства.
Декартовым произведением $H_1$ и $H_2$ обозначается линейной пространство
такое, что
$$
H_1 \times H_2 = \left\{
    u_1 \times u_2 \equiv \left(
    \begin{array}{c}
    u_1\\
    u_2
    \end{array}\right)
    \mid
    \begin{array}{c}
    u_1 \in H_1\\
    u_2 \in H_2
    \end{array}
    \right\}
$$
Тогда $\forall u_k, \tilde{u}_k$, $\forall \alpha \in \mathbb C$
$$
\left(
\begin{array}{c}
    u_1\\
    u_2
\end{array}
\right)
+
\left(
\begin{array}{c}
    \tilde{u}_1\\
    \tilde{u}_2
\end{array}
\right)
=
\left(
\begin{array}{c}
    u_1 + \tilde{u}_1\\
    u_2 + \tilde{u}_2
\end{array}
\right)
$$
и
$$
\alpha
\left(
\begin{array}{c}
    u_1\\
    u_2
\end{array}
\right)
=
\left(
\begin{array}{c}
    \alpha u_1\\
    \alpha u_2
\end{array}
\right)
$$
$0 = H_1 \times H_2$, $0 = \left(\begin{array}{c}
0\\
0
\end{array}\right) = 0 \times 0$.
$$
(u_1 \times u_2, \tilde{u}_1 \times \tilde{u}_2) = (u_1, \tilde{u}_1)_{H_1} + (u_2, \tilde{u}_2)
_{H_2} \ge 0
$$
тогда 
$$
(u_1 \times u_2, u_1 \times u_2) = \|u_1\|^2_{H_1} + \|u_2\|^2_{H_2} \ge 0
$$
Посмотрим, когда скалярное произведение в таком случае будет ноль.
Тогда $u_1 = 0 \in H_1$ и $u_2 = 0 \in H_2$, а значит $\left(\begin{array}{c}
    u_1\\
    u_2
\end{array}
\right) = 0 \in H_1 \times H_2$.
Очевидно $\|u_1 \times u_2\|_{H_1 \times H_2} = \sqrt{\|u_1\|^2_{H_1} + \|u_2\|^2_{H_2}}
$.

Пусть $\e{e}$~--- базис в $H_1$ и $\e{g}$~--- базис в $H_2$.
Рассмотрим систему ортогональных векторов в $H_1 \times H_2$
$$
\left\{\left(
\begin{array}{c}
    e_k\\
    0
\end{array}\right);
\left(
\begin{array}{c}
    0\\
    g_m
\end{array}\right)
\mid
\begin{array}{c}
k \in \mathbb N\\
m \in \mathbb N
\end{array}
\right\}
$$
$u_1 \times u_2 \in H_1 \times H_2$, и $\|u_1 - \sump[k][1][N] \alpha_k e_k\|_{H_1} \le 
\varepsilon$ , и $\|u_2 = \sump[k][1][N] \beta_k g_k\|_{H_2} \le \varepsilon$, тогда
$$
\left(
\begin{array}{c}
    u_1\\
    u_2
\end{array}
\right)
-
\left(
\begin{array}{c}
    \sump[k][1][N] \alpha_k e_k\\
    \sump[k][1][N] \beta_k g_k
\end{array}
\right)
=
\left(
\begin{array}{c}
    u_1\\
    u_2
\end{array}
\right)
-
\sump[k][1][N] \alpha_k
\left(
\begin{array}{c}
    e_k\\
    0
\end{array}\right)
-
\sump[k][1][N] \beta_k
\left(
\begin{array}{c}
    0\\
    g_m
\end{array}
\right) \in \Lin\{\left(
\begin{array}{c}
    e_k\\
    0
\end{array}\right);
\left(
\begin{array}{c}
    0\\
    g_m
\end{array}\right)
\}
$$
Пусть
$$
u = 
\left(
\begin{array}{c}
    u_1 - \sump[k][1][N] \alpha_k e_k\\
    u_2 - \sump[k][1][N] \beta_k g_k
\end{array}
\right)
$$
тогда $\|u\|_{H_1 \times H_2} \le \sqrt{2}\varepsilon$, а значит
$\{\left(
\begin{array}{c}
    e_k\\
    0
\end{array}\right);
\left(
\begin{array}{c}
    0\\
    g_m
\end{array}\right)
\}$ полная в $H_1 \times H_2$, поэтому базис в $H_1 \times H_2$
$$
\left(
\begin{array}{c}
    u_1\\
    u_2
\end{array}
\right)
=
\sump[k] \alpha_k
\left(
\begin{array}{c}
    e_k\\
    0
\end{array}\right)
-
\sump[k] \beta_k
\left(
\begin{array}{c}
    0\\
    g_m
\end{array}
\right)
$$

Тензорное произведение $H_1$ и $H_2$ $H_1 \otimes H_2$.
$\forall u_1 \in H_1$ и $\forall u_2 \in H_2$ $u_1 \otimes u_2 = w \in H_1 \otimes H_2$~--- 
пара.
Введём сложение и умножение на скаляр этих пар $u_1 \otimes u_2$ $u_1 \otimes u_2
+ \tilde{u}_1 \otimes \tilde{u}_2$ и $\alpha(u_1 \otimes u_2)$, $u_k, \tilde{u}_k \in H_k, 
k=1,2$ так, чтобы $\alpha(u_1 \otimes u_2) = (\alpha u_1) \otimes u_2 = u_1 \otimes (\alpha
u_2)$, $(u_1 + \tilde{u}_1) \otimes u_2 = u_1 \otimes u_2 + \tilde{u}_1 \otimes u_2$ и
$u_1  \otimes (u_2 + \tilde{u}_2) = u_1 \otimes u_2 + u_1 \otimes \tilde{u}_2$.
Вводим 
$$
M = \{\text{всевозможные конечные линейные комбинации пар $u_1 \otimes 
u_2$, $u_k \in H_k, k=1,2$}\}
$$
$u_1 \otimes 0$ или $0 \otimes u_2$~--- нули в $H$~--- линейное пространство.
Тогда
\begin{gather*}
u_1 \otimes 0 + u_1 \otimes u_2 = u_1 \otimes u_2\\
u_1 \otimes (0*u_2) = 0 (u_1 \otimes u_2) = 0 \in M
\end{gather*}
Введём скалярное произведение как
$$
(u_1 \otimes u_2, \tilde{u}_1 \otimes \tilde{u}_2)_M = (u_1, \tilde{u}_1)_{H_1}(u_2, \tilde{u}_2)
_{H_2}
$$
Тогда
$$
(u_1 \otimes u_2, u_1 \otimes u_2) = \|u_1\|_{H_1}^2\|u_2\|_{H_2}^2 \ge 0
$$
Оно равно нулю тогда и только тогда, когда $u_1 = 0$ и $u_2 = 0$, следовательно
$u_1 \otimes u_2 = 0$.
Потребуем, чтобы
\begin{multline*}
(u_1 \otimes u_2 + \hat{u}_1 \otimes \hat{u}_2, \tilde{u}_1 \otimes \tilde{u}_2) =\\= 
(u_1 \otimes u_2, \tilde{u}_1 \otimes \tilde{u}_2) +
(\hat{u}_1 \otimes \hat{u}_2, \tilde{u}_1 \otimes \tilde{u}_2)  =\\=
(u_1, \tilde{u}_1)(u_2, \tilde{u}_2) +
(\hat{u}_1, \tilde{u}_1)(\hat{u}_2 , \tilde{u}_2) 
\end{multline*}

Наделим $(M, \|\;\|)$ нормой, порождённой скалярным произведением.
Пополняя $(M, \|\;\|)$ получим $H_1 \otimes H_2$.
$M$ всюду плотно в $H_1 \otimes H_2$ по построению.
В смысле евклидовой нормы
$$
\|u_1 \otimes u_2\| = \sqrt{(u_1 \otimes u_2)(u_1 \otimes u_2)} = \|u_1\| + \|u_2\|
$$

Если $\e{e}$~--- ортогональный базис в $H_1$ и $\e[\infty][m]{g}$~--- ортогональный
базис в $H_2$.
Пусть $h_{km} = e_k \otimes g_m \in M \subset H_1 \otimes H_2$, тогда
$$
(h_{km}, h_{\tilde{k}\tilde{m}})_{H_1 \otimes H_2} = (e_k, e_{\tilde{k}})_{H_1}
(g_m, g_{\tilde{m}})_{H_2}
$$
Если $h_{km} \ne h_{\tilde{k}\tilde{m}}$, тогда либо $k \ne \tilde{k}$, либо
$m \ne \tilde{m}$, тогда $(e_k, e_{\tilde{k}}) = 0$, либо $(h_m, h_{\tilde{m}}) = 0$, а значит
$(h_{km},h_{\tilde{k}\tilde{m}}) = 0$, отсюда система $h_{km}$ ортогональная в $H_1
\otimes H_2$

Берём $w \in H_1 \otimes H_2$, тогда $\forall \varepsilon\; \exists v \in H\colon \|w - v\|
_{H_1 \otimes H_2} \le \varepsilon$.
Пусть
$$
v = \sump[k][1][N] \alpha_k u_{1k} \otimes u_{2k}
$$
где $u_{1k} \in H_1, u_{2k} \in H_2$.
$$
\|u_{1k} - \sump[j][1][M]\beta_{jk}e_j\|_{H_1} \le \delta
$$
Обозначим как $S_1$ частичную сумму $\sump[j][1][M]\beta_{jk}e_j$.
Аналогично
$$
\|u_{2k} - \sump[s][1][M]\gamma_{sk}g_s\|_{H_2} \le \delta
$$
и $S_2 = \sump[s][1][M]\gamma_{sk}g_s$.
Тогда рассмотрим
$$
\|u_{1k} \otimes u_{2k} - S_1 \otimes S_2\|_{H_1 \otimes H_2}
$$
Так суммы сходятся, то $u_{1k} - S_1 = \xi_1$ и $u_{1k} = \xi_1 + S_1$, аналогично
$u_{2k} = \xi_2 + S_2$.
Тогда
$$
\|u_{1k} \otimes u_{2k} - S_1 \otimes S_2\|_{H_1 \otimes H_2} = \|(S_1 + \xi_1) \otimes
(S_2 + \xi_2) - S_1 \otimes S_2\|
$$
Из наложенного требования на скалярное произведение
$$
(S_1 + \xi_1) \otimes (S_2 + \xi_2) = S_1 \otimes S_2 +S_1 \otimes \xi_2 + \xi_1 \otimes S_2 +
\xi_1 \otimes \xi_2
$$
Отсюда
\begin{multline*}
\|u_{1k} \otimes u_{2k} - S_1 \otimes S_2\|_{H_1 \otimes H_2} =\\=
\|S_1 \otimes \xi_2 + \xi_1 \otimes S_2 + \xi_1 \otimes S_2\| \le\\\le
\|S_1 \otimes \xi_2\| + \|\xi_1 \otimes S_2\| + \|\xi_1 \otimes \xi_2\| \le\\\le
(\|u_{1k}\| + \delta)\delta + (\|u_{2k}\| + \delta)\delta + \delta^2
\end{multline*}
подерём $\delta$, чтобы полученное выражение было бы меньше ,чем
$$
\dfrac{\varepsilon}{\sump[k][1][N]|\alpha_k| + 1}
$$
тогда
$$
\|v - \sump[k][1][N] \alpha_k S_{1k} \otimes S_{2k}\| \le \sump[k][1][N]|\alpha_k|
\|u_{1k} \otimes u_{2k} - S_{1k} \otimes S_{2k}\| \le \varepsilon
$$
Отсюда $\e[\infty][km]{h}$ полная ортогональная система в $H_1 \otimes 
H_2$~--- ортогональный базис.

В качестве приложения пусть $L_2[0,1] = H_1 = H_2$, $g, f \in L_2[0,1]$, тогда
определим
$$
f(x) \otimes g(y) = f(x)g(y) \subset L_2([0,1]^2)
$$
тогда 
$$
f_1 \otimes g_2 + \tilde{f}_1 \otimes \tilde{g}_1 = f(x)g(y) + \tilde{f}(x)\tilde{g}(y)
$$
Скалярное произведение
$$
(f \otimes g, \tilde{f} \otimes \tilde{g}) = (f, \tilde{f})_{L_2[0,1]}(g, \tilde{g})_{L_2[0,1]}
= \int_0^1f\tilde{f}dx\int_0^1g\tilde{g}dy
$$
и
$$
(f \otimes g + \hat{f} \otimes \hat{g}, \tilde{f} \otimes \tilde{g}) = 
(f, \tilde{f})(g, \tilde{g}) + (\hat{f},\tilde{f})(\hat{g}\tilde{g})
$$
также
$$
\|f \otimes g\| = \|f\|_{L_2[0,1]}\|g\|_{L_2[0,1]}
$$
Множество всех конечных линейных комбинаций
$$
M = \{\sump[k][1][N] \alpha_k f_k(x) g_k(y) \mid \alpha_k \in \mathbb C; N \in 
\mathbb N; f_k,g_k \in L_2[0,1]\}
$$
\begin{MathCl}{Утверждение}
    Пополнение $M$~--- это $L_2([0,1]^2)$.
    $L_2([0,1] \times [0,1]) = L_2[0,1] \otimes L_2[0,1] = H$
\end{MathCl}
\begin{Proof}
    $\forall w \in L_2([0,1]^2)\;\forall \varepsilon > 0\; \exists v \in C([0,1]^2)$, где 
    $C([0,1]^2)$~--- пополнение $L_2$, такой, что $\|w - v\|_{L_2([0,1]^2)} \le 
    \varepsilon$.
    По теореме Стоуна~--- Вейерштрасса $\exists P(x,y)$ многочлен $\max \limits_
    {[0,1]^2}|P - v| \le \varepsilon$ и $P \in M$.
    Тогда $\|v - P\|_{L_2([0,1]^2)} \le \max \limits_
    {[0,1]}|P - v| \le \varepsilon$, а значит $\|w - P\|_{L_2([0,1]^2)} \le 2\varepsilon$
\end{Proof}

Построим базис.
$\sin \pi k x = e_k(x)$, $\sin \pi m y = g_m(y)$~--- ортогональный базис в $L_2[0,1]$.
Тогда $h_{km}(x,y) = \sin \pi k x \sin \pi m y$~--- ортогональный базис в $L_2
([0,1]^2)$
\begin{Prim}[электрон со спином\?]
$L_2[0,1] \otimes \mathbb C^2$. $\vect{1}{0}$ и $\vect{0}{1}$ образуют базис в 
$\mathbb C^2$.
Пусть $f(x) \in L_2[0,1]\colon f : [0,1] \mapsto \mathbb C$ и $\vect{a}{b} \in 
\mathbb{C}^2$, тогда $f(x)\vect{a}{b} = f(x) \otimes \vect{a}{b}$.
Пусть $\{e_k\}$~--- базис в $L_2[0,1]$, тогда $e_k \vect{1}{0}$ и $e_k\vect{0}{1}$
базис в $L_2[0,1] \otimes \mathbb C^2$. Но это также и базис в $L_2[0,1] \times
L_2[0,1]$. Отсюда $L_2[0,1] \times L_2[0,1] = L_2[0,1] \otimes \mathbb C^2$ и
$$
\vect{\Psi_1(x)}{\Psi_2(x)} = \Psi_1(x) \vect{1}{0} + \Psi_2(x) \vect{0}{1} =
\sump[k]\left(\alpha_k e_k \vect{1}{0} + \beta_k e_k \vect{0}{1}\right)
$$
\end{Prim}
\begin{Theor}[Гильберта, Шмидта]
    Пусть $H$~--- гильбертово пространство и $A : H \mapsto H$~--- компактный
    самосопряжённый оператор. Тогда в $(\ker A)^\perp \ne 0$ есть базис 
    из собственных функций $A$.
\end{Theor}
\begin{Proof}
    $A \ne 0 \Leftrightarrow \|A\| > 0$, так как $A$~--- самосопряжённый, то 
    спектральный радиус $r(A) = \|A\|$, а значит $\exists \lambda \in \sigma(A)
    \setminus\{0\}$, что $|\lambda| = r(A) = \|A\| > 0$.
    По теореме о спектре компактного оператора для любого нетривиального
    элемента его спектра, являющийся собственным значением следует, что
    $\sigma_P(A) \setminus\{0\} \ne \varnothing$. Пусть $\lambda \in \sigma_P(A)
    \setminus\{0\}$, тогда $f \ne 0$ такой, что $f \in \ker \Al$ следует, что
    $f \in (\ker A)^\perp$ так как $\forall g \in \ker A$
    $$
    (Af, g) = \lambda(f, g) = (f, Ag) = 0 \Rightarrow (f, g) = 0
    $$
    
    По 4-ой теореме Фредгольма $\sigma_P(A)\setminus\{0\}$ не более чем 
    счётное с быть может единственной предельной точкой 0. Поэтому 
    занумеруем $\e[N][k]{\lambda} = \sigma_P(A) \setminus\{0\}$, где $N 
    \in \mathbb N \cup \{+\infty\}$. Если $N = +\infty$, то $\lambda_n \to 0\;n \to
    \infty$.
    По 1-ой теореме Фредгольма $\ker \Al[n]$ конечномерно.
    Отсюда $\forall n \in 1 \dots N\; \exists h_{n_1}\dots h_{n_m} \in \ker \Al[n]$
    ортогональный базис в $\ker \Al[n]$, где $m_n = \dim \ker \Al[n]$.
    Так как $A$~--- самосопряжённый, то $\ker \Al[n] \perp \ker \Al[m]$ при $n \ne 
    m$.
    $$
    M = \bigoplus\limits_{n = 1}^N \ker(\Al[n])
    $$
    Это линейная оболочка, то есть
    $$
    M = \Lin\{h_{n_k}\}_{k = 1\dots m_n}^{n = 1 \dots N}
    $$
    Замыкание $\overline{M} = L$ является замкнутым подпространством в
    $(\ker A)^\perp$, тогда 
    $$
    \{h_{n_k}\}_{k = 1\dots m_n}^{n = 1 \dots N}
    $$
    полная ортогональная система в $L$ и значит $\{h_{n, k}\}$~--- 
    ортогональный базис в $L$.
    Таким образом $L \subset (\ker A)^\perp$~--- замкнутое подпространство с
    ортогональным базисом $\{h_{n,k}\}$ из собственных функция $A$.
    Хотелось бы доказать, что $L = (\ker A)^\perp$.
    
    Если вдруг $L \ne (\ker A)^\perp$, тогда по теореме Рисса об ортогональном 
    разложении $\exists K \subset (\ker A)^\perp$~--- замкнутое подпространство,
    $K \perp L$ и $K \oplus L = (\ker A)^\perp$ и $K \ne 0$.
    По построению видно, что $A(L) \subset L$ ($L$~--- инвариантное
    подпространство).
    Действительно, пусть $f \in L$, тогда
    $$
    f = \sump[n][1][N] \sump[k][1][m_n] \alpha_{n,k} h_{n, k}
    $$
    В случае $N = \infty$  ряд сходится в $H$.
    Тогда из непрерывности $Af$ получаем
    $$
    Af = \sump[n][1][N] \sump[k][1][m_n] \alpha_{n,k} A h_{n, k} =
    \sump[n][1][N] \sump[k][1][m_n] \alpha_{n,k} \lambda_{n, k} h_{n, k} \in L
    $$
    Так как $A$~--- самосопряжённый оператор, то $A(K) \subset K$ так как
    $\forall g \in K\;\forall f \in L$ так как $Af \in L$, то $Af \perp g$ и
    $$
    (Af, g) = (f, Ag) = 0
    $$
    А значит $Ag \perp f$ и $Ag \in (\ker A)^\perp$, что равносильно $Ag \in K$.
    Если вдруг $A \ne 0$ на $K$, то $A$ наследует свои свойства из $H$ то есть
    является компактным самосопряжённым оператором из $K$ в $K$.
    Тогда по свойствам компактного самосопряжённого оператора существует
    собственное значение $\lambda \neq 0$ для $A : K \mapsto K$ и
    существует $g \in K\colon Ag = \lambda g, \lambda \ne 0$, тогда 
    $\lambda \in \sigma_P(A)\setminus\{0\}$.
    Но тогда $\lambda = \lambda_n$ и $g \in \ker \Al[n]$, но это ядро по 
    построению ортогонально $K$, поэтому получаем $g \perp g$, значит
    $g = 0$ противоречие, следовательно $A = 0$ на $K$, значит
    $K \subset \ker A$, но $K \subset (\ker A)^\perp$ следовательно $K = 0$.
\end{Proof}

Если $H$~--- сепарабельное гильбертово пространство, то есть содержит
счётное всюду плотное множество, то если $\ker A \ne 0$, то оно тоже
сепарабельное (доказать в качестве упражнения), тогда в $\ker A$ существует
полная система $\e[S]{g}$, где $S = \mathbb N \cup \{+\infty\}$.
Ортогонализуем её и получим ортогональный базис $\{e_k\}$ в $\ker A$.
\end{document}