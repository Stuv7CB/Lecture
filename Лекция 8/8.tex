\documentclass[14pt]{extarticle}
\usepackage{../preamble}
\newcommand{\vect}[2]{\left(
\begin{array}{c}
    #1\\
    #2
\end{array}\right)}
%=============================================================================!
\begin{document}
\section*{Задача Штурма-Лиувиля}

Пусть оператор $A$ задан, как:
$$
A = a(x)\dfrac{d^2}{dx^2} + b(x)\dfrac{d}{dx} + c(x)I, \quad x \in [\alpha, \beta]
$$
где $a(x), b(x), c(x) \in C[\alpha, \beta]$~--- вещественнозначные функции, причём $a \ne 0\;
\forall x \in [\alpha, \beta]$.
Область определения оператора $A$:
$$
D(A) = \left\{ u \in C^2[\alpha,\beta] \left\arrowvert
\begin{array}{l}
    \mu_1 u'(\alpha) + \nu_1 u(\alpha) = 0 \quad (1)\\
    \mu_2 u'(\beta) + \nu_2 u(\beta) = 0 \quad (2)
\end{array}
\right.\right\}
$$
где $\mu_k, \nu_k \in \mathbb R\;k = 1,2$ и
\begin{gather*}
    |\mu_1| + |\nu_1| > 0\\
    |\mu_2| + |\nu_2| > 0
\end{gather*}
Оператор $A$ действует $A : D(A) \mapsto L_2[\alpha, \beta]$, где $L_2[\alpha, \beta] = H$ и
$D(A) \subset H$, причём $\overline{D(A)} = H$.
\begin{MathCl}{Утверждение}
    $\ker A = 0$ равносильно тому, что существует фундаментальная система решений
    уравнения
    $$
    \left\{
    \begin{array}{l}
        Av = 0\\
        v \in C^2[\alpha, \beta]
    \end{array}
    \right.
    $$
    $\{v_1, v_2\} \subset C^2[\alpha, \beta]$, $\forall k\;Av_k = 0$ и $v_1$ удовлетворяет (1), но
    не удовлетворяет (2), и $v_2$ удовлетворяет (2), но не удовлетворяет (1).
\end{MathCl}
\begin{Proof}
    Граничные условия, определяющие $v_1$ и $v_2$ эквивалентны заданию задачи для 
    Коши.
    $v_1 \ne 0$ и $v_1 \in C^2[\alpha, \beta]$
    $$
    \left\{
    \begin{aligned}
        &Av_1 = 0\\
        &v_1(\alpha) = \mu_1\\
        &v'_1(\alpha) = -\nu_1
    \end{aligned}
    \right.
    $$
    И соответственно $v_2 \ne 0$ и $v_2 \in C^2[\alpha, \beta]$
    $$
    \left\{
    \begin{aligned}
        &Av_2 = 0\\
        &v_2(\beta) = \mu_2\\
        &v'_2(\beta) = -\nu_2
    \end{aligned}
    \right.
    $$
    Так как $\mu_k, \nu_k \in \mathbb R$ и $a, b, c$~--- вещественнозначные функции, 
    то и $v_1$ и $v_2$ вещественнозначные.
    Тогда решение уравнения
    $$
    \left\{
    \begin{aligned}
        &Au = 0\\
        &u \in C^2[\alpha, \beta]
    \end{aligned}
    \right.
    $$
    принимает вид $u = C_1 v_1 + C_2 v_2$ в силу независимости $v_1$ и $v_2$.
    Поэтому ядро $\ker A = 0$.
\end{Proof}

Когда $\ker A = 0$ надо построить $A^{-1} : \Im A \mapsto D(A)$, 
верно ли, что $\Im A = C[\alpha, \beta]$?
Существует ли $u \in D(A)\colon Au = f \in C[\alpha, \beta]$.
Если существует, то единственность автоматически следует из $\ker A = 0$.
Ищем решение в виде $u(x) = C_1(x)v_1(x) + C_2(x)v_2(x)$, где $C_1, C_2 \in C^2[\alpha, \beta]$.
$$
\left\{
\begin{aligned}
    &C'_1 v_1 + C'_2 v_2 = 0\\
    &a(C'_1 v'_1 + C'_2 v'_2) = f
\end{aligned}
\right.
$$
Тогда
$$
\left(
\begin{array}{cc}
    v_1 & v_2\\
    v'_1 & v'_2
\end{array}
\right)
\vect{C'_1}{C'_2}
=
\vect{0}{\frac{f}{a}}
$$
на $[\alpha, \beta]$.
Тогда определитель Вронского
$$
w = \det \left|
\begin{array}{cc}
    v_1 & v_2\\
    v'_1 & v'_2
\end{array}
\right|
=
\mathrm{const} e^{-\int \frac{b}{a}dx}
$$
Тогда
$$
    \vect{C'_1}{C'_2} = \vect{-\frac{v_2}{wa}f}{\frac{v_1}{wa}f}
$$
В итоге
\begin{gather*}
    C_1(x) = \int \limits_x^\beta \dfrac{v_2(t)}{w(t)a(t)}f(t)dt + D_1\\
    C_2(x) = \int \limits_\alpha^x \dfrac{v_1(t)}{w(t)a(t)}f(t)dt + D_2
\end{gather*}
Получаем соответственно
$$
u(x) = \intl[\alpha][x][dt]{\dfrac{v_1(t) v_2(x)}{a(t) w(t)} f(t)} +
\intl[x][\beta][dt]{\dfrac{v_1(x) v_2(t)}{a(t) w(t)} f(t)} + D_1 + D_2
$$
Используем, что $u \in D(A)$.
Подставим в первое граничное условие:
\begin{multline*}
    \mu_1(\dfrac{v_1(\alpha) v_2(\alpha)}{(aw)(\alpha)} f(\alpha) +
    \intl[\alpha][\alpha][dt]{\dfrac{v_1(t) v'_2(\alpha)}{a(t) w(t)} f(t)} +\\+
    (-1)\dfrac{v_1(\alpha)v_2(\alpha)}{(aw)(\alpha)}f(\alpha) +
    \intl[\alpha][\beta][dt]{\dfrac{v'_1(\alpha) v_2(t)}{a(t) w(t)} f(t)} +
    D_1 v'_1(\alpha) + D_2 v'_2(\alpha)) +\\+
    \nu_1 (\intl[\alpha][\beta][dt]{\dfrac{v_1(\alpha) v_2(t)}{a(t) w(t)} f(t)} +
    D_1 v_1(\alpha) + D_2 v_2(\alpha)
    ) = 0
\end{multline*}
Упрощаем и получаем
\begin{multline*}
    (\mu_1 v'_1(\alpha) + \nu_1 v_1(\alpha))
    \intl[\alpha][\beta][dt]{\dfrac{v'_1(\alpha) v_2(t)}{a(t) w(t)} f(t)} +\\+
    D_1 (\mu_1 v'_1(\alpha) + \nu_1 v_1(\alpha)) +
    D_2 (\mu_1 v'_2(\alpha)) + \nu_1 v_2(\alpha))
    = 0
\end{multline*}
Отсюда $D_2 = 0$.
Теперь подставим во второе граничное условие
\begin{multline*}
    \mu_2(\dfrac{v_1(\beta) v_2(\beta)}{(aw)(\beta)}f(\beta) +
    \intl[\alpha][\beta][dt]{\dfrac{v_1(t) v'_2(\beta)}{a(t) w(t)} f(t)} +\\+
    (-1)\dfrac{v_1(\beta) v_2(\beta}{(aw)(\beta)}f(\beta) +
    \intl[\beta][\beta][dt]{\dfrac{v'_1(\beta)) v_2(t)}{a(t) w(t)} f(t)}+
    D_1 v'_1(\alpha)) +\\+
    \nu_2(\intl[\alpha][\beta][dt]{\dfrac{v_1(t) v_2(\beta)}{a(t) w(t)} f(t)} +
    D_1 v_1(\beta)) = 0
\end{multline*}
Тогда
\begin{multline*}
    (\mu_2 v'_2(\beta) + \nu_2 v_2(\beta))
    \intl[\alpha][\beta][dt]{\dfrac{v_1(t)}{a(t) w(t)} f(t)} +\\+
    D_1((\mu_2 v'_1(\beta) + \nu_2 v_1(\beta))
    = 0
\end{multline*}
А значит и $D_1 = 0$.
Осталось проверить, что $u \in C^2[\alpha, \beta]$.
$\forall x \in [\alpha, \beta]$
\begin{multline*}
u'(x) = (\dfrac{v_1 v_2}{aw}f)(x) + 
\intl[\alpha][x][dt]{\dfrac{v_1(t)v'_2(x)}{a(t) w(t)} f(t)} -\\-
(\dfrac{v_1 v_2}{aw}f)(x) + \intl[x][\beta][dt]{\dfrac{v_2(t)v'_1(x)}{a(t) w(t)} f(t)}
=\\=
\intl[\alpha][x][dt]{\dfrac{v_1(t)v'_2(x)}{a(t) w(t)} f(t)} +
\intl[x][\beta][dt]{\dfrac{v_2(t)v'_1(x)}{a(t) w(t)} f(t)}
\end{multline*}
Так как $v'_1, v'_2 \in C^1[\alpha, \beta]$ и интеграл от функций в $C[\alpha, \beta]$, то
$u' \in C^1[\alpha, \beta]$, а значит $u \in C^2[\alpha, \beta]$.
Таким образом для любая непрерывная функция $f$ порождает решение, значит
действительно $\Im A = C[\alpha, \beta]$.

Определим теперь $T : L_2[\alpha, \beta] \mapsto L_2[\alpha, \beta]$.
$$
(Tf)(x) = \intl[\alpha][\beta][dt]{K(t, x) f(t)} \quad f \in L_2[\alpha, \beta]
$$
где
$$
K(t, x) = \dfrac{1}{(aw)(t)}
\left\{
\begin{aligned}
    &v_1(t)v_2(x) \quad \alpha \le t \le x \le \beta\\
    &v_1(x)v_2(t) \quad \alpha \le x \le t \le \beta
\end{aligned}
\right.
$$
$K(t, x) \in C^2[\alpha, \beta]^2) \subset L_2([\alpha,\beta]^2)$.
Далее $\Im T \subset C[\alpha, \beta] = \Im A$ и $T|_{C[\alpha, \beta]} = A^{-1}$.
$\forall f \in C[\alpha, \beta]$:
$$
\left\{
\begin{aligned}
    &Au = f\\
    &u \in D(A)
\end{aligned}
\right. \Leftrightarrow
\left\{
\begin{aligned}
    &u = Tf \in D(A)\\
    &f \in C[\alpha, \beta]
\end{aligned}
\right.
$$
\begin{MathCl}[когда $T$~--- самосопряжённый оператор]{Наблюдение}
    Так как $a, b, c$~--- вещественнозначный и $\mu_k, \nu_k \in \mathbb R$, $k = 1, 2$, то
    $v_1, v_2$~--- вещественнозначные, а значит и $K$ также вещественнозначная.
    Таким образом условием самоспряжённости для $T$ является $K(t, x) = K(x, t)\;\forall t, x
    \in [\alpha, \beta]$.
\end{MathCl}
\\Этому условию препятствует $\dfrac{1}{(aw)(t)}$, поэтому нужно, чтобы это была бы
константа. $aw = \mathrm{const}$ на $[\alpha, \beta]$, если $a \in C^1[\alpha, \beta]$ и
$a'(x) = b(x)\;\forall x \in [\alpha, \beta]$.
Тогда
$$
\int\dfrac{b}{a} \mathop{dx} = \int \dfrac{a'}{a} \mathop{dx} = \ln |a| 
$$
тогда
$$
w = \mathrm{const} e^{-\ln |a|} = \dfrac{\mathrm{const}}{|a|}
$$
на $[\alpha, \beta]$.
Так как $|a| = a \mathop{\mathrm{sgn}}\nolimits(a)$ и $a \ne 0\;\forall x \in [\alpha, \beta]$
и $a \in C^1[\alpha, \beta]$, то $\mathop{\mathrm{sgn}}\nolimits(a) \equiv \mathrm{const}$
$$
w = \dfrac{\mathrm{const}}{a}
$$
значит $aw = \mathrm{const}$ на $[\alpha, \beta]$.
При $a' = b$ оператор $A$ имеет вид 
$$
(Au)(x) = \dfrac{d}{dx}(a(x)\dfrac{d}{dx}u(x)) + c(x) u(x)
$$
Такой оператор является симметричным на $D(A)$, а значит и $T$ является
самосопряжённым в $H$.

Пусть $a' = b$ и $T$~--- компактный самосопряжённый оператор.
Тогда по теореме Гильберта~--- Шмидта в $H$ существует ортогональный базис 
$\e{e} \subset L_2[\alpha, \beta]$ из собственных функций $T$, то есть $Te_k = \lambda_k
e_k$, где $\lambda_k \ne 0$ так как $\ker  T = (\Im T^*)^\perp = (\Im T)^\perp$, где
$(\Im T)^\perp \subset (D(A))^\perp = \left(\overline{D(A)}\right)^\perp = H^\perp = 0$.
$\e[N]{\lambda}$~--- набор собственных значений, где $N \in \mathbb N \cup \{+\infty\}$.
$\ker T_{\lambda_k}$~--- конечномерный по 1-ой теореме Фредгольма.
Пусть $\dim \ker T_{\lambda_k} = m_\lambda$.
$e_{1, k}\dots _{m_k, k}$~--- ортогональный базис в $\ker T_{\lambda_k}$, $Te_{i,k} = 
\lambda_k e_{i,k}$, $i =1\dots m_k$. $\{e_{i,k}\}^{k = 1 \dots N}_{i = 1 \dots m_k}$,
$N \in \mathbb N \cup \{+\infty\}$.
Так как $L_2[\alpha, \beta] = H$ бесконечномерно и $T_{\lambda_k}$ конечномерно, то
$N$ должно быть равно $+\infty$.
Тогда $\lambda_k \to 0\; k \to +\infty$ по 4-ой теореме Фредгольма и $\lambda_k \in \mathbb 
R$ так как $T$~--- самосопряжённый.
\begin{MathCl}{Утверждение}
    $\forall k \in \mathbb N\; \forall i = 1 \dots m_k$ $e_{i, k} \in D(A)$ и $A e_{i,k} =
    \dfrac{1}{\lambda_k}e_{i,k}$, то есть $\{e_{i,k}\}^{k = 1 \dots N}_{i = 1 \dots m_k}$~---
    ортогональный базис в $H$ из собственных функций $A$.
\end{MathCl}
\begin{Proof}
    $T e_{i, k} = \lambda_k e_{i, k}$, где $\lambda_k \ne 0$.
    Тогда $e_{i, k} = \dfrac{1}{\lambda_k} Te_{i, k}$.
    $\Im T \subset C[\alpha, \beta] = \Im A \Rightarrow e_{i, k} \in C[\alpha, \beta]$.
    С учётом того, что $T|_{C[\alpha, \beta]} = A^{-1} : C[\alpha, \beta] \mapsto D(A)$
    получаем $T e_{i, k} \in D(A)$, значит $e_{i,k} \in D(A)$.
    А значит $A e_{i, k} = \dfrac{1}{\lambda_k}A A^{-1} e_{i, k} = \dfrac{1}{\lambda_k} e_{i,k}$.
\end{Proof}
\begin{Theor}[Стеклова]
    Пусть $a \in C^1[\alpha, \beta]$, $b = a'$, $c \in C[\alpha, \beta]$. Пусть $a, b ,c \in \mathbb R
    $ и $\mu_k, \nu_k \in \mathbb R$, $k = 1, 2$ и пусть существует специальная 
    фундаментальная система решений $A$ (то есть $\ker A = 0$).
    Тогда а) $A$ обладает в $H$ ортогональным базисом из своих собственных функций,
    отвечающих различным собственным значениям $\dfrac{1}{\lambda_k}, \lambda_k \to 0$.
    б) $\forall \lambda \ne \dfrac{1}{\lambda_k}\;\forall k \in \mathbb N$; $\forall f \in C[\alpha,
    \beta]\; \exists! u \in D(A)$ уравнения
    $$
    \left\{
    \begin{aligned}
        &Au = \lambda u + f\\
        &u \in D(A)
    \end{aligned}
    \right\}
    $$
    такое, что 
    $$
    u = \sump[k] \sump[i][1][m_k]\dfrac{f_{i,k}}{\frac{1}{\lambda_k} - \lambda} e_{i,k}
    $$
    сходящаяся в $H = L_2[\alpha, \beta]$, где $f_{i,k} = \dfrac{(f, e_{i,k})}{(e_{i,k}, e_{i,k})}$.
\end{Theor}
\begin{Proof}
    а) уже доказано.
    Докажем б).
    Помним, что $\forall g \in C[\alpha, \beta]$ уравнение
    $$
    \left\{
    \begin{aligned}
        &Au = g\\
        &u \in D(A)
    \end{aligned}
    \right. \Leftrightarrow
    \left\{
    \begin{aligned}
        &u = Tg\in D(A)\\
        &g \in C[\alpha, \beta]
    \end{aligned}
    \right.
    $$
    А так как $\lambda u + f \in C[\alpha, \beta]$, то
    $$
    \left\{
    \begin{aligned}
        &A u = \lambda u + f\\
        &u \in D(A)
    \end{aligned}
    \right.
    \Rightarrow
    \left\{
    \begin{aligned}
        &u = \lambda T u  + T f \in D(A)\\
        &f \in C[\alpha, \beta]\\
        &u \in C[\alpha, \beta]
    \end{aligned}
    \right.
    $$
    В обратную сторону рассмотрим уравнение
    $$
    u = \lambda T u + T f
    $$
    для $u \in L_2[\alpha, \beta]$.
    Это уравнение Фредгольма второго рода.
    $f \in C[\alpha, \beta]$, то помним, что $T u \in C[\alpha, \beta]$, так как $\Im T \subset
    C[\alpha, \beta]$, а значит $Tu$ и $Tf$ автоматически из $C[\alpha, \beta]$.
    Следовательно $u \in C[\alpha, \beta]$, поэтому
    $$
    \left\{
    \begin{aligned}
        &u = \lambda T u  + T f \in D(A)\\
        &f \in C[\alpha, \beta]\\
        &u \in C[\alpha, \beta]
    \end{aligned}
    \right.
    \Rightarrow
     \left\{
    \begin{aligned}
        &A u = \lambda u + f\\
        &u \in D(A)\\
        &f \in C[\alpha, \beta]
    \end{aligned}
    \right.
    $$
    
    Таким образом $Au = \lambda u + f$, $u \in D(A)$  и $f \in C[\alpha, \beta]$ равносильно
    $$
    \left\{
    \begin{aligned}
        &u = \lambda T u  + T f \in D(A)\\
        &f \in C[\alpha, \beta]\\
        &u \in H
    \end{aligned}
    \right.
    $$
    Разложим
    \begin{gather*}
    u = \sump[k]\sump[i][1][m_k]u_{i,k}e_{i,k}\\
    f  = \sump[k]\sump[i][1][m_k]f_{i,k}e_{i,k}
    \end{gather*}
    и подставим в $u = \lambda T u + T f$.
    Так как
    $$
    Tu = \sump[k]\sump[i][1][m_k]u_{i,k} T e_{i,k} = \sump[k]\sump[i][1][m_k]u_{i,k} 
    \lambda_ke_{i,k}
    $$
    сходится в $H$.
    $$
    Tf = \sump[k]\sump[i][1][m_k]f_{i,k} T e_{i,k} = \sump[k]\sump[i][1][m_k]f_{i,k} 
    \lambda_ke_{i,k}
    $$
    Получаем
    $$
    \sump[k]\sump[i][1][m_k]u_{i,k}e_{i,k} = \sump[k]\sump[i][1][m_k] \lambda_k
    (\lambda u_{i,k} + f_{i,k})e_{i,k}
    $$
    Следовательно
    $$
    u_{i,k}(1 - \lambda \lambda_k) = \lambda_k f_{i,k}
    $$
    значит
    $$
    u_{i,k} = \dfrac{f_{i,k}}{\frac{1}{\lambda_k} - \lambda}
    $$
    Так как $A$ симметрична на $D(A)$, то
    $$
    Au = \sump[k]\sump[i][1][m_k]\dfrac{(Au, e_{i,k})}{(e_{i,k}, e_{i,k})}e_{i,k} = 
    \sump[k]\sump[i][1][m_k]\dfrac{(u, Ae_{i,k})}{(e_{i,k}, e_{i,k})}e_{i,k} =
    \sump[k]\sump[i][1][m_k]\dfrac{1}{\lambda_k}\dfrac{(u, e_{i,k})}{(e_{i,k}, e_{i,k})}e_{i,k} = 
    $$
    а $\dfrac{(u, e_{i,k})}{(e_{i,k}, e_{i,k})} = u_{i,k} = \dfrac{f_{i,k}}{\frac{1}{\lambda_k} - \lambda}$.
    Значит
    $$
    A u =  \sump[k]\sump[i][1][m_k]\dfrac{f_{i,k}}{1 - \lambda \lambda_k}e_{i,k}
    $$
    сходится в $H$.
    $$
    \|Au\|^2 =  \sump[k]\sump[i][1][m_k]\dfrac{|f_{i,k}|^2}{|1 - \lambda_k \lambda|^2} \le +\infty
    \|e_{i,k}\|^2
    $$
    $|f_{i,k}|^2\|e_{i,k}\|^2$ член сходящегося ряда, так как $f \in H$.
    $u \in D(A)$, значит $\exists Au \in C[\alpha, \beta] \in H$ что как будто бы
    $$
    Au =  \sump[k]\sump[i][1][m_k] u_{i,k}Ae_{i,k}
    $$
    Как будто потому что на самом деле $A$ разрывный оператор, а такое свойство 
    появилось в силу симметричности на $D(A)$.
\end{Proof}
\section*{Собственные функции оператора Лапласа в круге с однородными 
граничными условиями} 

Пусть $G \subset \mathbb R^2$~--- ограниченная область с кусочно гладкой границей.
$H = L_2(G)$, $\Delta : D(A) \mapsto H$, $D(\Delta) \subset H$.
$$
D(\Delta) = \left\{u \in C^2(\overline{G}) \mid u|_{\partial G} = 0\right\}
$$
Можно также использовать
$$
D_1(\Delta) = \left\{u \in C^2(\overline{G}) \mid \left.\dfrac{\partial u}{\partial n}\right|_{\partial 
G} = 0\right\}
$$
где $n$~--- единичный нормальный вектор на $\partial G$.
\begin{MathCl}{Утверждение}
    $\Delta$~--- симметричный отрицательно определённый оператор на $D(\Delta)$ 
    (на $D_1(\Delta)$ он симметричный и отрицательно полуопределённый).
\end{MathCl}
\begin{Proof}
    Пусть $u, v \in D(\Delta)$
    $$
    (\Delta u, v) = \intl[G][][dxdy]{\Delta u \overline{v}} = 
    \intl[G][][dxdy]{(u_{xx}\overline{v} + u_{yy}\overline{v})}
    $$
    Используем формулу Грина
    $$
    \oint\limits_{\partial G}(P \mathop{dx} + Q \mathop{dy}) = 
    \intl[G][][dxdy]{(Q_x - P_y)}
    $$
    и получаем
    $$
    \intl[G][][dxdy]{u_{xx}\overline{v}} = \intl[G][][dxdy]{((u_x \overline{v})'_x - u_x v_x)}    
    $$
    Аналогично
    $$
    \intl[G][][dxdy]{u_{yy}\overline{v}} = \intl[G][][dxdy]{((u_y \overline{v})'_y - u_y v_y)}    
    $$
    В итоге
    \begin{multline*}
        (\Delta u, v) = \intl[G][][dxdy]{\Delta u \overline{v}} = 
        \intl[G][][dxdy]{(u_{xx}\overline{v} + u_{yy}\overline{v})} =\\=
        \intl[G][][dxdy]{((u_{x}\overline{v})'_x + (u_{yy}\overline{v})'_y)} -
        \intl[G][][dxdy]{(u_x v_x + u_y v_y)} =\\=
        \oint \limits_{\partial G} \overline{v}(u_x \mathop{dy} - u_y \mathop{dx}) - \intl[G][][dxydy]
        {(\nabla u, \nabla v)}
    \end{multline*}
    Пусть $r(t)$ задаёт границу $\partial G$.
    Пусть $r = \vect{a}{b}$, тогда единичная нормаль имеет вид $n =
    \dfrac{1}{\sqrt{a^2 + b^2}}\vect{b}{-a}$.
    Тогда
    \begin{multline*}
    \oint \limits_{\partial G} \overline{v}(u_x \mathop{dy} - u_y \mathop{dx}) - \intl[G][][dxydy]
    {(\nabla u, \nabla v)} =\\=
    \oint \limits_{\partial G} \overline{v} (\nabla u, n)|r|\mathop{dt} - \intl[G][][dxydy]
    {(\nabla u, \nabla v)} =\\=
    \oint \limits_{\partial G} \overline{v} \dfrac{\partial u}{\partial n} \mathop{ds} - \intl[G][][dxydy]
    {(\nabla u, \nabla v)}
    \end{multline*}
    $u, v \in D(\Delta)$, тогда
    $$
    (\Delta u, v) = -\intl[G][][dxdy]{(\nabla u, \nabla v)}
    $$
    с другой стороны
    $$
    (u, \Delta v) = \overline{(\Delta u, v)} = -\intl[G][][dxdy]{\overline{(\nabla u, \nabla v)}} = -
    \intl[G][][dxdy]{(\nabla u, \nabla v)}
    $$
    Значит
    $$
    (\Delta u, v) = (u, \Delta v)
    $$
    Аналогично для $u, v \in D_1(\Delta)$:
    \begin{gather*}
        (\Delta u, v) = -\intl[G][][dxdy]{(\nabla u, \nabla v)}\\
        (u, \Delta v) = \overline{(\Delta u, v)} = -\intl[G][][dxdy]{\overline{(\nabla u, \nabla v)}} = -
        \intl[G][][dxdy]{(\nabla u, \nabla v)}
    \end{gather*}
    Значит
    $$
    (\Delta u, v) = (u, \Delta v)
    $$
    
    Теперь пусть $u \in D(\Delta)$, тогда
    $$
    (\Delta u, u) = - \intl[G][][dxdy]{|\nabla u|^2} \le 0
    $$
    Так как $|\nabla u|^2 \ge 0$ и $|\nabla u|^2 \in C(\overline{G})$, значит $|\nabla u|^2 = 0$
    равносильно $\nabla u \equiv 0$ в $G$, значит $u \equiv \mathrm{const}$, а так как
    $u \in D(\Delta)$, то $u \equiv 0$ в $G$, значит $(\Delta u, u) < 0\; \forall u \not \equiv 0 \in 
    D(\Delta)$

    Если же $u \in D_1(\Delta)$, то
    $$
    (\Delta u, u) = - \intl[G][][dxdy]{|\nabla u|^2} \le 0
    $$
    равно нулю при $u = \mathrm{const} \in D_1(\Delta)$
\end{Proof}
\begin{MathCl}{Следствие}
    $\Delta : D(\Delta) \mapsto H$ симметрично отрицательно определённый оператор,
    значит все его собственные значения вещественные и отрицательные, а собственные
    функции, отвечающие различным собственным значениям ортогональны в $H$.
\end{MathCl}
\begin{Proof}
    Пусть $\Delta u = \lambda u$, $u \in \mathbb C$, $u \in D(\Delta)\setminus\{0\}$.
    Тогда
    $$
    (\Delta u, u) = \lambda \|u\|^2 = (u, \Delta u) = \overline{\lambda} \|u\|^2
    $$
     значит $\lambda = \overline{\lambda}$.
     Далее
     $$
     (\Delta u, u) = \lambda\|u\|^2 < 0
     $$
     значит $\lambda < 0$ так как $\|u\|^2 > 0$.
     Если $v \in D(\Delta) \setminus \{0\}$ и $\Delta v = \mu v$, $\mu \ne \lambda$, то
     $$
     (\Delta u, v) = \lambda (u, v) = (u, \Delta v) = \mu(u, v)
    $$
    Значит $(u, v) = 0$.
\end{Proof}

Ищем ортогональный базис в $L_2(G)$, где
$$
G = C_R(0) = \left\{ \vect{x}{y} \left\arrowvert x^2 + y^2\right.\right\}
$$
из собственных функций $\Delta : D(\Delta) \mapsto H$
$$
D(\Delta) = \{u \in C^2(\overline{C}_R(0)) \mid u|_{x^2 + y^2 = R^2} = 0\}
$$

Сделаем замену координат $x = r \cos \varphi$, $y = r \sin \varphi$, где $0 \le r < R$ и
$\varphi \in [0, 2\pi]$
Тогда
$$
\Delta = \dfrac{\partial^2}{\partial r^2} + \dfrac{1}{r} \dfrac{\partial}{\partial r} + \dfrac{1}{r^2}
\dfrac{\partial^2}{\partial \varphi^2}
$$
и 
$$
C_R(0) = \left\{
\begin{array}{c}
    0 \le r < R\\
    0 \le \varphi < 2\pi
\end{array}
\right\}
$$
Пусть $f = f(r, \varphi) \in L_2(C_R(0))$
$$
\|f\| = \sqrt{\intl[0][R][dr]{r\intl[0][2\pi][d\varphi]{|f|^2}}}
$$
Если бы $f(r, \varphi) = g(r)h(\varphi)$, где $h \in L_2[0,2\pi] = H_2$
и $g \in L_{2r}[0, R] = H_1 = \{g : [0, R] \mapsto \mathbb C \mid \intl[0][R][dr]{r|g(r)|^2} \le 
+\infty\}$,
тогда $\|f\|_H = \|g\|_{H_1} \|h\|_{H_2}$.
Пусть $g_1, g_2 \in H_1$, тогда
$$
(g_1, g_2) = \intl[0][R][dr]{r g_1 \overline{g_2}}
$$
\end{document}

































