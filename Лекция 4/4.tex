\documentclass[12pt]{article}
\usepackage[russian]{babel}
\usepackage[utf8]{inputenc}
\frenchspacing
\usepackage{amssymb, amsmath, amscd}
\usepackage[left=20mm, right=20mm, top=20mm, bottom=20mm]{geometry}
\usepackage{comment}
\usepackage{theorem}
\renewcommand{\Im}{\operatorname{Im}}
\DeclareMathOperator{\Lin}{Lin}
\newcommand{\Al}{A_\lambda}
\newcommand{\Alo}{(\Al)^{-1}}
\newcommand{\Rez}{(I - \lambda A)^{-1}}
\begin{document}
\newtheorem{Theor}{Теорема}
\newtheorem{Utv}{Утверждение}
\newtheorem{Opr}{Опреление}
\newtheorem{Prim}{Пример}
\newtheorem{Upr}{Упражнение}
\newtheorem{Nabl}{Наблюдение}
\newtheorem{Zam}{Замечание}
%=============================================================================!
\section*{Свойства самосопряжённых линейных операторов в гильбертовом
пространств}
Пусть $H$ --- гильбертово пространство.
Пусть $A : H \mapsto H$ --- линейный и непрерывный оператор.
Пусть $A$ --- самосопряжённый оператор, то есть $A = A^*$.
\begin{Utv}
    Если $A$ --- самосопряжённый оператор, тогда выполняются следующие
    свойства:
    \begin{enumerate}
        \item{$\|A^n\| = \|A\|^n\; \forall n \in \mathbb N$}
        \item{$r(A) = \|A\|$}
        \item{$\sigma(A) \subset \mathbb R$}
        \item{пусть $m_+ = \sup \limits_{\|f\| = 1} (Af, f)$, где $(Af, f) \in 
                \mathbb R$ для самосопряжённого оператора, пусть $m_- = \inf 
                \limits_{\|f\| = 1} (Af, f)$, тогда $m_{+-} \in \sigma(A)$ и
                $\sigma(A) \subset [m_-(A), m_+(A)]$ (доказать в качестве
            упражнения)}
    \end{enumerate}
\end{Utv}
\begin{Proof}
    Докажем первое утверждение.
    $\|A^n\| \le \|A\|^n$ по определению операторной нормы.
    Надо показать, что $\|A^n\| \ge \|A\|^n$.
    Для $n = 1$ очевидно верно.
    Если для $k = 1 \dots n$ имеем $\|A^k\| = \|A\|^k$ и $A \ne 0$, тогда 
    без ограничения общности $\forall f: \|f\| = 1$
    $$
    \|A^n f\|^2 = (A^n f, A^n f)
    $$
    В силу того, что $A$ --- самосопряжённый оператор
    $$
    (A^n f, A^n f) = (A^{n - 1} f, A^{n + 1} f)
    $$
    Последнее в силу неравенства Коши-Буняковского
    $$
    (A^{n - 1} f, A^{n + 1} f) \le \|A^{n - 1}f\| \|A^{n + 1} f\| 
    \le \|A^{n - 1}\| \|A^{n + 1}\|
    $$
    Получаем, что
    $$
    \|A^n f\|^2 \le \|A^{n - 1}\| \|A^{n + 1}\|
    $$
    Из индукции $\|A^{n - 1}\| \le \|A\|^{n -1}$ и $\|A^n\| = \|A\|^n$
    Возмём теперь супремум $\forall \|f\| = 1$
    $$
    \|A^n\|^2 = \|A\|^{2n} \le \|A\|^{n - 1} \|A^{n + 1}\|
    $$
    Отсюда при $A \ne 0$ получаем
    $$
    \|A\|^{n + 1} \le \|A^{n + 1}\|
    $$
    Что и требовалось доказать.
%=============================================================================!
    
    Второе утверждение очевидно следует из первого для самосопряжённого
    оператора:
    $$
    r(A) = \lim_{n \to \infty} \sqrt[n]{\|A^n\|} = \lim_{n \to \infty}
    \sqrt[n]{\|A\|^n} = \|A\|
    $$

    Перейдём к третьему утверждению.
    Воспоминание: пусть $A$ --- самосопряжённый оператор.
    Пусть $\lambda \in \sigma_p(A)$, то есть $\ker \Al \ne 0$, следовательно
    $\exists f \ne 0 \in \ker \Al\colon Af = \lambda f$.
    Тогда 
    $$
    (Af, f) = \lambda (f, f)
    $$
    С другой стороны, в силу самосопряжённости
    $$
    (Af, f) = (f, Af) = \overline{\lambda}(f, f)
    $$
    И так как $f \ne 0$, то $\lambda = \overline{\lambda}$
\end{Proof}
\end{document}
