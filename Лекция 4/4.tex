\documentclass[12pt]{article}
\usepackage[russian]{babel}
\usepackage[utf8]{inputenc}
\frenchspacing
\usepackage{amssymb, amsmath, amscd}
\usepackage[left=20mm, right=20mm, top=20mm, bottom=20mm]{geometry}
\usepackage{comment}
\usepackage{../theorem}
\renewcommand{\Im}{\operatorname{Im}}
\DeclareMathOperator{\Lin}{Lin}
\newcommand{\Al}{A_\lambda}
\newcommand{\Alo}{(\Al)^{-1}}
\newcommand{\Rez}{(I - \lambda A)^{-1}}
\begin{document}
\newtheorem{Theor}{Теорема}
\newtheorem{Utv}{Утверждение}
\newtheorem{Opr}{Определение}
\newtheorem{Upr}{Упражнение}
\newtheorem{Nabl}{Наблюдение}
\newtheorem{Zam}{Замечание}
%=============================================================================!
\section*{Свойства самосопряжённых линейных операторов в гильбертовом 
пространстве}
Пусть $H$ --- гильбертово пространство.
Пусть $A : H \mapsto H$ --- линейный и непрерывный оператор.
Пусть $A$ --- самосопряжённый оператор, то есть $A = A^*$.
\begin{Utv}
    Если $A$ --- самосопряжённый оператор, тогда выполняются следующие
    свойства:
    \begin{enumerate}
        \item{$\|A^n\| = \|A\|^n\; \forall n \in \mathbb N$}
        \item{$r(A) = \|A\|$}
        \item{$\sigma(A) \subset \mathbb R$}
        \item{пусть $m_+ = \sup \limits_{\|f\| = 1} (Af, f)$, где $(Af, f) \in 
                \mathbb R$ для самосопряжённого оператора, пусть $m_- = \inf 
                \limits_{\|f\| = 1} (Af, f)$, тогда $m_{+-} \in \sigma(A)$ и
                $\sigma(A) \subset [m_-(A), m_+(A)]$ (доказать в качестве
            упражнения)}
    \end{enumerate}
\end{Utv}
\begin{Proof}
    Докажем первое утверждение.
    $\|A^n\| \le \|A\|^n$ по определению операторной нормы.
    Надо показать, что $\|A^n\| \ge \|A\|^n$.
    Для $n = 1$ очевидно верно.
    Если для $k = 1 \dots n$ имеем $\|A^k\| = \|A\|^k$ и $A \ne 0$, тогда 
    без ограничения общности $\forall f: \|f\| = 1$
    $$
    \|A^n f\|^2 = (A^n f, A^n f)
    $$
    В силу того, что $A$ --- самосопряжённый оператор
    $$
    (A^n f, A^n f) = (A^{n - 1} f, A^{n + 1} f)
    $$
    Последнее в силу неравенства Коши-Буняковского
    $$
    (A^{n - 1} f, A^{n + 1} f) \le \|A^{n - 1}f\| \|A^{n + 1} f\| 
    \le \|A^{n - 1}\| \|A^{n + 1}\|
    $$
    Получаем, что
    $$
    \|A^n f\|^2 \le \|A^{n - 1}\| \|A^{n + 1}\|
    $$
    Из индукции $\|A^{n - 1}\| \le \|A\|^{n -1}$ и $\|A^n\| = \|A\|^n$
    Возмём теперь супремум $\forall \|f\| = 1$
    $$
    \|A^n\|^2 = \|A\|^{2n} \le \|A\|^{n - 1} \|A^{n + 1}\|
    $$
    Отсюда при $A \ne 0$ получаем
    $$
    \|A\|^{n + 1} \le \|A^{n + 1}\|
    $$
    Что и требовалось доказать.

    Второе утверждение очевидно следует из первого для самосопряжённого
    оператора:
    $$
    r(A) = \lim_{n \to \infty} \sqrt[n]{\|A^n\|} = \lim_{n \to \infty}
    \sqrt[n]{\|A\|^n} = \|A\|
    $$

    Перейдём к третьему утверждению.
    Воспоминание: пусть $A$ --- самосопряжённый оператор.
    Пусть $\lambda \in \sigma_p(A)$, то есть $\ker \Al \ne 0$, следовательно
    $\exists f \ne 0 \in \ker \Al\colon Af = \lambda f$.
    Тогда 
    $$
    (Af, f) = \lambda (f, f)
    $$
    С другой стороны, в силу самосопряжённости
    $$
    (Af, f) = (f, Af) = \overline{\lambda}(f, f)
    $$
    И так как $f \ne 0$, то $\lambda = \overline{\lambda}$.
    Перейдём теперь непосредственно к доказательству.
    Рассмотрим $\lambda = \mu + \imath \nu,\; \nu \ne 0$ и докажем, что
    $\lambda \in \rho(A)$ и тогда, очевидно, $\sigma(A)
    = \mathbb C \setminus \rho(A) \subset \mathbb R$.
    $\forall f \in H$
    $$
    \|A_\lambda f\|^2 = \|A_\mu f - \imath \nu f\|^2 = (A_\mu f - \imath \nu
    f, A_\mu f - \imath \nu f) = \|A_\nu f\|^2 + 
    \nu^2 \|f\|^2 - \imath \nu (f, A_\mu) + \imath \nu (A_\mu f, f)
    $$
    где $A_\lambda f = Af - \lambda f$.
    Раз $\mu \in \mathbb R$, то 
    $$
    (A_\mu)^* = A^*_{\overline{\mu}} = A^*_\mu = A_\mu
    $$
    так как $A^* = A$.
    Следовательно
    $$
    (f, A_\mu f) = (A_\mu f, f)
    $$
    Отсюда получаем, что
    $$
    (A_\mu f - \imath \nu f, A_\mu f - \imath \nu f) = \|A_\mu f\|^2 + \nu^2
    \|f\|^2
    $$
    Тогда
    $$
    \|A_\lambda f\|^2 = \|A_\mu f\|^2 + \nu^2\|f\|^2 \ge \nu^2 \|f\|^2
    $$
    Получили оценку снизу:
    $$
    \|A_\lambda f\| \ge |\nu| \|f\| \quad \forall f \in H
    $$
    Пусть теперь $f \in \ker A_\lambda$.
    Это равносильно $A_\lambda f = 0 \Rightarrow 0 \ge |\nu|\|f\| \ge
    0$,отсюда $f = 0$.
    По теореме Фредгольма 
    $$
    \overline{(\Im A_\lambda)} = (\ker(A_\lambda)^*)^\perp = (\ker
    A_{\overline
    {\lambda}})^\perp = 0^\perp = H
    $$
    так как в силу самосопряжённости $(A_\lambda)^* = A^*_{\overline{\lambda}} 
    = A_{\overline{\lambda}}$.
    Аналогично как для $A_\lambda$
    $$
    \|A_{\overline{\lambda}}\| \ge |\nu| \|f\| \Rightarrow \ker
    A_{\overline{\lambda}} = 0
    $$
    
    Покажем теперь, что $\Im A_\lambda$ замкнут.
    Пусть $\|A_\lambda\|f = g$, тогда $\Alo g =f$, тогда
    $$
    \|\Alo g\| \le \frac{\|g\|}{\nu} \quad \forall g \in \Im \Al
    $$
    Поэтому
    $$
    \|\Alo\| \le \frac{1}{\nu}
    $$
    Возмём $\forall g \in \overline{\Im \Al}$, тогда $\exists \{g_n\} \in \Im
    \Al\colon g_n \to g$ в $H$.
    $g_n = \Al f_n$
    $$
    \|f_n - f_m\| \le \frac{1}{\nu}\|g_n - g_m\|
    $$
    где $\|g_n - g_m\| \to 0\; n,m \to \infty$, отсюда $f_n - f_m \to 0\; n,m 
    \to \infty$.
    Так как $\{f_n\}$ --- фундаментальная последовательность в полном 
    пространстве $H$, то $\exists f \in H\colon f_n \to f$ 
    в $H$.
    Так $g_n \to g\; n \to \infty$ и так как $\Al$ --- непрерывный оператор, 
    то $g = \Al f$ и $g \in \Im \Al$.
    Таким образом получаем
    $$
    \left\{
        \begin{aligned}
            \Im \Al = H\\
            \ker \Al = 0
        \end{aligned}
   \right.
   $$
   поэтому $\exists \Alo : H \mapsto H$ такой, что $\|\Alo\| \le \frac{1}{|
   \nu|}$ и $\nu = \Im \lambda \ne 0$.
\end{Proof}
\nopagebreak
\begin{Prim}
    Пусть $H = L_2[0,1]$.
    Пусть задан оператор $A : H \mapsto H$ такой, что
    $$
    (A f)(x) = x f(x) \quad 0 \le x \le 1, f \in H
    $$
    Норма этого оператора будет
    $$
    \|Af\| = \sqrt{\int\limits_0^1 x^2 |f|^2 dx} \le \|f\|_{L_2}
    $$
    так как $x^2 \le 1$.
    Отсюда следует, что $\|A\| \le 1$.
    Рассмотрим функцию $f_\varepsilon \in H$ такую:
    $$
    f_\varepsilon =
    \left\{
        \begin{aligned}
            0 \quad 0 \le x \le \varepsilon\\
            1 \quad \varepsilon \le x \le 1
        \end{aligned}
    \right.
    $$
    Тогда получаем, что
    $$
    1 \ge \|A\| \ge \frac{\|A f_\varepsilon\|_{L_2}\|}{\|f_\varepsilon\|
    _{L_2}}
    $$
    Выписываем явный вид нормы
    $$
    \|A\| \ge \frac{\sqrt{\int\limits_{1 - \varepsilon}^1 x^2 dx}}{\sqrt{\int
    \limits_{1-\varepsilon}^1 dx}} \ge (1  - \varepsilon)
    \frac{\sqrt{\int\limits_{1 - \varepsilon}^1 dx}}{\sqrt{\int\limits_{1 - 
    \varepsilon}^1 dx}}
    $$
    В итоге мы получили
    $$
    1 \ge \|A\| \ge 1 - \varepsilon \to 1 \quad \varepsilon \to +0
    $$
    Покажем теперь, что этот оператор самосопряжённый
    $$
    (Af, g) = \int\limits_0^1 x f \overline{g} dx = \int \limits_0^1 f 
    \overline{xg} dx = (f, xg) = (f, A^*g)
    $$
    Поэтому $A^*g = xg =Ag\; \forall g \in H$.
    Рассмотрим теперь $\forall \lambda \in \mathbb C\colon \lambda \ne [0,1] 
    \subset \mathbb R$.
    Тогда
    $$
    \Al f = g = (x - \lambda)f
    $$
    И тогда
    $$
    \Alo g = f(x) = \frac{g(x)}{x -\lambda} \in L_2[0,1] \quad x \in [0,1]
    $$
    Найдём норму $f$:
    $$
    \|f\| = \sqrt{\int \limits_0^1 \frac{|g|^2}{|x - \lambda|^2}dx} \le 
    \rho(\lambda, [0,1] \|g\|
    $$
    где $0 \le \rho(\lambda, [0, 1]) \le |x - \lambda|\; \forall x \in [0, 1]$ 
    --- расстояние, причём $\rho(\lambda, [0, 1]) = 0 
    \Leftrightarrow \lambda \in [0,1]$.
    Таким образом 
    $$
    \|\Alo\| \le \frac{1}{\rho(\lambda, [0, 1])}
    $$
    И образ естественно всё пространство так как $g$ --- любой объект из $H$.
    Таким образом непрерывность очевидна, определена на всём пространстве и 
    поэтому за пределами этого отрезка спектра
    нет.
    $\forall \lambda \in [0,1] \Rightarrow \lambda \in \sigma(A)$, причём $
    \lambda \notin \sigma_P(A)$ так как иначе
    $Af = \lambda f, f \ne 0$ в $H$.
    $x f(x) = \lambda f(x)$ для почти всех $x \in [0,1]$, $\|f\| > 0$ на 
    множестве положительной меры $S$. 
    Получаем $x = \lambda$ почти всюду на $S$.
    Но мера $S > 0$, то есть не одно число, поэтому $x = \lambda$ невозможно.
    
    Покажем, что $\forall \lambda \in [0,1]$ выполняется $\lambda \in 
    \sigma(A)$ тогда $\ker \Al = 0$ и $g \in \Im \Al 
    \Leftrightarrow (x - \lambda)f(x) = g(x)$, то есть $x \in [0,1]$.
    Тогда $f(x) = \frac{g(x)}{x - \lambda} \in L_2[0,1]$ для почти всех $x \in 
    [0, 1] \setminus \{\lambda\}$.
    Но это справедливо не для всякой $g$.
    Пусть $g(x) = 1$, тогда $\frac{1}{x - \lambda} \notin L_2[0,1]$, поэтому 
    $1 \notin \Im A \Rightarrow \Im A \ne H$, 
    Значит это элемент спектра $\sigma(A) = \sigma_C(A)$.
    Так как оператор самосопряженный, то для $\lambda \in \mathbb R$ $\ker 
    \Al^* = \ker \Al = 0$ и получается, что
    $\overline{\Im \Al} = (\ker \Al)^\perp = 0^\perp = H$. 
    Таким образом для $\lambda \in [0,1]$ оператор $\Alo : \Im \Al \mapsto 
    L_2[0,1]$, где $\Im \Al = \{g \in L_2[0,1] \mid
    \frac{g(x)}{x - \lambda} \in L_2[0,1]\}$. 
    $\Alo g = \frac{g}{x - \lambda}$.
    Но норма такого оператора $\|\Alo\| = +\infty$ (доказать в качестве 
    упражнения): надо подобрать функции такие, что
    $\forall n \in \mathbb N\; \exists g_n \in \Im \Al$
    $$
    \frac{\Alo g_n\|}{\|g_n\|} \ge n
    $$
    Если бы $\|\Alo\| \le +\infty$, то получили ту же самую оценку:
    $$
    \|f\| = \|\Alo g\| \le \|\Alo\| \|g\|
    $$
    для любого $f$ из $H$, причём $g = \Al f$.
    Отсюда
    $$
    k\|f\| = \frac{\|f\|}{\|\Alo\|} \le \|\Al f\|
    $$
    гдe $k = \frac{1}{\|\Alo\|}$.
    И отсюда можно доказать, если норма образа больше или равна константы на 
    норму прообраза, то мгновенно
    доказывается, что $\Im \Al$ замкнут, что не так, так как замыкание образа 
    совпадает с $H$, сам образ не совпадает.
\end{Prim}

Ещё одно свойство самосопряжённого оператора:
\begin{Utv}
    Если $A : H \mapsto H$ --- линейный и непрерывный самосопрежённый оператор 
    и $\lambda, \mu \in \sigma(A)\colon
    \lambda \ne \mu$, тогда $\ker \Al \perp \ker A_\mu$
\end{Utv}
\begin{Proof}
    Утверждение нетривиально для $\lambda, \mu \in \sigma_P(A)$ так как если 
    одно из них из непрерывного спектра, то
    соответствующее ядро тривиально, а нулевое подпространство перпендикулярно 
    чему угодно.
    Пусть $f \ne 0 \in \ker \Al \ne 0$ и $g \ne 0 \in \ker A_\mu \ne 0$, тогда 
    надо доказать, что $f \perp g$, то есть $(f, g) = 0$.
    В силу пункта 3 предыдущего утверждения $\lambda, \mu \in \mathbb R$.
    Тогда $A f = \lambda f$ и $A g = \mu g$.
    Получается
    $$
    (Af, g) = \lambda (f, g)
    $$
    с другой стороны
    $$
    (Af, g) = (f, Ag) = (f, \mu g) = \overline{\mu}(f, g) = \mu (f, g)
    $$
    Отсюда получаем, что
    $$
    (\lambda - \mu)(f, g) = 0
    $$
    И отсюда $(f, g) = 0$.
    Утверждение доказано.
\end{Proof}
\begin{Upr}[решается, например, с помощью теоремы Банаха-Штейнгаусса, (теорема 
Хеллингера, Теплица)]
    Пусть $A : H \mapsto H$ --- линейный и симметричный оператор, то есть $(Af 
    ,g) = (f, AG)\; \forall f,g \in H$.
    Тогда $A$ --- непрерывный оператор ($\|A\| < +\infty$).
\end{Upr}
\section*{Теория компактных операторов в гильбертовом пространстве.}
\begin{Opr}
    $A : H \mapsto H$ --- линейный оператор называется компактным (или вполне 
    непрерывным), если $\forall \{f_n\} \subset
    H$ $\{f_n\}$ ограничена в $H$.
    Иначе говоря $\forall n\; \|f_n\| \le R \Rightarrow \exists \{Af_{n_k}\}$ 
    --- фундаментальная подпоследовательность в $H$, 
    где $n_1 < n_2 < \dots$
\end{Opr}
\begin{Utv}
    $A$ --- компактный, тогда $A$ --- непрерывный.
    Обратное не верно. 
\end{Utv}
\begin{Prim}
    Пусть $U$ --- унитарный оператор, то есть $U : H \mapsto H$ и $\|Uf\| = \|
    f\|\; \forall f \in H$ в
    бесконечномерном гильбертовом пространстве $H$.
    Очевидно, что $\|U\| = 1$, он он не компактный.
    Так как $\dim H = +\infty$, то существует линейно независимая 
    последовательность $\{f_n\}_{n = 1}^{\infty}$.
    Ортагонализируем её по Грамму-Шмитду и получим $\{e_n\}_{n = 1}^{\infty}
    $,тогда
    $$
    (U e_n, U e_m) = (e_n, e_m) = 0
    $$
    $\|U e_n\| = 1$, тогда
    $$
    \|U e_n - U e_m\| = \sqrt{\|U e_n\|^2 + \|U e_m\|^2} = \sqrt{2}
    $$
    для любых $n \ne m$, поэтому нет фундаментальной подпоследовательности и 
    $U$ не компактный.
\end{Prim}
\begin{Upr}
    Пусть оператор $A f(x) = x f(x)$ в $L_2[0,1]$.
    $A$ --- самосопряжённный оператор.
    Показать, что $A$ --- не компактный оператор.
\end{Upr}

Докажем теперь утверждение
\begin{Proof}
    Если вдруг $A$ --- компактный и $\|A\| = +\infty$, тогда $\exists \{f_n\}$ 
    с единичной сферы такая, что $\|A f_n\| \to 
    +\infty$.
    Из компактности следует, что $\exists A f_{n_k}$ фундаментальная, то есть 
    $A f_{n_k} \le R$, а с другой стороны 
    $\|A f_{n_k}\| \to +\infty$.
    Получаем противоречие.
\end{Proof}
 
\begin{Utv}
    Пусть $A_1$ и $A_2$ --- компактные операторы.
    Тогда их линейная комбинация также компактна (доказательство очевидно).
    Если $\{A_n\}_{n = 1}^\infty$ --- последовательность компактных операторов 
    и $A_n \to T$ по операторной норме, тогда
    $T$ также компактный.
\end{Utv}
\begin{Proof}
    Возмём последовательность $f_n$ ограниченную в $H$ и $\|f_n\| \le R$.
    Рассмотрим $\{A_1 f_n\}$ --- это действие компактного оператора на 
    ограниченную последовательность, тогда по
    определению $\exists \{n_k(1)\}_{k = 1}^\infty \colon n_1(1) < n_2(1) < 
    \dots \Rightarrow A_1 f_{n_k(1)}$ ---
    фундаментальная.
    Дальше действуем оператором $A_2$, тогда $\exists \{n_k(2)\} \subset 
    \{n_k(1)\} \colon n_1(2) < n_2(2) < \dots 
    \Rightarrow A_2 f_{n_k(2)}$ --- фундаментальная, причём $A_1 f_{n_k(2)}$ 
    осталась фундаментальной как
    подпоследовательность фундаментальной.
    Будем действовать так дальше.
    Реализуем Канторов диагональный процесс.
    Если для $m \in \mathbb N$ имеем $\{n_k(m)\} \subset \{n_k(m - 1)\} 
    \subset \dots \subset \{n_k(1)\} \subset \mathbb N$
    так что $\{A_s f_{n_k(m)}\}_{k = 1}^{\infty}$ --- фундаментальная $\forall 
    s = 1 \dots m$.
    Тогда рассмотрим $\{A_{m + 1} f_{n_k(m)}\}$, тогда, используя 
    компактность, $\exists \{n_k(m + 1)\} \subset \{n_k(m)\}$
    такой, что $n_1(m + 1) < n_2(m + 1) < \dots$ такая, что $\{A_{m + 1} 
    f_{n_k(m + 1)}\}$ фундаментальная.
    Остальные же сохранили фундаментальность, так как мы перешли к 
    подпоследовательности фундаментальной 
    последовательности, которая также фундаментальна.
    Таким образом мы по индукции реализовали счётный набор последовательностей 
    натуральных чисел.
    Рассмотрим $\{n_k(k)\}_{k = 1}^\infty$.
    $n_{k + 1} \ge n_{k + 1}(k)$ по построению и $n_{k + 1}(k) > n_k(k)$ по 
    определению, тогда $\{n_k(k)\}_{k = 1}^\infty$
    строго возрастающая последовательность чисел.
    Тогда $\{f_{n_k(k)}\}$ --- фундаментальная подпоследовательность.
    Рассмотрим действие оператора $T$ на эту подпоследовательность.
    Надо доказать, что мало при большом $p$
    $$
    \|T f_{n_k(k)} - T f_{n_{k + p}(k + p)}\|
    $$
    Добавим умный ноль
    $$
    \|T f_{n_k(k)} -T f_{n_{k + p}(k + p)} \pm A_s f_{n_k(k)} \pm 
    A_s f_{n_{k + p}(k + P}\|
    $$
    Какое $s$ надо взять?
    $\|f_n\| \le R$ по условию, тогда 
    $$
    \|f_n\| \le R\; \forall n \colon \forall
    \varepsilon > 0\; \exists S(\varepsilon)\colon \forall s \ge 
    S(\varepsilon)\; \|T - A_s\| \le \dfrac{\varepsilon}{R + 1}
    $$
    $A_s f_{n_k(m)}$ --- фундаментальна по $k$ если $s \ge m$.
    Возмём $s = S(\varepsilon)$ и $k \ge S(\varepsilon)$.
    Следовательно получается, что $A_s f_{n_k(k)}$, если $k \ge S(\varepsilon)
    $ --- подпоследовательность $A_s f_{n_k(S(\varepsilon))}$, которая
    фундаментальная и значит наша подпоследовательность также фундаментальна.
    Получаем, что 
    $$
    \|(T - A_{S(\varepsilon)})f_{n_k(k)}\| \le \frac{\varepsilon}{R + 1}
    $$ 
    Аналогично
    $$
    \|(T - A_{S(\varepsilon)})f_{n_{k + p}(k + p)}\| \le \frac{\varepsilon}{
    R + 1}
    $$
    А из фундаментальности
    $$
    \|A_{S(\varepsilon)}f_{n_k(k)} - A_{S(\varepsilon)}f_{n_{k + p}(k + p)}
    \| \le \varepsilon
    $$
    Итого имеем
    \begin{multline*}
        \|T f_{n_k(k)} - T f_{n_{k + p}(k + p)}\| \le \\
        \le \|(T - A_{S(\varepsilon)})f_{n_k(k)}\| + 
        \|(T - A_{S(\varepsilon)})f_{n_{k + p}(k + p)}\| +
        \|A_{S(\varepsilon)}f_{n_k(k)} - A_{S(\varepsilon)}
        f_{n_{k + p}(k + p)}\| \le \\
        \le \frac{\varepsilon}{R + 1} + \frac{\varepsilon}{R + 1} + 
        \varepsilon \le 3 \varepsilon
    \end{multline*}
    Отсюда образ $\|T\|$ содержит фундаментальную подпоследовательность и
    $\|T\|$ компактен.
\end{Proof}

Докажем ещё одно свойство компактного оператора.
\begin{Utv}
    Пусть $A : H \mapsto H$ --- компактный оператор и $T : H \mapsto H$ --- 
    непрерывный оператор, тогда $TA$ и $AT$ --- тоже компактный оператор.
\end{Utv}
\begin{Proof}
    Пусть $\{f_n\}$ --- ограниченная последовательность ,тогда 
    $A f_{n_k}$ --- фундаментальная подпоследовательность.
    Смотрим на $T A f_{n_k}$ и получаем
    $$
    \|T A f_{n_k} - T A f_{n_{k + p}}\| \le \|T\|
    \|A f_{n_k} - A f_{n_{k + p}}\|
    $$
    стремиться к нулю, так как $T$ непрерывный следовательно $T A f_{n_k}$
    тоже фундаментальная.
    Расмотрим теперь $A T f_n$.
    $T$ --- непрерывный оператор, $\|f_n\| \le R$ --- ограниченная в $H$, то
    $T f_n$ тоже ограниченная
    $$
    \|T f_n\| \le \|T\| \|f\| \le \|T\| \|R\|
    $$
    Отсюда $T f_n$ ограниченная и по определению $\exists n_k A(T f_{n_k})$
    фундаментальная.
\end{Proof}

\begin{Prim}
    Если $A : H \mapsto H$ линейный и непрерывный.
    Пусть $\dim \Im A < +\infty$ (такой $A$ называется конечномерным).
    Тогда $A$ --- компактный оператор.
    
    Действительно, пусть $f_n$ --- ограниченная последовательность, тогда
    $A f_n$ также ограничен в конечном $\Im A$.
    Тогда по теореме Больцано-Вейерштрасса $\exists A f_{n_k}$ фундаментальная
    в $\Im A$.
    
    Покажем вид этих операторов.
    Если $A$ --- линейный и непрерывный и его образ конечномерный, тогда возмём
    базис $g_1, \dots, g_N$ в $\Im A$, где $N = \dim \Im A$.
    Тогда $Af$ раскладывается
    $$
    Af = \sum\limits_{k = 1}^N\alpha_k(f)g_k
    $$
    где $\alpha_k$ линейный и непрерывный функционал.
    Если мы будем скалярно умножать
    $$
    (Af, g_m) = \sum \alpha_k(f) (g_k, g_m)
    $$
    И если мы введём матрицу Грамма
    $$
    \Gamma = ((g_k, g_m))_{k,m=1}^N
    $$
    Тогда получается, что это всё равно, что
    $$
    \left(
        \begin{array}{c}
            (Af, g_1)\\
            \dots\\
            (Af, g_N)
        \end{array}
    \right)
    =
    \Gamma \left(
        \begin{array}{c}
            \alpha_1(f)\\
            \dots\\
            \alpha_N(f)
        \end{array}
        \right)
    $$
    Матрица $\Gamma$ не вырожденена в силу линейной независимости 
    $g_1, \dots g_N$ и тогда
    $$
    \left(
        \begin{array}{c}
            \alpha_1(f)\\
            \dots\\
            \alpha_N(f)
        \end{array}
    \right)
    =
    \Gamma^{-1}\left(
        \begin{array}{c}
            (Af, g_1)\\
            \dots\\
            (Af, g_N)
        \end{array}
    \right)
    $$
    где правая часть непрерывна по $f$ в $H$.
    Отсюда $\alpha_k(f) : H \mapsto \mathbb C$ линеен и непрерывен, а значит
    по теореме Риса-Фреше
    $$
    \alpha_k(f) = (f, h_k)\quad \exists! h_k \in H
    $$
    Отсюда
    $$
    Af = \sum\limits_{k = 1}^N(f, h_k)g_k
    $$
    где $h_k, g_k \in H$ и $k = 1 \dots N$
\end{Prim}
\begin{Prim}
    В случае $H = L_2(G)$, тогда $h_k(x), g_k(x) \in L_2(G)$ и
    $$
    Af = \sum\limits_{k = 1}^N \int\limits_G f(y)\overline{h_k(y)} \mathop{dy} 
    g_k(x) 
    $$
    Иначе
    $$
    Af = \int\limits_G \left(\sum\limits_{k = 1}^N \overline{h_k(y)} g_k(x)
    \right) f(y)\mathop{dy}
    $$
\end{Prim}

Если взять предел последовательности конечномерных операторов $\{A_n\}$ такие,
что $A_n \to T$ по операторной норме, тогда $T$ также будет компактным.
\begin{Utv}
    $T$ --- компактный оператор в гильбертовом пространстве $H$, тогда 
    $\forall \varepsilon > 0\; \exists A_\varepsilon$ --- конечномерные
    операторы
    и $\|T - A_\varepsilon\| \le \varepsilon$
\end{Utv}

В 1972 году датский математик Энфло доказал, что существуют $X$ --- полные
линейные непрерывные пространства и $A : X \mapsto X$ компактный оператор, не
апроксимирующийся конечномерым.
\begin{Proof}
    Посмотрим на множество $TB_1(0)$ --- образ шара, $B_1(0) = \{f \in H \mid
    \|f\| \le 1\}$.
    Любая последовательность $Tf_n$, где $f_n \in B_1(0)$ имеет фундаментальную 
    подпоследовательность $T f_{n_k}$, тогда утверждается, что $\forall 
    \varepsilon > 0\; \exists g_1,\dots,g_N \in TB_1(0)\colon TB_1(0) \subset
    \bigcup\limits_{k = 1}^N B_\varepsilon(g_k)$, где $B_\varepsilon(g_k)$ 
    называют эпсилон-сетью.
    
    Если вдруг $\exists \varepsilon_0\colon \forall g_1,\dots,g_N \in TB_1(0)
    \colon TB_1(0) \nsubseteq \bigcup\limits_{k = 1}^N B_\varepsilon(g_k)$.
    Пусть $f_1 \in B_1(0)$ и $g_1 = T f_1$, тогда $B_{\varepsilon_0}(g_1)
    \nsupseteq TB_1(0)$, следовательно $\exists g_2 \in TB_1(0)\setminus 
    B_{\varepsilon_0}(g_1)$, а значит $\exists f_2 \in B_1(0) \colon g_2 =
    Tf_2$ и так далее.
    Если есть $f_1 \dots f_n \in B_1(0)$, таких, что
    $$
    \|T f_k - T f_s\| \ge \varepsilon_0
    $$
    где $1 \le k \ne s \le n$.
    То эти функции порождают набор $g_k = T f_k$ и эти функции после 
    объединения шаров не вместят образ шара, а значит существует $f_{n + 1}
    \in B_1(0)\colon Tf_{n + 1} \nsubseteq B_{\varepsilon_0}(g_k)$.
    Таким образом
    $$
    \exists\{f_k\}_{k = 1}^\infty \subset B_1(0) \colon \|T f_k - T f_s\| \ge
    \varepsilon_0
    $$
    а значит нет фундаментальной подпоследовательности в $\{T f_k\}_{k = 1}^
    \infty$, хотя $f_k$ ограничена.
    Получаем противоречие.
    
    Пользуясь этой эпсилон-сетью займёмся апроксимацией.
    Пусть $L_N = \Lin \{g_1,\dots g_N\}$ --- конечномерное пространство, а
    значит замкнутое в $H$.
    Следовательно $\exists P_N$ --- ортопроектор из $H$ на $L_N$ и 
    $L_N \oplus L_N^\perp = H$ и соответственно $P_N h = h_N \in L_N$.
    Норма $\|P_N\| = 1$.
    Рассмотрим теперь оператор $A_\varepsilon = P_N T$.
    Образ $\Im A_\varepsilon \subset L_N$ и
    $$
    A_\varepsilon h = P_N T h \quad \forall n \in H
    $$
    поэтому $A_\varepsilon$ конечномерен и непрерывен как суперпозиция 
    непрерывных.
    Возмём $\forall f\colon \|f\| = 1$, $Tf \in TB_1(0)$, значит
    $\exists g_k\colon \|Tf - g_k\| \le \varepsilon$ и отсюда
    \begin{multline*}
    \|Tf - A_\varepsilon f\| = \|Tf + g_k - g_k - A_\varepsilon f\| \le \\ \le
    \varepsilon + \|g_k - A_\varepsilon\| = \varepsilon + \|P_N g_k - P_N T f\|
    \le \\ \le\varepsilon  \|P_N(g_k - T_f)\| \le \varepsilon + \|g_k - Tf\| 
    \le 2 \varepsilon 
    \end{multline*}
\end{Proof}
\begin{Prim}
    Пусть $H = L_2(G)$, где $G$ --- компакт в $\mathbb R^m$ и $A : L_2(G)
    \mapsto L_2(G)$ имеет вид
    $$
    (Af)(x) = \int\limits_G K(t, x)f(t)\mathop{dt}
    $$
    где $K \in L_2(G \times G)$.
    Ранее получено, что $\|A\| \le \|K\|_{L_2(G \times G)}$.
    Утверждается, что $A$ компактен.
    $L_2(G \times G)$ --- пополнение $C(G \times G)$, $C$ --- компакт с нормой
    $L_2$.
    Тогда $\forall \varepsilon > 0 \exists F_\varepsilon(t, x) \in
    C(G \times G)\colon \|K - F_\varepsilon\|_{L_2(G \times G)} \le 
    \varepsilon$.
    \begin{Theor}[Стоун, Вейерштрасс (см. Рудин <<Основы математического 
    анализа>>)]
        Пусть $D \subset \mathbb R^N$ --- компакт функция $F \in C(D)$, тогда
        $\forall \varepsilon > 0\; \exists P : \mathbb R^N \mapsto \mathbb C$
        комплексный многочлен $n$ переменных такой, что
        $\|F - P\|_C = \max\limits_{x \in D} |F(x) - P(x)| \le \varepsilon$
    \end{Theor}
    
    Применяем теорему Стоуна-Вейерштрасса: $\exists P$ многочлен
    $$
    \max_{G \times G}|F_\varepsilon - P| \le \frac{\varepsilon}{\mu(G \times 
    G) + 1}
    $$
    следовательно
    $$
    \|F_\varepsilon - P\|_{L_2(G \times G)} \le \max_{G \times G}|F_\varepsilon 
    - P|\sqrt{\mu(G \times G)} \le \varepsilon
    $$
    Таким образо мы строим оператор
    $$
    A_\varepsilon f = \int \limits_G P(t, x) f(t) \mathop{dt}
    $$
    Тогда $\Im A_\varepsilon$ конечномерен и
    $$
    \|A_\varepsilon - A\| \le \|K - P \pm F_\varepsilon\|_{L_2{G \times G}} \le
    2 \varepsilon
    $$
\end{Prim}
\begin{Upr}
    Пусть $A : L_2(G) \mapsto L_2(\mathbb R)$ имеет вид
    $$
    Af(x) = \int\limits_{\mathbb R} e^{-(t - x)^2}f(t)\mathop{dt} = (e^{-x^2} *
    f)(x)
    $$
    Тогда $\|A\| \le \|e^{-x^2}\|_{L_2(\mathbb R)}$. 
    Показать, что $A$ не компактный оператор.
\end{Upr}
\end{document}