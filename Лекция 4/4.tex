\documentclass[12pt]{article}
\usepackage[russian]{babel}
\usepackage[utf8]{inputenc}
\frenchspacing
\usepackage{amssymb, amsmath, amscd}
\usepackage[left=20mm, right=20mm, top=20mm, bottom=20mm]{geometry}
\usepackage{comment}
\usepackage{theorem}
\renewcommand{\Im}{\operatorname{Im}}
\DeclareMathOperator{\Lin}{Lin}
\newcommand{\Al}{A_\lambda}
\newcommand{\Alo}{(\Al)^{-1}}
\newcommand{\Rez}{(I - \lambda A)^{-1}}
\begin{document}
\newtheorem{Theor}{Теорема}
\newtheorem{Utv}{Утверждение}
\newtheorem{Opr}{Опреление}
\newtheorem{Prim}{Пример}
\newtheorem{Upr}{Упражнение}
\newtheorem{Nabl}{Наблюдение}
\newtheorem{Zam}{Замечание}
%=============================================================================!
\section*{Свойства самосопряжённых линейных операторов в гильбертовом
пространств}
Пусть $H$ --- гильбертово пространство.
Пусть $A : H \mapsto H$ --- линейный и непрерывный оператор.
Пусть $A$ --- самосопряжённый оператор, то есть $A = A^*$.
\begin{Utv}
    Если $A$ --- самосопряжённый оператор, тогда выполняются следующие
    свойства:
    \begin{enumerate}
        \item{$\|A^n\| = \|A\|^n\; \forall n \in \mathbb N$}
        \item{$r(A) = \|A\|$}
        \item{$\sigma(A) \subset \mathbb R$}
        \item{пусть $m_+ = \sup \limits_{\|f\| = 1} (Af, f)$, где $(Af, f) \in 
                \mathbb R$ для самосопряжённого оператора, пусть $m_- = \inf 
                \limits_{\|f\| = 1} (Af, f)$, тогда $m_{+-} \in \sigma(A)$ и
                $\sigma(A) \subset [m_-(A), m_+(A)]$ (доказать в качестве
            упражнения)}
    \end{enumerate}
\end{Utv}
\begin{Proof}
    Докажем первое утверждение.
    $\|A^n\| \le \|A\|^n$ по определению операторной нормы.
    Надо показать, что $\|A^n\| \ge \|A\|^n$.
    Для $n = 1$ очевидно верно.
    Если для $k = 1 \dots n$ имеем $\|A^k\| = \|A\|^k$ и $A \ne 0$, тогда 
    без ограничения общности $\forall f: \|f\| = 1$
    $$
    \|A^n f\|^2 = (A^n f, A^n f)
    $$
    В силу того, что $A$ --- самосопряжённый оператор
    $$
    (A^n f, A^n f) = (A^{n - 1} f, A^{n + 1} f)
    $$
    Последнее в силу неравенства Коши-Буняковского
    $$
    (A^{n - 1} f, A^{n + 1} f) \le \|A^{n - 1}f\| \|A^{n + 1} f\| 
    \le \|A^{n - 1}\| \|A^{n + 1}\|
    $$
    Получаем, что
    $$
    \|A^n f\|^2 \le \|A^{n - 1}\| \|A^{n + 1}\|
    $$
    Из индукции $\|A^{n - 1}\| \le \|A\|^{n -1}$ и $\|A^n\| = \|A\|^n$
    Возмём теперь супремум $\forall \|f\| = 1$
    $$
    \|A^n\|^2 = \|A\|^{2n} \le \|A\|^{n - 1} \|A^{n + 1}\|
    $$
    Отсюда при $A \ne 0$ получаем
    $$
    \|A\|^{n + 1} \le \|A^{n + 1}\|
    $$
    Что и требовалось доказать.

    Второе утверждение очевидно следует из первого для самосопряжённого
    оператора:
    $$
    r(A) = \lim_{n \to \infty} \sqrt[n]{\|A^n\|} = \lim_{n \to \infty}
    \sqrt[n]{\|A\|^n} = \|A\|
    $$

    Перейдём к третьему утверждению.
    Воспоминание: пусть $A$ --- самосопряжённый оператор.
    Пусть $\lambda \in \sigma_p(A)$, то есть $\ker \Al \ne 0$, следовательно
    $\exists f \ne 0 \in \ker \Al\colon Af = \lambda f$.
    Тогда 
    $$
    (Af, f) = \lambda (f, f)
    $$
    С другой стороны, в силу самосопряжённости
    $$
    (Af, f) = (f, Af) = \overline{\lambda}(f, f)
    $$
    И так как $f \ne 0$, то $\lambda = \overline{\lambda}$.
    Перейдём теперь непосредственно к доказательству.
    Рассмотрим $\lambda = \mu + \imath \nu,\; \nu \ne 0$ и докажем, что $\lambda \in \rho(A)$ и тогда, очевидно, $\sigma(A)
    = \mathbb C \setminus \rho(A) \subset \mathbb R$.
    $\forall f \in H$
    $$
    \|A_\lambda f\|^2 = \|A_\mu f - \imath \nu f\|^2 = (A_\mu f - \imath \nu f, A_\mu f - \imath \nu f) = \|A_\nu f\|^2 + 
    \nu^2 \|f\|^2 - \imath \nu (f, A_\mu) + \imath \nu (A_\mu f, f)
    $$
    где $A_\lambda f = Af - \lambda f$.
    Раз $\mu \in \mathbb R$, то 
    $$
    (A_\mu)^* = A^*_{\overline{\mu}} = A^*_\mu = A_\mu
    $$
    так как $A^* = A$.
    Следовательно
    $$
    (f, A_\mu f) = (A_\mu f, f)
    $$
    Отсюда получаем, что
    $$
    (A_\mu f - \imath \nu f, A_\mu f - \imath \nu f) = \|A_\mu f\|^2 + \nu^2\|f\|^2
    $$
    Тогда
    $$
    \|A_\lambda f\|^2 = \|A_\mu f\|^2 + \nu^2\|f\|^2 \ge \nu^2 \|f\|^2
    $$
    Получили оценку снизу:
    $$
    \|A_\lambda f\| \ge |\nu| \|f\| \quad \forall f \in H
    $$
    Пусть теперь $f \in \ker A_\lambda$.
    Это равносильно $A_\lambda f = 0 \Rightarrow 0 \ge |\nu|\|f\| \ge 0$,отсюда $f = 0$.
    По теореме Фредгольма 
    $$
    \overline{(\Im A_\lambda)} = (\ker(A_\lambda)^*)^\perp = (\ker A_{\overline{\lambda}})^\perp = 0^\perp = H
    $$
    так как в силу самосопряжённости $(A_\lambda)^* = A^*_{\overline{\lambda}} = A_{\overline{\lambda}}$.
    Аналогично как для $A_\lambda$
    $$
    \|A_{\overline{\lambda}}\| \ge |\nu| \|f\| \Rightarrow \ker A_{\overline{\lambda}} = 0
    $$
    
    Покажем теперь, что $\Im A_\lambda$ замкнут.
    Пусть $\|A_\lambda\|f = g$, тогда $\Alo g =f$, тогда
    $$
    \|\Alo g\| \le \frac{\|g\|}{\nu} \quad \forall g \in \Im \Al
    $$
    Поэтому
    $$
    \|\Alo\| \le \frac{1}{\nu}
    $$
    Возмём $\forall g \in \overline{\Im \Al}$, тогда $\exists \{g_n\} \in \Im \Al\colon g_n \to g$ в $H$.
    $g_n = \Al f_n$
    $$
    \|f_n - f_m\| \le \frac{1}{\nu}\|g_n - g_m\|
    $$
    где $\|g_n - g_m\| \to 0\; n,m \to \infty$, отсюда $f_n - f_m \to 0\; n,m \to \infty$.
    Так как $\{f_n\}$ --- фундаментальная последовательность в полном пространстве $H$, то $\exists f \in H\colon f_n \to f$ 
    в $H$.
    Так $g_n \to g\; n \to \infty$ и так как $\Al$ --- непрерывный оператор, то $g = \Al f$ и $g \in \Im \Al$.
    Таким образом получаем
    $$
    \left\{
        \begin{aligned}
            \Im \Al = H\\
            \ker \Al = 0
        \end{aligned}
   \right.
   $$
   поэтому $\exists \Alo : H \mapsto H$ такой, что $\|\Alo\| \le \frac{1}{|\nu|}$ и $\nu = \Im \lambda \ne 0$.
\end{Proof}
\nopagebreak
\begin{Prim}
    Пусть $H = L_2[0,1]$.
    Пусть задан оператор $A : H \mapsto H$ такой, что
    $$
    (A f)(x) = x f(x) \quad 0 \le x \le 1, f \in H
    $$
    Норма этого оператора будет
    $$
    \|Af\| = \sqrt{\int\limits_0^1 x^2 |f|^2 dx} \le \|f\|_{L_2}
    $$
    так как $x^2 \le 1$.
    Отсюда следует, что $\|A\| \le 1$.
    Рассмотрим функцию $f_\varepsilon \in H$ такую:
    $$
    f_\varepsilon =
    \left\{
        \begin{aligned}
            0 \quad 0 \le x \le \varepsilon\\
            1 \quad \varepsilon \le x \le 1
        \end{aligned}
    \right.
    $$
    Тогда получаем, что
    $$
    1 \ge \|A\| \ge \frac{\|A f_\varepsilon\|_{L_2}\|}{\|f_\varepsilon\|_{L_2}}
    $$
    Выписываем явный вид нормы
    $$
    \|A\| \ge \frac{\sqrt{\int\limits_{1 - \varepsilon}^1 x^2 dx}}{\sqrt{\int\limits_{1-\varepsilon}^1 dx}} \ge (1  - \varepsilon)
    \frac{\sqrt{\int\limits_{1 - \varepsilon}^1 dx}}{\sqrt{\int\limits_{1 - \varepsilon}^1 dx}}
    $$
    В итоге мы получили
    $$
    1 \ge \|A\| \ge 1 - \varepsilon \to 1 \quad \varepsilon \to +0
    $$
    Покажем теперь, что этот оператор самосопряжённый
    $$
    (Af, g) = \int\limits_0^1 x f \overline{g} dx = \int \limits_0^1 f \overline{xg} dx = (f, xg) = (f, A^*g)
    $$
    Поэтому $A^*g = xg =Ag\; \forall g \in H$.
    Рассмотрим теперь $\forall \lambda \in \mathbb C\colon \lambda \ne [0,1] \subset \mathbb R$.
    Тогда
    $$
    \Al f = g = (x - \lambda)f
    $$
    И тогда
    $$
    \Alo g = f(x) = \frac{g(x)}{x -\lambda} \in L_2[0,1] \quad x \in [0,1]
    $$
    Найдём норму $f$:
    $$
    \|f\| = \sqrt{\int \limits_0^1 \frac{|g|^2}{|x - \lambda|^2}dx} \le \rho(\lambda, [0,1] \|g\|
    $$
    где $0 \le \rho(\lambda, [0, 1]) \le |x - \lambda|\; \forall x \in [0, 1]$ --- расстояние, причём $\rho(\lambda, [0, 1]) = 0 
    \Leftrightarrow \lambda \in [0,1]$.
    Таким образом 
    $$
    \|\Alo\| \le \frac{1}{\rho(\lambda, [0, 1])}
    $$
    И образ естественно всё пространство так как $g$ --- любой объект из $H$.
    Таким образом непрерывность очевидна, определена на всём пространстве и поэтому за пределами этого отрезка спектра
    нет.
    $\forall \lambda \in [0,1] \Rightarrow \lambda \in \sigma(A)$, причём $\lambda \notin \sigma_P(A)$ так как иначе
    $Af = \lambda f, f \ne 0$ в $H$.
    $x f(x) = \lambda f(x)$ для почти всех $x \in [0,1]$, $\|f\| > 0$ на множестве положительной меры $S$. 
    Получаем $x = \lambda$ почти всюду на $S$.
    Но мера $S > 0$, то есть не одно число, поэтому $x = \lambda$ невозможно.
    
    Покажем, что $\forall \lambda \in [0,1]$ выполняется $\lambda \in \sigma(A)$ тогда $\ker \Al = 0$ и $g \in \Im \Al 
    \Leftrightarrow (x - \lambda)f(x) = g(x)$, то есть $x \in [0,1]$.
    Тогда $f(x) = \frac{g(x)}{x - \lambda} \in L_2[0,1]$ для почти всех $x \in [0, 1] \setminus \{\lambda\}$.
    Но это справедливо не для всякой $g$.
    Пусть $g(x) = 1$, тогда $\frac{1}{x - \lambda} \notin L_2[0,1]$, поэтому $1 \notin \Im A \Rightarrow \Im A \ne H$, 
    Значит это элемент спектра $\sigma(A) = \sigma_C(A)$.
    Так как оператор самосопряженный, то для $\lambda \in \mathbb R$ $\ker \Al^* = \ker \Al = 0$ и получается, что
    $\overline{\Im \Al} = (\ker \Al)^\perp = 0^\perp = H$. 
    Таким образом для $\lambda \in [0,1]$ оператор $\Alo : \Im \Al \mapsto L_2[0,1]$, где $\Im \Al = \{g \in L_2[0,1] \mid
    \frac{g(x)}{x - \lambda} \in L_2[0,1]\}$. 
    $\Alo g = \frac{g}{x - \lambda}$.
    Но норма такого оператора $\|\Alo\| = +\infty$ (доказать в качестве упражнения): надо подобрать функции такие, что
    $\forall n \in \mathbb N\; \exists g_n \in \Im \Al$
    $$
    \frac{\Alo g_n\|}{\|g_n\|} \ge n
    $$
    Если бы $\|\Alo\| \le +\infty$, то получили ту же самую оценку:
    $$
    \|f\| = \|\Alo g\| \le \|\Alo\| \|g\|
    $$
    для любого $f$ из $H$, причём $g = \Al f$.
    Отсюда
    $$
    k\|f\| = \frac{\|f\|}{\|\Alo\|} \le \|\Al f\|
    $$
    гдe $k = \frac{1}{\|\Alo\|}$.
    И отсюда можно доказать, если норма образа больше или равна константы на норму прообраза, то мгновенно
    доказывается, что $\Im \Al$ замкнут, что не так, так как замыкание образа совпадает с $H$, сам образ не совпадает.
\end{Prim}

Ещё одно свойство самосопряжённого оператора:
\begin{Utv}
    Если $A : H \mapsto H$ --- линейный и непрерывный самосопрежённый оператор и $\lambda, \mu \in \sigma(A)\colon
    \lambda \ne \mu$, тогда $\ker \Al \perp \ker A_\mu$
\end{Utv}
\begin{Proof}
    Утверждение нетривиально для $\lambda, \mu \in \sigma_P(A)$ так как если одно из них из непрерывного спектра, то
    соответствующее ядро тривиально, а нулевое подпространство перпендикулярно чему угодно.
    Пусть $f \ne 0 \in \ker \Al \ne 0$ и $g \ne 0 \in \ker A_\mu \ne 0$, тогда надо доказать, что $f \perp g$, то есть $(f, g) = 0$.
    В силу пункта 3 предыдущего утверждения $\lambda, \mu \in \mathbb R$.
    Тогда $A f = \lambda f$ и $A g = \mu g$.
    Получается
    $$
    (Af, g) = \lambda (f, g)
    $$
    с другой стороны
    $$
    (Af, g) = (f, Ag) = (f, \mu g) = \overline{\mu}(f, g) = \mu (f, g)
    $$
    Отсюда получаем, что
    $$
    (\lambda - \mu)(f, g) = 0
    $$
    И отсюда $(f, g) = 0$.
    Утверждение доказано.
\end{Proof}
\begin{Upr}[решается, например, с помощью теоремы Банаха-Штейнгаусса]
\end{Upr}
\end{document}