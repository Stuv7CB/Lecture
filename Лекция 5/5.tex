\documentclass[12pt]{article}
\usepackage[russian]{babel}
\usepackage[utf8]{inputenc}
\frenchspacing
\usepackage{amssymb, amsmath, amscd}
\usepackage[left=20mm, right=20mm, top=20mm, bottom=20mm]{geometry}
\usepackage{comment}
\usepackage{theorem}
\usepackage{indentfirst}
\renewcommand{\Im}{\operatorname{Im}}
\DeclareMathOperator{\Lin}{Lin}
\newcommand{\Al}{A_\lambda}
\newcommand{\Alo}{(\Al)^{-1}}
\newcommand{\Rez}{(I - \lambda A)^{-1}}
\begin{document}
\newtheorem{Theor}{Теорема}
\newtheorem{Utv}{Утверждение}
\newtheorem{Opr}{Определение}
\newtheorem{Prim}{Пример}
\newtheorem{Upr}{Упражнение}
\newtheorem{Nabl}{Наблюдение}
\newtheorem{Zam}{Замечание}
%=============================================================================!
\section*{Теория компактных операторов в гильбертовом пространстве.}
\textbf{Напоминание. }
{\sl
    Пусть  $H$~--- гильбертово пространство, $A : H \mapsto H$ линейный
    непрерывный оператор.
    $A$ называется компактным, если $\forall\{f_n\} \subset H$ такой, что
    $\|f_n\|\le R\;\forall n$ следует $\exists A f_{n_k}$~--- фундаментальная
    в $H$ подпоследовательность или, что равносильно $\forall \varepsilon > 0
    \;\exists A_\varepsilon : H \mapsto H$ линейный и непрерывный и $\dim \Im
    A_\varepsilon < + \infty$, тогда $\|A - A_\varepsilon\| \le \varepsilon\|$
    .
}

Для прикладных целей пусть $A$~--- компактный оператор и $\lambda \in
\mathbb C$, и $f \in H$.
$$
(I - \lambda A)u = f
$$
где $u \in H$ и нужно найти $u$.
\end{document}

















































