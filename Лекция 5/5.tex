\documentclass[14pt]{extarticle}
\usepackage{../preamble}
\begin{document}
%=============================================================================!
\section*{Теория компактных операторов в гильбертовом пространстве.}
\begin{MathCl}{Напоминание}
    Пусть  $H$~--- гильбертово пространство, $A : H \mapsto H$ линейный
    непрерывный оператор.
    $A$ называется компактным, если $\forall\{f_n\} \subset H$ такой, что
    $\|f_n\|\le R\;\forall n$ следует $\exists A f_{n_k}$~--- фундаментальная
    в $H$ подпоследовательность или, что равносильно $\forall \varepsilon > 0
    \;\exists A_\varepsilon : H \mapsto H$ линейный и непрерывный и $\dim \Im
    A_\varepsilon < + \infty$, тогда $\|A - A_\varepsilon\| \le \varepsilon$.
\end{MathCl}

Для прикладных целей пусть $A$~--- компактный оператор и $\lambda \in
\mathbb C$, и $f \in H$.
$$
(I - \lambda A)u = f
$$
где $u \in H$ и нужно найти $u$.

Из компактности $\forall \varepsilon > 0\colon \varepsilon|\lambda| < 1\;
\exists A_\varepsilon : H \mapsto H$~--- линейный и непрерывный оператор,
$\dim \Im A_\varepsilon < + \infty$ и $\|A_\varepsilon - A \| < \varepsilon$.
Тогда
$$
(I - \lambda A) = (I - \lambda A \pm \lambda A_\varepsilon) = (I - 
\lambda(A - A_\varepsilon) - \lambda A_\varepsilon)
$$
Обозначим $\lambda(A - A_\varepsilon)$ как $T_\varepsilon(\lambda)$.
$$
\|T_\varepsilon(\lambda)\| = \|\lambda\|\|A - A_\varepsilon\| \le |\lambda|
\varepsilon < 1
$$
А значит по теореме Неймана $\exists (I - T_\varepsilon(\lambda))^{-1} : H 
\mapsto H$ линейный и непрерывный и
$$
(I - T_\varepsilon(\lambda))^{-1} = \sum\limits_{n = 0}^\infty(T_\varepsilon
(\lambda))^n
$$
ряд сходящийся по операторной норме.
Уравнение
$$
(I - \lambda A)u = f
$$
называется уравнением Фредгольма 2-го рода.
Оно равносильно выражению
$$
(I - T_\varepsilon(\lambda) - \lambda A_\varepsilon)u = f
$$
Разобьём полученное выражение
$$
(I - T_\varepsilon(\lambda))(I - \lambda(I - T_\varepsilon(\lambda))^{-1}
A_\varepsilon))u = f
$$
Обозначим $(I - T_\varepsilon(\lambda))^{-1}$ как $L_\varepsilon(\lambda)$
Получаем
$$
(I - \lambda L_\varepsilon(\lambda) A_\varepsilon)u = f_\varepsilon(\lambda)
= L_\varepsilon(\lambda) f
$$
Обозначим $C_\varepsilon(\lambda) = L_\varepsilon(\lambda)A_\varepsilon$.
Этот оператор линейно непрерывный и его образ изоморфен образу $A_\varepsilon$
.
Отсюда $\dim C_\varepsilon = \dim A_\varepsilon$.
\begin{Utv}
    $A$~--- компактный оператор в $H$, тогда $A^*$ также компактный оператор в
    $H$.
\end{Utv}
\begin{Proof}
    Рассмотрим оператор такой, что
    $$
    \forall \varepsilon\; \exists A_\varepsilon\colon 
    A_\varepsilon f = \sum\limits_{k = 1}^N(f, h_k)g_k\quad h_k,g_k \in H
    $$
    Тогда $\|A - A_\varepsilon\| < \varepsilon$.
    $$
    \|A^* - A_\varepsilon^*\| = \|(A - A_\varepsilon)^*\| = 
    \|A - A_\varepsilon\| < \varepsilon
    $$
    Надо показать, что $A_\varepsilon^*$~--- компактный оператор.
    Покажем, что
    $$
    A_\varepsilon^* g = \sum\limits_{k = 1}^N(g, g_k)h_k
    $$
    так как
    \begin{multline*}
    (f, A_\varepsilon^* g) = (A_\varepsilon f, g) = \sum\limits_{k = 1}^{N}
    (f, h_k)(g_k, g) =\\= \sum\limits_{k = 1}^N (f, (g, g_k)h_k) = (f, \sum
    \limits_{k = 1}^N(g, g_k)h_k) = (f, A_\varepsilon^* g)
    \end{multline*}
    Получаем, что $\dim A_\varepsilon^* \le N \subset \Lin(h_1,\dots, h_N)$.
    Следовательно $A_\varepsilon^*$~--- компактный оператор.
\end{Proof}
\begin{Theor}[первая теорема Фредгольма]
    Пусть $A$~--- компактный оператор в $H$ и $\lambda \ne 0$, тогда
    $\dim \ker A_\lambda < +\infty$, где $A_\lambda = A - \lambda I)$
\end{Theor}
\begin{Proof}
    Заметим, что, во-первых, если $L \subset H$ подпространство и $\dim \ker L
    < +\infty$, то это равносильно тому, что для любой ограниченной
    последовательность из $L$ имеет фундаментальную подпоследовательность.
    В прямую сторону это следует из теоремы Больцано-Вейерштрасса.
    Покажем справедливость в обратную сторону.
    Если вдруг $\dim L = +\infty$, то $\exists\{f_n\}_{n = 1}^\infty$
    последовательность линейно независимых векторов.
    Подвергнем её процедуре ортогонализации Грама-Шмитда и получим $\{g_n\}
    _{n = 1}^\infty \subset L$, и $g_m \perp g_m\; n \ne m$, и $g_n = \Lin
    \{f_1,\dots,f_n\}$ так как
    \begin{gather*}
    0 \ne g_1 = f_1\\
    0 \ne g_2 = f_2 + \alpha g_1 \perp g_1\\
    \vdots\\
    0 \ne g_n = f_n + p_1 g_1 + \dots + p_{n - 1} g_{n - 1} \perp g_1, \dots,
    g_{n - 1}
    \end{gather*}
    Строим $h_n = \dfrac{g_n}{\|g_n\|}$.
    Тогда $\{h_n\}_{n = 1}^\infty$~--- ортонормированная последовательность в
    $L$, а значит не имеет фундаментальной подпоследовательности, так как
    $$
    \|h_n - h_m\|^2 = \|h_n\| + \|h_m\|^2 = \sqrt{2} \quad n \ne m
    $$
    получили противоречие с условием, что $\forall \{f_n\} \subset \ker A_
    \lambda$~--- ограниченная последовательность.
    $A f_{n_k}$~--- фундаментальная последовательность образов в силу
    компактности $A$.
    $$
    A_\lambda f_{n_k} = A f_{n_k} - \lambda f_{n_k} \equiv 0
    $$
    Следовательно $f_{n_k} = \dfrac{1}{\lambda}A f_{n_k}$~--- автоматически
    фундаментальная.
\end{Proof}
\newtheorem{Lemm}{Лемма}
\begin{Lemm}
    Пусть $A$~--- компактный оператор в $H$ и $\lambda \ne 0$.
    Тогда $\Im A_\lambda$ замкнуто в $H$.
\end{Lemm}
\begin{Proof}
    Имеем, что $\ker A_\lambda \oplus (\ker A_\lambda)^\perp = H$.
    Образ $\Im A_\lambda$ символически равен
    $$
    A_\lambda(H) = A_\lambda(\ker A_\lambda) + A_\lambda((\ker A_\lambda)^
    \perp) = A_\lambda((\ker A_\lambda)^\perp)
    $$
    Покажем, что $\exists k > 0\colon \forall f \in (\ker A_\lambda)^\perp$
    выполняется $\|A_\lambda\| \ge k \|f\|$.
    Отсюда вытекает, что $\exists \Al : \dfrac{1}{k}(\ker A_\lambda)^\perp 
    \mapsto \Im 
    \Al$ такой что
    $$
    \|f\| = \|\Alo \Al f\| \le \dfrac{1}{k}\|\Al f\|
    $$
    и следовательно
    $$
    \|\Alo\| \le \dfrac{1}{k}
    $$
    Если показать существование такого $k$, то получим, что $\forall g \in 
    \overline{\Im \Al} = \overline{\Al (\ker A_\lambda)^\perp}\; \exists f_n
    \in (\ker A_\lambda)^\perp\colon g = \lim\limits_{n \to \infty}\Al f_n$ в
    $H$.
    Тогда
    $$
    \|A_\lambda f_n - A_\lambda f_m\| \to 0 \quad n,m \to \infty
    $$
    С другой стороны
    $$
    \|\Al f_n - \Al f_m\| = \| \Al (f_n - f_m)\| \ge k \|f_n - f_m\|
    $$
    Поэтому
    $$
    \|f_n - f_m\| \to 0 \quad n,m \to \infty
    $$
    Следовательно, так как $H$~--- полное
    $$
    f_n \to h \in H
    $$
    Следовательно
    $$
    g = \lim_{n \to \infty}\Al f_n = \Al h \in \Im \Al
    $$
    
    Теперь увидим, что $k > 0$ действительно существует.
    Если вдруг такого $k > 0$ нет, то
    $\forall k > 0\; \exists f_k \in (\ker A_\lambda)^\perp\colon \|\Al f_k\| 
    < k \|f_k\|\quad f_k \ne 0$.
    Пусть $k = \dfrac{1}{n}, n \in \mathbb N$, тогда $f_k = f_{\dfrac{1}{n}}$ и
    $$
    \|\Al \dfrac{f_{\dfrac{1}{n}}}{\|f_{\dfrac{1}{n}}\|}\| \le \dfrac{1}{n}
    $$
    Введём $g_n = \dfrac{f_{\dfrac{1}{n}}}{\|f_{\dfrac{1}{n}}\|} \in H$ и $\|g_n
    \| = 1$.
    Тогда
    $$
    \left\{
        \begin{array}{l}
            \|\Al g_n\| < \dfrac{1}{n} \quad \forall n \in \mathbb N\\
            \|g_n\| = 1
        \end{array}
    \right.
    $$
    Следовательно $\Al g_n \to 0$ при $n \to \infty$.
    Так как $A$~--- компактный оператор, то $\exists A g_{n_m}$~--
    фундаментальная подпоследовательность в $H$, а значит сходится $Ag_{n_m}
    \to h \in H$.
    Тогда
    $$
    \left\{
        \begin{array}{l}
            \Al g_{n_m} \to 0\\
            A g_{n_m} \to h
        \end{array}
    \right.
    $$
    И тогда
    $$
    g_{n_m} = \dfrac{A g_{n_m} - \Al g_{n_m}}{\lambda} \to \dfrac{h}{\lambda}
    $$
    С другой стороны в $H$ $\Al g_{n_m} \to \Al \dfrac{h}{\lambda}$ 
    следовательно $\Al h = 0$ и $h$ лежит в ядре $\ker \Al$.
    Но $g_{n_m} \in (\ker A_\lambda)^\perp$~--- замкнутое подпространство в 
    $H$.
    И $g_{n_m} \to \dfrac{h}{\lambda} \Rightarrow \dfrac{h}{\lambda} \in 
    (\ker A_\lambda)^\perp \Rightarrow h \in (\ker A_\lambda)^\perp$.
    Таким образом $h = 0$, $\|g_{n_m}\| = 1$ и
    $$
    |\|\dfrac{h}{\lambda}\| - \|g_{n_m}\|| \le \|\dfrac{h}{\lambda} - g_{n_m}\|
    \to 0
    $$
    Значит $\dfrac{\|h\|}{|\lambda|} = 1 \Rightarrow  \|h\| = |\lambda| > 0$.
    Противоречие.
\end{Proof}
\begin{MathCl}{Вопрос}
    $A$~--- компактный оператор в $H$; $\dim H = \infty$, то $0 \in \sigma(A)
    $
\end{MathCl}

\noindent Если вдруг $0 \notin \sigma(A)$ то есть $0 \in \rho(A)$.
Это равносильно $\exists A^{-1} : H \mapsto H$.
Суперпозиция непрерывного и компактного оператора также компактный оператор.
$$
A^{-1} A = I : H \mapsto H
$$
Значит $I$ также компактный оператор, а значит $\forall f_n \in H$ 
ограниченной $I f_n = f_n$ существует $\exists I f_{n_k} = f_{n_k}$ 
фундаментальная подпоследовательность, значит $H$ конечномерный~--- 
противоречие с $\dim H = \infty$
\begin{Theor}[третья теорема Фредгольма]
    Пусть $A$ --- компактный оператор в $H$, $\lambda \ne 0$, тогда $\Im \Al =
    (\ker (\Al)^*)^\perp = (\ker A^*_{\overline{\lambda}})^\perp$,
    где $(\Al)^* = A^* - (\lambda I)^* = A^* - \overline{\lambda} I$.
    То есть равнение $\Al u = f$ разрешимо тогда и только тогда, когда
    $f \in (\ker(\Al)^*)^\perp$, иначе говоря $(f, v) = 0\; \forall v \in H
    \colon (\Al)^*v = 0 \Leftrightarrow A^*v - \overline{\lambda}v = 0$~--- 
    однородное союзное уравнение.
    Раскроем подробней.
    $\lambda \ne 0$ и $(I - \lambda A)u = f$
    $$
    \Im (I - \lambda A) = \Im (-\lambda)(A - \dfrac{I}{\lambda}) =
    \Im A_{\dfrac{1}{\lambda}} = (\ker(A_{\dfrac{1}{\lambda}})^*)^\perp = 
    (\ker(I - \overline{\lambda}A^*)^\perp)
    $$
    $f \in (I - \lambda A) \Leftrightarrow f \in (\ker(I - \overline{\lambda}
    A^*)^\perp)$, тогда $(f, v) = 0\; \forall f \in H$ и $v = 
    \overline{\lambda} A^*v$~--- однородное союзное уравнение.
\end{Theor}
\begin{Proof}
    Как было уже показано $\ker (\Al)^* = (\Im \Al)^\perp$, что равносильно
    $\overline{\Im \Al} = (\ker (\Al)^*)^\perp$.
    А по лемме 1 $\overline{\Im \Al} = \Im \Al$, что и требовалось доказать.
\end{Proof}
\begin{MathCl}[альтернатива Фредгольма]{Следствие}
    Пусть $A$~--- компактный оператор в $H$; $\lambda \ne 0$, то $\forall f 
    \in H\;\exists u \in H\colon \Al u = f$, либо $\exists v \in H\; v \ne 0
    \colon (\Al)^* v = 0$
\end{MathCl}
\begin{Proof}
    Либо $\ker(\Al)^* = 0$ и тогда по теореме Фредгольма $\Im \Al = H$, а по
    второй теореме Фредгольма тогда $\ker \Al = 0$ следовательно $\forall f 
    \in H\;\exists!u \in H \colon \Al u = f$
    
    Либо $\ker(\Al)^* \ne 0$ (и по второй теореме Фредгольма $\ker \Al \ne 0$
    и $\lambda \in \sigma_P(A)$), тогда $v \in \ker (\Al)^*\; v \ne 0$.
\end{Proof}
\begin{Theor}[вторая теорема Фредгольма]
    Пусть $A$~--- компактный оператор в $H$ и $\lambda \ne 0$, тогда
    $\dim \ker \Al = \dim \ker (\Al)^*$, где $\dim \ker \Al$ конечномерен по 
    1-ой теореме Фредгольма.
\end{Theor}
\begin{Lemm}
    $A$~--- компактный оператор в $H$ и $\lambda \ne 0 \in \sigma_P(A)$, 
    тогда $\Im \Al \ne H$.
\end{Lemm}
\begin{Prim}
    $A : H \mapsto H$~--- линейный непрерывный и не компактный оператор.
    $\lambda \in \sigma_P(A)$, $\lambda \ne 0$ и $\ker \Al \ne 0$, тогда может
    быть, что $\Im \Al = H$.
    
    Пусть $\{e_k\}_{ k = 1}^\infty$~--- ортонормированный базис в $H$ ($H =
    L_2[0,1]$ и $e_k(x) = \sqrt{2}\sin \pi k x,k \in \mathbb{N}$).
    Тогда 
    $$
    f = \sum\limits_{k = 1}^\infty \alpha_k(f) e_k,\quad \alpha_k(f) = (f,
    e_k)
    $$
    По равенству Парсеваля
    $$
    \|f\|^2 = \sum\limits_{k = 1}^\infty |\alpha_k(f)|^2
    $$
    Пусть оператор $T$ действует как
    $$
    Tf = \sum\limits_{k = 1}^\infty \alpha_{k + 1}(f)e_k
    $$
    Тогда
    $$
    \|T_f\| = \sqrt{\sum\limits_{k = 2}^\infty |\alpha_k(f)|^2} \le = \|f\|
    $$
    Норма $\|T\| \le 1$.
    $T e_2 = _1$, тогда
    $$
    \|T\| \ge \|T e_2\| = \|e_1\| = 1
    $$
    Отсюда $\|T\| = 1$.
    Пусть $f \in \ker T$, что равносильно $\alpha_k(f) = 0\;\forall \alpha \ge 
    2$, значит $f = \Lin\{e_1\}$.
    Тогда $\forall g \in H$
    $$
    g = \sum\limits_{k = 1}^\infty \beta_k(f) e_k
    $$
    и $\|g\|^2 = \sum\limits_{k = 1}^\infty |\beta_k(f)|^2 < \infty$.
    $$
    \exists f = \sum\limits_{k = 2}^\infty \beta_{k - 1}(f)e_k \in H
    $$
    что $Tf = g$, значит $\Im T = H$.
    Пусть $A = T + I$, $\lambda = 1$.
    Тогда $\Al = T + I - I = T$.
    Отсюда $\ker \Al = \ker T \ne 0$, но $\Im \Al = H = \Im T = H$.
\end{Prim}

Докажем теперь лемму.
\begin{Proof}
    Если вдруг $\Im \Al = H$, тогда можно показать последовательность
    замкнутых подпространств.
    $$
    0 \ne L_1 \varsubsetneq L_2 \varsubsetneq L_3 \varsubsetneq \dots 
    \varsubsetneq L_n \varsubsetneq \dots
    $$
    и $\Al L_n \subset L_{n - 1}\quad \forall n \ge 2$.
    Если указать, то тогда существует последовательность $\{f_n\}$:
    \begin{gather*}
        0 \ne f_1 \in L_1\\
        0 \ne f_2 \in L_2 \cap L_1^\perp\\
        0 \ne f_n \in L_n \cap L_{n - 1}^\perp\quad n \ge 2
    \end{gather*}
    Пусть теперь $g_1 = \dfrac{f_1}{\|f_1\|}$, $g_2 = \dfrac{f_2}{\|f_2\|}$, 
    $g_n = \dfrac{f_n}{\|f_n\|}$.
    $\{g_n\}$ также ограниченная последовательность в $H$.
    $\forall n, p \in \mathbb N$
    \begin{multline*}
        \|Ag_n - Ag_{n + p} \pm \lambda g_n \pm \lambda g_{n + p}\| =\\
        = \|\Al g_n - \Al g_{n + p} + \lambda g_n - \lambda g_{n + p}\| =\\
        = |\lambda|\|\dfrac{\Al g_n - \Al g_{n + p} + \lambda g_n}{\lambda} -
        g_{n + p}\| \ge\\\ge |\lambda| \|g_{n + p}\| = |\lambda|
    \end{multline*}
    Следовательно
    $$
    \|A g_n - A g_{n + p}\| > |\lambda| > 0
    $$
    Следовательно $A g_n$ не содержит фундаментальную подпоследовательность.
    
    Рассмотрим $L_n = \ker(\Al)^n$, тогда $L_1 = \ker \Al \ne 0$ по условию
    так как $\lambda \in \sigma_P(A)$.
    Верно ли, что $\forall n \in \mathbb{N}\; L_n \varsubsetneq L_{n + 1}$?
    Пусть $f \in L_n$, тогда $(\Al)^nf = 0$, значит $\Al (\Al)^n f = (\Al)^{n
    + 1}f = 0$, следовательно $f \in L_{n + 1}$.
    По условию $\exists f_1 = L_1 \setminus \{0\}$, тогда по предположению
    $\exists f_2 \in H$:
    $$
    \Al f_2 = f_1 \ne 0
    $$
    $\Rightarrow$ $f_2 \notin L_1$, но
    $$
    (\Al)^2 f_2 = \Al f_1 = 0
    $$
    отсюда $f_2 \in L_2 \setminus L_1$.
    $f_n \ne 0\; f_n \in L_n$, тогда $\exists f_{n + 1} \in H$:
    $$
    \Al f_{n + 1} = f_n \ne 0
    $$
    Следует ли $f_{n + 1} \notin L_n$?
    $$
    (\Al)^n \Al f_{n + 1} = (\Al)^{n + 1} f_{n + 1} = 0
    $$
    Тогда $f_{n + 1} \in L_{n + 1}\setminus L_n$.
    $f \in L_n \Leftrightarrow (\Al)^n f = 0$, $g = \Al f$, тогда
    $$
    (\Al)^{n - 1}g = (\Al)^n f = 0
    $$
    значит $g \in L_{n - 1}$.
    Таким образом получили противоречие с тем, что $A$ компактный оператор.
\end{Proof}

Перейдём теперь к доказательству второй теоремы Фредгольма.
\begin{Proof}
    По теореме Фредгольма для произвольного непрерывного оператора 
    \begin{gather*}
        \ker\Al = (\Im (\Al)^*)^\perp\\
        \ker (\Al)^* = (\Im \Al)^\perp
    \end{gather*}
    Если увидем, что $\dim \ker \Al \le \dim (\Im \Al)^\perp$, то это 
    аналогично для сопряжённого оператора $\dim \ker (\Al)^* \le \dim (\Im(\Al
    )^*)^\perp = \dim (\Im A^*_{\overline{\lambda}})^\perp$ и $
    \overline{\lambda} \ne 0$ и $A^*$~--- компактный оператор.
    Тогда $\dim \ker \Al \le \dim \ker (\Al)^*$ и $\dim \ker (\Al)^* \le \dim 
    \ker \Al$ и доказательство будет завершено.
    
    Если вдруг $\dim \ker \Al  \dim (\Im \Al)^\perp \ge 0$, тогда
    $\lambda \in \sigma_P(A) \setminus \{0\}$.
    Из первой теоремы Фредгольма $\ker \Al$ конечномерен и $(\Im \Al)^\perp$
    также конечномерен и меньшей размерности.
    $\exists \Phi : \ker \Al \mapsto (\Im \Al)^\perp$ линейный конечномерный
    $\ker \Phi \ne 0$.
    Оператор $\Phi$ будет компактный в конечномерном гильбертовом пространстве
    так как по теореме Больцано-Вейерштрасса все линейные операторы являются
    компактными.
    Рассмотрим $P : H \mapsto \ker \Al$, где $\ker \Al \ne 0$, ортопроектор.
    $\ker \Al \oplus (\ker \Al)^\perp = H$ и $\|P\| = 1$.
    Смотрим на
    $$
    T = A + \Phi P
    $$
    так как $A$ компактен, $\Phi$ также компактен и $P$ линеен, то $T$ 
    компактный оператор.
    Рассмотрим
    $$
    T_\lambda = \Al + \Phi P
    $$
    так как $\ker \Phi \ne 0$, то $\exists f_0 \ne 0 \in \ker \Al$ такой, что
    $\Phi f_0 = 0$ и $P f_0 = f_0$.
    Тогда
    $$
    T_\lambda f_0 = \Al f_0 + \Phi P f_0 = 0
    $$
    Таким образом $0 \ne f_0 \in \ker T_\lambda$, значит $0 \ne \lambda \in 
    \sigma_P(T)$.
    $$
    \Im T_\lambda = T_\lambda(H) = \Al(H) + \Phi P(\ker \Al)
    $$
    $\Al(H) = \Im \Al$, которое замкнуто по лемме 1, и
    $$
    \Phi P(\ker \Al) = \Phi \ker \Al = (\Im \Al)^\perp
    $$
    Также $H = \ker \Al + (\ker \Al)^\perp$ и по теореме Риса об ортогональном
    разложении
    $$
    \Im T_\lambda = \Im \Al + (\Im \Al)^\perp = H
    $$
    Следовательно $\Im T_\lambda = H$.
    Получили противоречие с леммой 2.
\end{Proof}
\begin{Theor}[4-ая теорема Фредгольма]
    Пусть $A$~--- компактный оператор в $H$.
    Тогда $\forall \varepsilon > 0$ множество
    $$
    \Lambda_\varepsilon = \{\lambda \in \mathbb{C}\mid
    \begin{array}{l}
        \lambda \in \sigma_P(A)\\
        |\lambda| > \varepsilon    
    \end{array}
    \}
    $$
    содержит конечное число элементов или пусто.
\end{Theor}
\begin{Proof}
    Если $\exists \varepsilon > 0\colon \Lambda_{\varepsilon_0}$ содержит 
    счётное и больше элементов, тогда $\exists \{\lambda_n\}_{n = 1}^\infty
    \subset \Lambda_{\varepsilon_0}$ и $\lambda_n \ne \lambda_m$ при
    $n \ne m$, $\lambda_n \in \sigma_P(A)$.
    Из курса линейной алгебры известно, что если $f_k \in \ker A_{\lambda_k},
    k=1\dots N$, то $\{f_1,\dots f_N\}$ линейно независимы.
    Рассмотрим $L_n = \Lin\{f_1 \dots f_N\}$.
    Оператор $A : L_n \mapsto L_n$.
    Рассмотри последовательность $L_1 \varsubsetneq L_2 \varsubsetneq L_3
    \varsubsetneq \dots$ и $L_N$~--- замкнутое конечномерное подпространство
    в $H$.
    Покажем, что $A_{\lambda_N}L_N \subset L_{N - 1}$ при $N \ge 2$
    $$
    A_{\lambda_N} (\sum\limits_{i = 1}^N \alpha_i f_i) = (A - \lambda_N I)
    \sum\limits_{i = 1}^N \alpha_i f_i = \sum\limits_{i = 1}^N
    \alpha_i \lambda_i f_i - \sum\limits_{i = 1}^N \alpha_i \lambda_N f_i =
    \sum\limits_{i = 1}^{N - 1} \alpha_i (\lambda_i - \lambda_N) f_i \in 
    L_{N - 1}
    $$
    Теперь как в доказательстве леммы 2 для подпространств $\{L_n\}_{N = 1}^
    \infty$ строим ортонормированную систему $g_N \in L_N \cap (L_{N - 1})^
    \perp$, где $g_1 = \dfrac{f_1}{\|f_1\|} \in L_1$ и $\|g_N\| = 1, N \ge 2$.
    $$
    \|A g_N - A g_{N + p} \pm \lambda_N g_N \pm \lambda_{N + p} g_{N + p}\| =
    |\lambda_{N + p}|\|\dfrac{A_{\lambda_N}g_N - A_{\lambda_{N + p}}g_{N + p} +
    \lambda_N g_N}{\lambda_{N +}} - g_{N + p}\| \ge |\lambda_{N + p}|\||
    \lambda_{N + p}|\| \ge \varepsilon_0
    $$
    Отсюда
    $$
    \|A g_N - A g_{N + p}\| \ge \varepsilon_0
    $$
    Следовательно $\{A g_N\}$ не содержит фундаментальную 
    подпоследовательность~--- противоречие с тем, что $A$~--- компактный
    оператор.
\end{Proof}
\begin{MathCl}[теорема о спектре компактного оператора]{Следствие}
    Пусть $A$~--- компактный оператор в $H$, то $\sigma(A)$~--- не более, 
    чем счётное множество.
    $\sigma(A) \setminus \{0\} \subset \sigma_P(A)$.
    $\sigma(A)$ имеет не более одной предельной точки и эта точка~--- ноль.
\end{MathCl}
\begin{Proof}
    $\lambda \in \sigma(A)\setminus\{0\}$ и если вдруг $\lambda \notin
    \sigma_P(A)$, то есть $\ker \Al = 0$, тогда по 2-ой теореме Фредгольма
    $\ker(\Al)^* = 0$, тогда по 3-ей теореме Фредгольма $\Im \Al = (\ker
    \Al^*)^\perp = H$, следовательно $\lambda \in \rho(A)$~--- противоречие.
    Следовательно $\lambda \in \sigma_P(A)$.
    По 4-ой теореме Фредгольма $\sigma_P(A)$ не более чем счётно,
    следовательно $\sigma(A)$ не более чем счётно если $\dim H = +\infty$,
    следовательно $0 \in \sigma(A)$.
    Если $\sigma_P(A)$ счётно, то $\exists \{\lambda\}_{n = 1}^\infty = 
    \sigma_P(A) \setminus\{0\}$, а по 4-ой теореме Фредгольма $\lambda_n \to 0
    $ при $n \to 0$, следовательно $\lambda = 0$~--- предельная точка спектра.
\end{Proof}
\end{document}