\documentclass[12pt]{article}
\usepackage[russian]{babel}
\usepackage[utf8]{inputenc}
\frenchspacing
\usepackage{amssymb, amsmath, amscd}
\usepackage[left=20mm, right=20mm, top=20mm, bottom=20mm]{geometry}
\usepackage{comment}
\usepackage{../theorem}
\usepackage{indentfirst}
\renewcommand{\Im}{\operatorname{Im}}
\DeclareMathOperator{\Lin}{Lin}
\newcommand{\Al}{A_\lambda}
\newcommand{\Alo}{(\Al)^{-1}}
\newcommand{\Rez}{(I - \lambda A)^{-1}}
\begin{document}
\newtheorem{Theor}{Теорема}
\newtheorem{Utv}{Утверждение}
\newtheorem{Opr}{Определение}
\newtheorem{Upr}{Упражнение}
\newtheorem{Nabl}{Наблюдение}
\newtheorem{Zam}{Замечание}
%=============================================================================!
\section*{Теория компактных операторов в гильбертовом пространстве.}
\textbf{Напоминание. }
{\sl
    Пусть  $H$~--- гильбертово пространство, $A : H \mapsto H$ линейный
    непрерывный оператор.
    $A$ называется компактным, если $\forall\{f_n\} \subset H$ такой, что
    $\|f_n\|\le R\;\forall n$ следует $\exists A f_{n_k}$~--- фундаментальная
    в $H$ подпоследовательность или, что равносильно $\forall \varepsilon > 0
    \;\exists A_\varepsilon : H \mapsto H$ линейный и непрерывный и $\dim \Im
    A_\varepsilon < + \infty$, тогда $\|A - A_\varepsilon\| \le \varepsilon$.
}

Для прикладных целей пусть $A$~--- компактный оператор и $\lambda \in
\mathbb C$, и $f \in H$.
$$
(I - \lambda A)u = f
$$
где $u \in H$ и нужно найти $u$.

Из компактности $\forall \varepsilon > 0\colon \varepsilon|\lambda| < 1\;
\exists A_\varepsilon : H \mapsto H$~--- линейный и непрерывный оператор,
$\dim \Im A_\varepsilon < + \infty$ и $\|A_\varepsilon - A \| < \varepsilon$.
Тогда
$$
(I - \lambda A) = (I - \lambda A \pm \lambda A_\varepsilon) = (I - 
\lambda(A - A_\varepsilon) - \lambda A_\varepsilon)
$$
Обозначим $\lambda(A - A_\varepsilon)$ как $T_\varepsilon(\lambda)$.
$$
\|T_\varepsilon(\lambda)\| = \|\lambda\|\|A - A_\varepsilon\| \le |\lambda|
\varepsilon < 1
$$
А значит по теореме Неймана $\exists (I - T_\varepsilon(\lambda))^{-1} : H 
\mapsto H$ линейный и непрерывный и
$$
(I - T_\varepsilon(\lambda))^{-1} = \sum\limits_{n = 0}^\infty(T_\varepsilon
(\lambda))^n
$$
ряд сходящийся по операторной норме.
Уравнение
$$
(I - \lambda A)u = f
$$
называется уравнением Фредгольма 2-го рода.
Оно равносильно выражению
$$
(I - T_\varepsilon(\lambda) - \lambda A_\varepsilon)u = f
$$
Разобьём полученное выражение
$$
(I - T_\varepsilon(\lambda))(I - \lambda(I - T_\varepsilon(\lambda))^{-1}
A_\varepsilon))u = f
$$
Обозначим $(I - T_\varepsilon(\lambda))^{-1}$ как $L_\varepsilon(\lambda)$
Получаем
$$
(I - \lambda L_\varepsilon(\lambda) A_\varepsilon)u = f_\varepsilon(\lambda)
= L_\varepsilon(\lambda) f
$$
Обозначим $C_\varepsilon(\lambda) = L_\varepsilon(\lambda)A_\varepsilon$.
Этот оператор линейно непрерывный и его образ изоморфен образу $A_\varepsilon$
.
Отсюда $\dim C_\varepsilon = \dim A_\varepsilon$.
\begin{Utv}
    $A$~--- компактный оператор в $H$, тогда $A^*$ также компактный оператор в
    $H$.
\end{Utv}
\begin{Proof}
    Рассмотрим оператор такой, что
    $$
    \forall \varepsilon\; \exists A_\varepsilon\colon 
    A_\varepsilon f = \sum\limits_{k = 1}^N(f, h_k)g_k\quad h_k,g_k \in H
    $$
    Тогда $\|A - A_\varepsilon\| < \varepsilon$.
    $$
    \|A^* - A_\varepsilon^*\| = \|(A - A_\varepsilon)^*\| = 
    \|A - A_\varepsilon\| < \varepsilon
    $$
    Надо показать, что $A_\varepsilon^*$~--- компактный оператор.
    Покажем, что
    $$
    A_\varepsilon^* g = \sum\limits_{k = 1}^N(g, g_k)h_k
    $$
    так как
    \begin{multline*}
    (f, A_\varepsilon^* g) = (A_\varepsilon f, g) = \sum\limits_{k = 1}^{N}
    (f, h_k)(g_k, g) =\\= \sum\limits_{k = 1}^N (f, (g, g_k)h_k) = (f, \sum
    \limits_{k = 1}^N(g, g_k)h_k) = (f, A_\varepsilon^* g)
    \end{multline*}
    Получаем, что $\dim A_\varepsilon^* \le N \subset \Lin(h_1,\dots, h_N)$.
    Следовательно $A_\varepsilon^*$~--- компактный оператор.
\end{Proof}
\begin{Theor}[первая теорема Фредгольма]
    Пусть $A$~--- компактный оператор в $H$ и $\lambda \ne 0$, тогда
    $\dim \ker A_\lambda < +\infty$, где $A_\lambda = A - \lambda I)$
\end{Theor}
\begin{Proof}
    Заметим, что, во-первых, если $L \subset H$ подпространство и $\dim \ker L
    < +\infty$, то это равносильно тому, что для любой ограниченной
    последовательность из $L$ имеет фундаментальную подпоследовательность.
    В прямую сторону это следует из теоремы Больцано-Вейерштрасса.
    Покажем справедливость в обратную сторону.
    Если вдруг $\dim L = +\infty$, то $\exists\{f_n\}_{n = 1}^\infty$
    последовательность линейно независимых векторов.
    Подвергнем её процедуре ортогонализации Грама-Шмитда и получим $\{g_n\}
    _{n = 1}^\infty \ subset L$, и $g_m \perp g_m\; n \ne m$, и $g_n = \Lin
    \{f_1,\dots,f_n\}$ так как
    \begin{gather*}
    0 \ne g_1 = f_1\\
    0 \ne g_2 = f_2 + \alpha g_1 \perp g_1\\
    \vdots\\
    0 \ne g_n = f_n + p_1 g_1 + \dots + p_{n - 1} g_{n - 1} \perp g_1, \dots,
    g_{n - 1}
    \end{gather*}
    Строим $h_n = \frac{g_n}{\|g_n\|}$.
    Тогда $\{h_n\}_{n = 1}^\infty$~--- ортонормированная последовательность в
    $L$, а значит не имеет фундаментальной подпоследовательности, так как
    $$
    \|h_n - h_m\|^2 = \|h_n\| + \|h_m\|^2 = \sqrt{2} \quad n \ne m
    $$
    получили противоречие с условием, что $\forall \{f_n\} \subset \ker A_
    \lambda$~--- ограниченная последовательность.
    $A f_{n_k}$~--- фундаментальная последовательность образов в силу
    компактности $A$.
    $$
    A_\lambda f_{n_k} = A f_{n_k} - \lambda f_{n_k} \equiv 0
    $$
    Следовательно $f_{n_k} = \frac{1}{\lambda}A f_{n_k}$~--- автоматически
    фундаментальная
\end{Proof}
\end{document}

















































