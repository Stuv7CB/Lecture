\documentclass[14pt]{extarticle}
\usepackage{../preamble}
\DeclareMathOperator{\rh}{\rho}
\begin{document}
\section*{Элементы теории линейных операторов в гильбертовых пространствах и её приложение к линейным операторным уравнениям в гильбертовом пространстве.}
Пусть $H$~--- гильбертово пространство, то есть полное евклидово пространство со скалярным произведением.
Скалярное произведение определим в комплексном пространстве: $(f,g) \in \mathbb C$.
Тогда $\|f\| = \sqrt{(f,g)}$~--- евклидова норма.
В этом пространстве выполняется неравенство треугольника:
$$
\|f + g\| \le \|f\| + \|g\|\text{,}
$$
которое следует из неравенства Коши~--- Буняковского $|(f,g)| \le \|f\|\|g\|$.

Полнота по определению: из
$$
\forall \{f_n\} \subset H\; \forall n,m \in \mathbb N\; \|f_n - f_m\| \to 0 \quad \text{при} \quad n,m \to \infty
$$
следует, что 
$$
\exists h \in H\; \|f_n - h\| \to 0\quad \text{при} \quad n \to \infty
$$

$G \subset \mathbb R^m$~--- открытое множество.
$CL_2(G) = \{f\colon G \to \mathbb C \mid f\text{ непрерывная}, \int\limits_G |f|^2 dx < +\infty\}$.
Введём скалярное произведение как
$$
(f, g) = \int\limits_G f(x) \overline{g(x)} dx
$$
Тогда такое множество неполное.
$L_2(G)$~--- пополнение $CL_2(G)$, или по-другому $f:G \mapsto \mathbb C$ измерима по Лебегу и $\int\limits_G |f|^2 dx < +\infty$.
\subsection*{Геометрия гильбертого пространства}
\begin{Theor}[Рисса о расстоянии]
    Пусть $L \subset H$, где $L$~--- замкнутое подпространство.
    Следовательно
    $$
    \forall f \in H\; \exists! g \in L\colon \|g - f\| = \rh(f, L)
    $$ 
    или по-другому 
    $$
    \inf\|f - h\|, \;h\in L
    $$
    Обозначим $g_f$~--- проекция $f$.
    Тогда отображение $P:H \mapsto L$ такое, что $Pf = g_f$, называется ортопроектором из $L$ на $H$.
\end{Theor}
\begin{Proof}
    Покажем существование.
    $$
    \forall f \in H\;\exists \{g_n\} \subset L\colon \rh(f, L) \le \|f-g_n\| \le \rh(f,L) + \dfrac{1}{n}
    $$
    Тогда
    $$
    \|f - g\| \to \rh(f,L)\quad \text{при} \quad n \to \infty
    $$
    Установим факт фундаментальности.
    Наблюдение: из равенства параллелограмма следует:
    $$
    f, g\in H\;\|f + g\|^2 + \|f - g\|^2 = 2\|f\|^2 + 2\|g\|^2
    $$
    Тогда
    $$
    \|g_n - g_m\|^2 = \|(g_n - f) - (g_m - f)\|^2 = 2\|g_n - f\|^2 + 2\|g_m - f\|^2 - \|g_n + g_m - 2f\|^2
    $$
    Теперь учтём, что
    $$
    \|g_m+g_m-2f\|^2 = 4\|\dfrac{g_n + g_m}{2} - f\|^2 \ge 4\rh^2(f, L)
    $$
    где $\dfrac{g_n + g_m}{2} \in L$, $\|g_{n,m} - f\|^2 = \rh^2(f, L)$
    Тогда получаем
    $$
    \|g_n - g_m\|^2 \le 2\|g_n - f\|^2 + 2\|g_m - f\|^2 - 4\rh^2(f,L) \to 0 \quad n,m \to 0
    $$
    $\exists g \in H\colon g_n \to g$, по этому $\|f - g_n\| \to \rh$, а так как $L$~--- замкнутое, то $g \in L$.
    Получаем 
    $$
    |\|f - g\| - \|f - g_n\|| \le \|g - g_n\| \to 0 \Rightarrow \|f-g\| = \rh(f,L)
    $$
    Отсюда следует существование.
    
    Покажем единственность.
    Пусть
    $$
    \|f - g_1\| = \|f - g_2\| = \rh(f, L)\quad g_1, g_2 \in L
    $$
    Рассмотрим $\|g_1 - g_2\|$:
    $$
    0 \le \|g_1 - g_2\|^2 = \|(g_1 - f) - (g_2 - f)\|^2 = 2\|g_1 - f\|^2 + 2\|g_2 - f\|^2 - \|\dfrac{g_1 + g_2}{2} - f\|^2 \le 4\rho^2 - 4\rho^2 = 0
    $$
    Следовательно $g_1 = g_2$.
    Доказательство закончено
\end{Proof}

Таким образом мы доказали теорему Рисса о расстоянии.
Пусть $P_L : H \mapsto L$, $P_L f = g_f$.
Тогда $\|P_Lf - f\| = \rh(f, L)$. $P_L$~--- линейный оператор.
\begin{MathCl}{Наблюдение}
	$\|g - f\| = \rh(f, L)\quad g \in L$ равносильно тому, что 
	$$
	\left\{
		\begin{aligned}
			&f - g \in L^\perp\\
        	&g \in L\\
    	\end{aligned}
	\right.
	$$
	Где $L^\perp = \{h \in H \mid (h, f) = 0 \quad \forall f \in L\}$~--- замкнутое подпространство.
	Это следует из неравенства Коши~--- Буняковского: пусть $h_n \to h$ и $h_n \in L^\perp$.
	Тогда
	$$
	|(h, f) - (h_n, f)| = |(h - h_n, f)| \le \|h - h_n\| \|f\|\to 0
	$$
	Поэтому $(h, f)=0$, а значит $h \in L$.
\end{MathCl}
Докажем теперь наше наблюдение
\begin{Proof}
    В прямую сторону.
    $$
    \|g - f\| = \rh(f, L)
    $$
    Прибавим к $g$ $th$ так, чтобы $g + th \in L$.
    $$
    \|g - f\| \le \|(g + t h) - f\| \quad \forall h \in L
    $$
    Возведём в квадрат и разложим выражение на компоненты
    \begin{gather*}
        \|g - f\|^2 \le \|(g - f) + t h\|^2\\
        \|g - f\|^2 \le ((g - f) + t h, (g - f) + t h)^2\\
        \|g - f\|^2 \le \|g - f\|^2 + (g - f, t h) + (t h, g - f) + |t|^2 \|h\|^2\\
	\end{gather*}
	Перенесём левую часть вправо и учтём, что по определению скалярного произведения $(a, b) = \overline{(b, a)}$
	\begin{gather*}
        0 \le (g - f, t h) + \overline{(g - f, t h)} + |t|^2 \|h\|^2\\ 
        0 \le 2\Re(g - f,t h) + |t|^2 \|h\|^2\\
        0 \le 2\Re(g - f, \dfrac{t}{|t|} h) + O(|t|)
	\end{gather*}
    Пусть $t \in \mathbb R$.
    $$
    0 \le 2\Re(g - f, h)
    $$
    подставляя $\pm h$ имеем
    $$
    \Re(g - f, h) = 0
    $$
    Пусть теперь $t = \imath \tau\; \tau \in \mathbb R$.
    $$
    2\Re(g - f, \imath h) = 2\Im(g - f, h) \ge 0\quad\forall h \in L
    $$
    Возмём $\pm h$ и получаем
    $$
    (g - f, h) = 0
    $$
    Таким образом для любого $h$ мы доказали в прямую сторону.
    
    В обратную сторону.
    \begin{multline*}
    \forall h \in L\;\|f - (g + h)\|^2 = \|(f - g) + h\|^2 =\\
     = (f - g, f - g) + (h, h) + (f - g, h) + (h, f - g) =\\
    = \|f - g\|^2+\|h\|^2 \ge \|f - g\|^2 \Rightarrow \\
    \Rightarrow \forall h\;\|f - (g + h)\|^2 \ge \|f - g\|^2 \Rightarrow \|f - g\| = \rh(f, L)
    \end{multline*}
    Что и требовалось доказать.
    Доказательство закончено.
\end{Proof}

Покажем линейность оператора $P_L$.
\begin{gather*}
    P_L(f_1 + f_2) = g\\
    P_Lf_1 = g_1 \Leftrightarrow (f_1 - g_1, h) = 0\\
    P_Lf_2 = g_2 \Leftrightarrow (f_2 - g_2, h) = 0
\end{gather*}
Сложим правые выражения и получим
$$
(f_1 + f_2 - (g_1 + g_2), h) = 0 \Rightarrow g_1 + g_2 = P_L(f_1 + f_2)
$$
Аналогично доказывается однородность $P_L(\alpha f) = \alpha P_L(f)$: 
$$
P_L(f) = g \Leftrightarrow (f - g, h) = 0\Rightarrow (\alpha f - \alpha g, h) = 0\Rightarrow P_L(\alpha f) = \alpha g
$$

\textbf{Проблема выпуклости в $H$}

Пусть $A \subset H$~--- назовём выпуклым, если $\forall f,g \in A\;\forall t \in [0;1]\;t f + (1 - t) g \in A$.
Мы уже доказали, что для любого выпуклого и замкнутого $A \subset H\;\forall f \in H\;\exists! g \in A\colon \|f - g\| = \rh(f, A)$.
Такое множество ещё называют чебышевским.
Верно ли обратное? Пусть $A \subset H$~--- чебышевское, то есть $\forall f \in H\;\exists! g \in A\colon \|f - g\| = \rh(f,A)$. 
Следует ли отсюда, что $A$~--- выпуклое и замкнутое?
Замкнутость была доказана в 1936 году для конечномерных $H$.
Интересные подвижки были получены на мехмате Бородиным Петром Анатольевичем.
Он ввёл 2-чебышевские множества.
$\forall f,h \in H$ пусть $\rho_2(f, h, A) = \inf\{\|f - g\| + \|h - g\|):g \in A\}$, $2\rh(f, A) = \rho_2(f, A)$.
Из 2-чебышевости следует выпуклость и замкнутость.
Но получаем вопрос: верно ли что из чебышевости следует 2-чебышевость?

\begin{Theor}[Рисса об ортогональном дополнении]
    $L \subset H$~--- замкнутое подпространство $\Rightarrow$ $L \oplus L^\perp = H$ и $L \cap L^\perp = \varnothing$.
    Последнее очевидно. 
    $\forall f \in H\;\exists! g \in L$ и $h \in L^\perp\colon f = g + h$.
    $$
    \left\{
        \begin{aligned}
            &f - g \in L^\perp\\
            &g \in L\\
        \end{aligned}
    \right.
    $$
    Это равносильно $P_Lf = g$.
\end{Theor}
\begin{Proof}
$f \in H$, смотрим на $g = P_Lf \Leftrightarrow h = f - g \in L^\perp\Rightarrow f = g + h$.
Доказательство закончено.
\end{Proof}

Пусть $f = g_1 + h_1$ и $f = g_2 + h_2$, и $g_{1,2} \in L$.
Тогда $g_1 - g_2 = h_2 - h_1 \in L^\perp$, но $L \cap L^\perp\Rightarrow g_1 = g_2$ и $h_1 = h_2$.

\begin{Theor}[Рисс, Фреше]
    $\Phi:H \mapsto \mathbb C$~--- линейный и непрерывный функционал ($f_n \to f$ в $H$ по норме, $\Phi(f_n) \to \Phi(f)$ в $\mathbb C$).
    Тогда $\exists! h \in H\colon \Phi(f) = (f,h)$.
\end{Theor}
\begin{Proof}
    $L = \ker \Phi$~--- замкнутое подпространство в $H$.
    Первое в связи непрерывности, второе в силу линейности.
    $L \oplus L^\perp = H$.
    Случай $\Phi = 0$ очевиден: $h = 0$.
    Пусть теперь $\Phi \ne 0\Rightarrow (\ker\Phi)^\perp \ne \{0\}$.
    Пусть $h_0 \in (\ker\Phi)^\perp\backslash\{0\}$, тогда отсюда следует $f \in H\;f = g +\alpha h_0$,
    где $g \in \ker \Phi$ и $\alpha = \dfrac{\Phi(f)}{\Phi(h_0)}$.
    Следовательно
    $$
    g = f - \dfrac{\Phi(f)}{\Phi(h_0)}h_0\Rightarrow (f,h_0) = (g, h_0) + \dfrac{\Phi(f)}{\Phi(h_0)}\|h_0\|^2
    $$
    Тогда
    $$
    \Phi(f) = (f, \dfrac{\overline{\Phi(h_0)}}{\|h_0\|}h_0)
    $$
    где $\dfrac{\overline{\Phi(h_0)}}{\|h_0\|}h_0 = h$.
    Пусть
    $$
    \Phi(f) = (f, h_1) = (f, h_2)\quad\forall f\in H
    $$
    Тогда
    $$
    f = h_1 - h_2\Rightarrow \|h_1 - h_2\|^2 = 0\Rightarrow h_1 = h_2
    $$
\end{Proof}
\end{document}
