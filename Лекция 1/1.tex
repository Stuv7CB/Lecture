\documentclass[12pt]{article}
\usepackage[russian]{babel}
\usepackage[utf8]{inputenc}
\frenchspacing
\usepackage{amssymb, amsmath, amscd}
\usepackage[left=20mm, right=20mm, top=20mm, bottom=20mm]{geometry}
\usepackage{comment}
\DeclareMathOperator{\rh}{\rho}
\DeclareMathOperator{\Ree}{Re}
\DeclareMathOperator{\Imm}{Im}
\begin{document}
\begin{center}
Элементы теории линейных операторов в гильбертовых пространствах и её приложение к линейным операторным уравнениям в гильбертовом пространстве.
\end{center}

Пусть $H$ --- гильбертово пространство, то есть полное евклидово пространство со скалярным произведением.
Скалярное произведение определим в комплексном пространстве, то есть $(f,g)\in \mathbb C$.
Тогда $\|f\|=\sqrt{(f,g)}$ --- евклидова норма.
Также в этом пространстве выполняется неравенство треугольника $\|f+g\|\le\|f\|+\|g\|$, которое следует из неравенства Коши-Буняковского $|(f,g)|\le\|f\|\|g\|$.

Полнота по определению: $\forall \{f_n\}\subset H$: $\forall n,m\in\mathbb N$ $\|f_n-f_m\|\to0$ при $n,m\to\infty$, следует, что $\exists h \in H$: $\|f_n-h\|\to0$ при $n\to\infty$.

$G\subset\mathbb R^m$ --- открытое множество.
$CL_2(G)=\{f:G\to\mathbb C|f\text{ непрерывная}, \int\limits_G|f|^2 dG<+\infty\}$.
Введём скалярное произведение, как $(f,g)=\int\limits_G f(x)\overline{g(x)} dx$. Тогда такое множество неполное.
$L_2(G)$ --- пополнение $CL_2(G)$, или по-другому $f:G\to\mathbb C$ измерима по Лебегу и $\int\limits_G|f|^2 dG<+\infty$.
\begin{center}
Геометрия гильбертого пространства
\end{center}

1) Пусть $L\subset H$: $L$ --- замкнутое подространство.
Следовательно $\forall f\in H$ $\exists! g\in L$: $\|g-f\|=\rh(f,L)$ или по-другому $\inf\|f-h\|$, $h\in L$.
Обозначим $g_f$ --- проекция $f$.
Тогда отображение $P:H\to L$ такое, что $Pf=g_f$, называется ортопроектором из $L$ на $H$.

\textbf{Доказательство}

Покажем существование.
$\forall f\in H$ $\exists \{g_n\}\subset L$: $\rh(f,L)\le \|f-g_n\|\le\rh(f,L)+\frac{1}{n}$.
Тогда $\|f-g\|\to\rh(f,L)$ $n\to\infty$.
Установим факт фундаментальности.
Наблюдение: из равенства параллелограмма следует: $f, g\in H$ $\|f+g\|^2+\|f-g\|^2=2\|f\|^2+2\|g\|^2$.
Тогда
\begin{gather*}
\|g_n-g_m\|^2=\|(g_n-f)-(g_m-f)\|^2=2\|g_n-f\|^2+2\|g_m-f\|^2-\|g_n+g_m-2f\|^2\boxed{\le}\\
\|g_m+g_m-2f\|^2=4\|\underbrace{\frac{g_n+g_m}2}_{\in L}-f\|^2\ge4\rh^2(f,L)\\
\boxed{\le}2\underbrace{\|g_n-f\|^2}_{\rh^2(f,L)}+2\underbrace{\|g_m-f\|^2}_{\rh^2(f,L)}-4\rh^2(f,L)\to0\text{ при }n,m\to0
\end{gather*}
$\exists g\in H$: $g_n\to  g$ по норме, а так как $L$ --- замкнутое, то $g\in L$.
Получаем 
$$
|\|f-g\|-\underbrace{\|f-g_n\|}_{\text{стремится к }\rho}|\le\|g-g_n\|\to0\Rightarrow\|f-g\|=\rh(f,L)
$$
Отсюда следует существование.

Покажем единственность. Пусть $\|f-g_1\|=\|f-g_2\|=\rh(f,L)$, $g_1$ и $g_2\in L$.
$$
0\le\|g_1-g_2\|^2=\|(g_1-f)-(g_2-f)\|^2=2\|g_1-f\|^2+2\|g_2-f\|^2-\|\frac{g_1+g_2}2-f\|^2\le4\rho^2-4\rho^2=0
$$
Следовательно $g_1=g_2$. Доказательство закончено

Таким образом мы доказали теорему Рисса о расстоянии.
Пусть $P_L:H\to L$, $P_L f=g_f$.
Тогда $\|P_Lf-f\|=\rh(f,L)$. $P_L$ --- линейный оператор.
Наблюдение: $\|g-f\|=\rh(f,L)$; $g\in L$ равносильно тому, что 
$$
\left\{
\begin{aligned}
&f-g\in L^{+}\\
&g\in L\\
\end{aligned}
\right.
$$
Где $L^+=\{h\in H|(h,f)=0\text{ }\forall f\in L\}$ --- замкнутое подпространство.
Замкнутое в силу неравенства Коши-Буняковского: пусть $h_n\to h$ и $h_n\in L^+$. Тогда
$$
|(h,f)-(h_n,f)|=|(h-h_n,f)|\le\|h-h_n\|\|f\|\to 0
$$
Поэтому $(h,f)=0$, а значит $h\in L$.
Докажем теперь наше наблюдение

\textbf{Доказательство}

В прямую сторону.
\begin{gather*}
\|g-f\|=\rh(f,L)\\
\|g-f\|\le\|\underbrace{(g+h^t)}_{\in L}-f\|\text{ }\forall h\in L\\
\|g-f\|^2\le\|(g-f)+h^t\|^2\\
0\le2\Ree(g-f,th)+|t|^2\|h\|^2\\
0\le2\Ree(g-f,\frac{t}{|t|}h)+\underbrace{O(|t|)}_{\text{стремиться к 0}}
\end{gather*}
Пусть $t\in\mathbb R$.
$0\le2\Ree(g-f,h)$, подствляя $\pm h$ имеем $\Ree(g-f, h)=0$.
Пусть теперь $t=\imath \tau$, $\tau \in \mathbb R$.
$2\Ree(g-f,\imath h)=2\Imm(g-f,h)\ge0$ $\forall h\in L$.
Возмём $\pm h$ и получаем $(g-f,h)=0$. Таким образом для люого $h$ мы доказали в прямую сторону.

В обратную сторону.
$\forall h\in L$ $\|f-(g+h)\|^2=\|f-g\|^2+\|h\|^2\ge\|f-g\|^2$ $\Rightarrow$ при $h=0$ $\|f-g\|^2=0$.
Доказательство закончено.

Докажем линейность оператора $P_L$.
\begin{gather*}
P_L(f_1+f_2)=g\\
P_Lf_1=g_1\Leftrightarrow(f_1-g_1,h)=0\\
P_Lf_2=g_2\Leftrightarrow(f_2-g_2,h)=0
\end{gather*}
Сложим правые выражения и получим $(f_1+f_2-(g_1+g_2),h)=0$ $\Rightarrow$ $g_1+g_2=P_L(f_1+f_2)$.
Аналогично доказывается однородность $P_L(\alpha f)=\alpha P_L(f)$: $P_L(f)=g\Leftrightarrow(f-g,h)=0$ $\Rightarrow$ $(\alpha f- \alpha g,h)=0$ $\Rightarrow$ $P_L(\alpha f)=\alpha g$.

\textbf{Проблема выпуклости в $H$}

Пусть $A\subset H$ --- назовём выпуклым, если $\forall f,g\in A$ $\forall t\in[0;1]$ $tf+(1-t)g\in A$.
Мы уже доказали, что для любого выпуклого и замкнутого $A\subset H$ $\forall f\in H$ $\exists!g\in A$: $\|f-g\|=\rh(f,A)$.
Такое множество ещё называют чебышевским.
А верно ли обратное? А именно, пусть $A\subset H$ --- чебышевское, то есть $\forall f\in H$ $\exists!g\in A$: $\|f-g\|=\rh(f,A)$. 
Следует ли отсюда, что $A$ --- выпуклое и замкнутое?
Замкнутость была доказана в 1936 году для конечномерных $H$.
Интересные подвижки были получены на мехмате Бородиным Петром Анатольевичем.
Он ввёл 2-чебышевские множества.
$\forall f,h\in H$ пусть $\rho_2(f,h,A)=\inf\{\|f-g\|+\|h-g\|):g\in A\}$, $2\rh(f,A)=\rho_2(f,A)$.
Из 2-чебышевости следует выпуклость и замкнутость.
Но получаем вопрос: верно ли что из чебышевости следует 2-чебышевость?

2) \textbf{Теорема} (Рисса об ортогональном дополнении)
$L\subset H$ --- замкнутое подпространство $\Rightarrow$ $L\oplus L^+=H$ и $L\cap L^+=\varnothing$.
Последнее очевидно. 
$\forall f\in H$ $\exists! g\in L$ и $h\in L^+$: $f=g+h$.
$$
\left\{
\begin{aligned}
&f-g\in L^{+}\\
&g\in L\\
\end{aligned}
\right.
$$
Это равносильно $P_Lf=g$.

\textbf{Доказательство}

$f\in H$, смотрим на $g=P_Lf\Leftrightarrow h=f-g\in L^+\Rightarrow f=g+h$ Доказательство закончено.

Пусть $g_1,g_2\in L$, $h_1, h_2\in L^+$,  $f=g_1+h_1=g_2+h_2$.
Тогда $\underbrace{g_1-g_2}_{\in L}=h_2-h_1\in L^+$, но $L\cap L^+$ $\Rightarrow$ $g_1=g_2$ и $h_1=h_2$.

\textbf{Теорема} (Рисс, Фреше)

$\Phi:H\to\mathbb C$ --- линейный и непрерывный функционал ($f_n\to f$ в $H$ по норме, $\Phi(f_n)\to\Phi(f)$ в $\mathbb C$).
Тогда $\exists!h\in H$: $\Phi(f)=(f,h)$.

\textbf{Доказательство}

$L=\ker \Phi$ --- замкнутое подпространство в $H$.
Первое в связи непрерывности, второе в силу линейности.
$L\oplus L^+$.
Случай $\Phi=0$ очевиден: $h=0$.
Пусть теперь $\Phi\ne0\Rightarrow(\ker\Phi)^+\ne\{0\}$
Пусть $h_0\in(\ker\Phi)^+\backslash\{0\}$, тогда отсюда следует $f\in H$ $f=\underbrace{g}_{\in \ker \Phi}+\alpha h_0$, где $\alpha=\dfrac{\Phi(f)}{\Phi(h_0)}$.
Следовательно $g=f-\dfrac{\Phi(f)}{\Phi(h_0)}h_0$ $\Rightarrow$ $(f,h_0)=\underbrace{(g,h_0)}_{0}+\dfrac{\Phi(f)}{\Phi(h_0)}\|h_0\|^2$.
Тогда $\Phi(f)=(f, \underbrace{\dfrac{\overline{\Phi(h_0)}}{\|h_0\|}h_0}_{\text{равно }h})$.
Пусть $\Phi(f)=(f,h_1)=(f,h_2)$ $\forall f\in H$.
$f=h_1-h_2$ $\Rightarrow$ $\|h_1-h_2\|^2=0$ $\Rightarrow$ $h_1=h_2$.
\end{document}