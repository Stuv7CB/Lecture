\documentclass[12pt]{article}
\usepackage[russian]{babel}
\usepackage[utf8]{inputenc}
\frenchspacing
\usepackage{amssymb, amsmath, amscd}
\usepackage[left=20mm, right=20mm, top=20mm, bottom=20mm]{geometry}
\usepackage{comment}
\usepackage{../theorem}
\renewcommand{\Im}{\operatorname{Im}}
\DeclareMathOperator{\Lin}{Lin}
\begin{document}
\section*{Линейные непрерывные операторы в евклидовом пространстве.}
\newtheorem{Theor}{Теорема}
\newtheorem{Opr}{Определение}
\newtheorem{Upr}{Упражнение}
\newtheorem{Nabl}{Наблюдение}
\newtheorem{Zam}{Замечание}

$\varepsilon_1, \varepsilon_2$ --- два евклидовых пространства.
$A : \varepsilon_1 \to \varepsilon_2$ --- линейный оператор.
Иначе говоря $\forall \alpha_{1, 2}\; \forall f_{1, 2} \in \varepsilon_1\; A(\alpha_1 f_1 + \alpha_2 f_2) = \alpha_1 A(f_1) + \alpha A(f_2)$.

\begin{Opr}
    $A$ непрерывна в $f_0 \in \varepsilon_1\; \Leftrightarrow\; \forall \varepsilon > 0\; \exists \delta_0(\varepsilon)$
    $\forall f \in \varepsilon_1\colon \|f - f_0\| \le \delta_0(\varepsilon) \Rightarrow  \|Af-Af_0\| \le \varepsilon$.
\end{Opr}

Из непрерывность в $f_0$ следует непрерывность оператора $A\; \forall g \in \varepsilon_1$.
Так как $\forall f \in \varepsilon_1\; \|f - g\| \le \delta_0(\varepsilon)$, то 
$$
\|Af - Ag\| = \|A(f - g) + A(f_0) - A(f_0)\| = \|A(f_0 + (f - g)
- A(f_0)\| \le \delta_0(\varepsilon)
$$
$\Rightarrow$ непрерывна в $g$.
В частности при $g = 0$
$$
\|Af\| \le \varepsilon\quad \forall \|f\| \le \delta_0(\varepsilon)
$$
Поэтому $\delta_0$ --- универсальное число.

Пусть $\varepsilon_1 = 1$, $\delta_0(1)$.
Тогда $\forall f \ne 0$:
$$
\left \| \frac{f}{\|f\|}\delta_0(1)\right \| = \delta_0(1)
$$
Подставим это выражение под знак оператора.
$$
\|A(\frac{f}{\|f\|}\delta_0(1))\| \le 1\; \Leftrightarrow\; \frac{\delta_0(1)}{\|f\|}\|Af\| \le 1
$$
$\Rightarrow$ оцениваем норму образа через норму
прообраза:
$$
\forall f \in \varepsilon_1\; \|A(f)\| \le \frac{\|f\|}{\delta_0(1)} \Rightarrow  \forall f,g \in \varepsilon_1 \|A(f-g)\| \le \frac{1}{\delta_0(1)}\|f - g\|
$$
Это липшецевость оператора $A$ на $\varepsilon_1$ с $L=\frac{1}{\delta_0(1)}$.
Рассмотрим наименьшую константу Липшеца и назовём её нормой.


\begin{Opr}
    $A : \varepsilon_1 \to \varepsilon_2$, такой что $A \ne 0$ --- линейный и непрерывный оператор, то
    $$
    \|A\| = \inf\{L > 0 \mid \|Af\| \le L\|f\| \quad \forall f \in \varepsilon_1\}
    $$
    Очевидно, что это так же равно
    $$
    \sup\limits_{f \in \varepsilon_1, f \ne 0} \frac {\|Af\|}{\|f\|}=L_0,\quad L_0 \le L
    $$
\end{Opr}


\begin{Prim}[Линейный разрывный оператор]
	$\varepsilon_1 = \{f \in C^1[0,1]\}$ со скалярным произведением
	$$
	(f, g) = \int\limits_0^1 f(t) \overline{g(t)} dt
	$$
	Также $\varepsilon_2 = \mathbb C$.
	Пусть $A : \varepsilon_1 \to \varepsilon_2$, $A(f)=f'(0)\; \forall f \in \varepsilon_1$.
	Конечность нормы --- критерий непрерывности.
	У этого оператора норма бесконечность: возмём, например, $f_n(x) = \sin{nx} 	\in \varepsilon_1$
	$$
	A(f_n) = n
	$$
	По норме это будет
	$$
	\|A(f_n)\| = |n|
	$$
	Рассмотрим теперь норму $f_n$:
	$$
	\|f_n\| = \sqrt{ \int \limits_0^1 \sin^2{nx} dx} \le 1 \Rightarrow \|A\| = \infty
	$$
	Иначе говоря
	$$
	\|A\| \ge \frac{|n|}{\|f_n\|} \ge n \to \infty
	$$
	Или по-другому $g_n = \frac{1}{\sqrt{n}} f_n$ по в $\varepsilon_1$ стремится к 	нулю
	$$
	\|g_n\|_{\varepsilon_1} \le \frac{1}{\sqrt{n}}
	$$
	Тогда $A(g_n) = \sqrt{n}$, поэтому
	$$
	\|Ag_n\| = \sqrt{n} \to \infty
	$$
\end{Prim}

Дадим теперь два других определения операторной нормы.
\begin{gather}
	\label{first}
	\|A\| = \sup\limits_{f \ne 0}{\frac{\|Af\|}{\|f\|}}\\
	\label{second}
	\|A\| = \sup\limits_{\|f\| = 0} \|Af\|\\
	\label{third}
	\|A\| = \sup\limits_{\|f\| \le 1}\|Af\|
\end{gather}
Покажем их равенство.
(\ref{first}) $\ge$ (\ref{second}), так как $\|f\| = 1$ является сужением.
С другой стороны $\sup\left\|A\frac{f}{\|f\|}\right\| \le $ (\ref{second}) $\Rightarrow$  (\ref{first}) $=$ (\ref{second}).
(\ref{third}) $\le$ (\ref{second}) так как при $f \ne 0$ и $\|f\| \le 1$ имеем 
$$
\|Af\| =\|f\| \left\|A\frac{f}{\|f\|}\right\|
$$
где $\left\|A\frac{f}{\|f\|}\right\| \le \sup\limits_{\|\phi\| = 1} \|A\phi\|$.
Но (\ref{second}) $\le$ (\ref{third}) так как является сужением, поэтому (\ref{second}) $=$ (\ref{third}).

\begin{Prim}
    Пусть $\varepsilon_1 = \mathbb C^n$, $\varepsilon_2 = \mathbb C^m$.
    $A : \mathbb C^n \to \mathbb C^m$ задаётся комплексной матрицей $m \times n$.
    $Af \in \mathbb C^m\; \forall f \in \mathbb C^n$ есть умножение матрицы на столбец.
    $$
    \|Af\|^2_{\mathbb C^m} = \overline{Af}^T Af = \overline{f}^T \overline{A}^T A f
    $$
    где обозначили $M = \overline{A}^T A$. 
    $M^*=\overline{M}^T=M \Rightarrow  M \in \mathbb C^{n \times n}$.
    Следовательно $\exists U : \mathbb C^n \to \mathbb C^n$ --- унитарная матрица, то есть сохраняющая норму.
    $$
    U^{-1}MU =
    \left(
        \begin{array}{ccc}
            \lambda_1 & \phantom{x} & \phantom{x}\\
            \phantom{x} & \ddots & \phantom{x}\\
            \phantom{x} & \phantom{x} & \lambda_n
        \end{array}
    \right), \quad \lambda_i \in \mathbb R
    $$
    Заметим, что $\overline{f}^TMf = \|Af\|^2 \ge 0$.
    $\|Uf\|=\|f\|$, поэтому можно перейти к базису из собственных векторов.
    $f=Ug$, тогда
    $$
    \|Af\|^2 = \overline{Ug}^TM Ug = \overline{g}^T \overline{U}^T M Ug
    $$
     но $U$ --- унитарная, следовательно
    $U^{-1} = U^* = \overline{U}^T \Rightarrow$
    $$
    \overline{U}^T M U = U^{-1} M U =
    \left(
        \begin{array}{ccc}
            \lambda_1 & \phantom{x} & \phantom{x}\\
            \phantom{x} & \ddots & \phantom{x}\\
            \phantom{x} & \phantom{x} & \lambda_n
        \end{array}
    \right)
    $$
    $\Rightarrow \overline{g}^T \overline{U}^T M Ug = \sum\limits_{i = 1}^n \lambda_i |g_i|^2$.
    Обозначим теперь $\lambda_{max} = \max\lambda_i$, тогда 
    $$
    \sum\limits_{i = 1}^n \lambda_i |g_i|^2 \le \lambda_{max}\sum\limits_{i = 1}^n |g_i|^2 = \lambda_{max} \|f\|^2
    $$
    Тогда $\|Af\| \le \sqrt{\lambda_{max}}\|f\|$
    Обозначим $\tilde{g}_k = \delta_{kk_*}$, $\lambda_{max} = \lambda_{k_*}$, $\tilde{f} = U\tilde{g}$ и из унитарности $U$ получаем$\|\tilde{f}\| = \|\tilde{g}\| = 1$ и 
    $$
    \sqrt{\lambda_{max}} = \|A\tilde{f}\| \le \|A\| \le \sqrt{\lambda_{max}} \Rightarrow \|A\| = \sqrt{\lambda_{max}(\overline{A}^T A)}
    $$
\end{Prim}

\begin{Prim}
    $\varepsilon_1 = \varepsilon_2 = L_2(G) = H$ --- гильбертово пространство, где $G \in \mathbb R^m$ --- измеримое множество.
    Пусть $A : L_2(G) \to L_2(G)$, 
    $$
    (Af)(x) = \int_G K(t, x) f(t) dt
    $$
    где $K(t, x)$ --- интегральное ядро.
    $K \in L_2(G \times G) $ 
    $$
    \|K\|^2_{L_2{G \times G}} = \iint_{G \times G} |K|^2 dt dx \le +\infty
    $$
    Рассмотрим норму $\|Af\|$
    $$
    \|Af\|^2 = \int_{G} |(Af)(x)|^2 dx = \int_G dx \left| \int_G dt K(t, x) f(t) \right|^2
    $$
    где модуль интеграла по Коши-Буняковскому в $L_2$ 
    меньше или равен $\int_G |K(t, x|^2 dt \int_G |f(t)|^2 dt$, поэтому
    $$
    \|Af\|^2 \le \left(\iint_{G \times G} dx dt |K|^2 \right)\|f\|^2
    $$
    Итого оценили операторную норму:
    $$
    \|Af\|_{L_2{G}} \le \|K\|_{L_2(G \times G)}\|f\|_{L_2(G)} \quad \forall f \in L_2(G)
    $$
    $\Rightarrow$ получаем важное соотношение
    $$
    \boxed{\|A\| \le \|K\|_{L_2(G \times G)}}
    $$
\end{Prim}

Свойства операторной нормы, $A, B : \varepsilon_1 \to \varepsilon_2$
\begin{enumerate}
    \item{$\|A + B\| \le \|A\| + \|B\|$}
    \item{$\|\alpha A\| = |\alpha| \|A\|$}
    \item{$\|A\| = 0 \Leftrightarrow Af = 0 \; \forall f \in \varepsilon_1$}
\end{enumerate}
Докажем эти свойства.
\begin{enumerate}
    \item{$\|(A + B)f\| \le \|A(f)\| + \|B(f)\| \le (\|A\| + \|B\|)\|f\| \Rightarrow \|A + B\| \le \|A\| + \|B\|$}
    \item{$\|(\alpha A)f\| = \|\alpha A(f)\| = |\alpha| \|A(f)\|$, $\sup \limits_{\|f\| = 1} \|(\alpha A)f\| = |\alpha| \sup \limits_{\|f\| = 1} \|Af\|$}
    \item{Очевидно}
\end{enumerate}
Также можно выделить как отдельное свойство 4.
Пусть заданы два оператора $A : \varepsilon_1 \mapsto \varepsilon_2$, $B : \varepsilon_2 \mapsto \varepsilon_3$, Обозначим оператор $T = B \bullet A$, действующий как $T(f) = B(A(f))\; \forall f \in \varepsilon_1$.
Тогда 
$$
\|T(f)\| = \|B(A(f))\| \le \|B\|\|A(f)\| \le \|B\|\|A\|\|f\| \Rightarrow  \boxed{\|T\| \le \|B\|\|A\|}
$$
\newtheorem{Sled1}{Следствие}
\begin{Sled1}
$A : \varepsilon \to \varepsilon$, $A$ --- линейный непрерывный оператор.
Тогда можно формально рассмотреть $A^n f = A(A(\dots A(f)) \dots )$, где $A$ 
применён  $n$ раз --- произведение в пространстве линейный операторов.
\end{Sled1}

Естественно получается $\|A^n\| \le \|A\|^n$ по индукции из пункта 4.


\begin{Nabl}
	$F : H \mapsto \mathbb C$ --- линейный непрерывный функционал.
	По теореме Риса-Фреше $\exists! h \in H\colon F(f) = (f, h)\; \forall f \in H$.
	Следовательно $\|F\|_{\text{операторная}} = \|h\|_H$.
	Получаем изометрический (так как сохраняет норму) изоморфизм между 	гильбертовым пространством и пространством непрерывных линейных 	операторов.
	В квантмехе это называют отождествление гильбертового пространства и 	наблюдателей над ним.
	Если ввести $H^* = \{\text{все }F : H \mapsto \mathbb C \text{ линейные}\*\text{ и непрерывные функционалы}\}$,
	то имеется линейная биекция (изоморфизм) сохраняющая норму по теореме Риса-Фреше.
	Норма сохраняется так как
	$$
	\|F\| = \sup \limits_{\|f\| = 1}|(f, h)| = \|h\|
	$$
	в прямую сторону по неравенству Коши-Буняковского, а в обратную
	если взять вектор $f = \frac{h}{\|h\|}$ при $h \ne 0$, то результат будет не 	меньше $\|h\|$.
\end{Nabl}

Два типа сходимости для последовательности операторов.
$\{A_n\} : \varepsilon_1 \to \varepsilon_2$, $\{T_n\} : \varepsilon_2 \to \varepsilon_1$ --- линейные непрерывные функционалы.

Говорят, что $\|A_n\| \to T$ по операторной норме, если $\|A_n - T\| \to 0\; n\to \infty$.
Фактически это равномерная сходимость на сфере или на шаре, при чём на любом.
$$
\|A_n - T\| \to 0 \Leftrightarrow A_n(f) \rightrightarrows T(f)\quad \|f\| < R\; \forall R$$
потому что
$$
\|A_n(f) - T(f)\| \le \|A_n - T\|\|f\|
$$
где $\|A_n - T\| \to 0$ и $\|f\| \le R$.
И наоборот, если
$$
\forall n \ge N(\varepsilon)\; \forall \|f\| \le 1\; \|A_n f - Tf\| \le \varepsilon
$$ 
то $\sup \limits_{\|f\| \le 1} \|A_n f -Tf\| = \|A_n - T\|$, получили сходимость
по операторной норме.
В итоге сходимость по норме синоним сходимости на произвольном шаре.

$A_n \to T$ сходится поточечно, если $\|A_nf-Tf\| \to 0\; \forall f \in \varepsilon_1$.
Ясно, что если $A_n \to T$ по операторной норме, то очевидно сходится и поточечно.
Обратное неверно.


\begin{Upr}
    Придумать пример, когда $A_n : \varepsilon_1 \to \varepsilon_2$ линейно непрерывный поточечно сходится к разрывному.
\end{Upr}

Пусть $\|T\| = \lim \limits_{n \to \infty} \|A_n\|$ тогда
$$
\|A_n - A_m\| \le \|A_n - T\| + \|A_m - T\|
$$
Получается, что
$$
\|T\| - \|A_n\| \le \|T - A_n\| \to 0
$$ 
если есть сходимость по операторной норме.

\begin{Theor}[Банаха-Штейнгаусса]
    Пусть $H$ --- гильбертово пространство, $\varepsilon$ --- евклидово, $A_n : H \mapsto \varepsilon$, $A_n$ поточечно сходится к $T$,
    где $T : H \mapsto \varepsilon$
    линейный оператор, тогда $T$ --- линейный непрерывный оператор, $\{A_n\}$ --- ограниченная числовая последовательность, $\|T\| \le \varliminf\limits_
    {n\to \infty}\|A_n\|$.
\end{Theor}
\begin{Proof}
    Шаг первый. Если $\forall f \in H\; \{A_nf\}$ ограничена в $\varepsilon$, тогда $\{\|A_n\|\}$ ограничена.
    Докажем это.
    Посмотрим на множество $\Gamma_N = \{f \in H \mid \|A_n f\| \le N \quad \forall n \in \mathbb N\}\; N \in \mathbb N$.
    Так как последовательность $\{\|A_nf\|\}$ ограничена в $\varepsilon$, то такая последовательность не будет превосходить какого-нибудь числа.
    Следовательно если взять числа $N$ и смотреть на функции, которые отвечают $\Gamma_N$, то перебирая все $N$ можно перебрать все функции.
    $\bigcup\limits_{N = 1}^\infty \Gamma_N = H$.
    Теперь, воспользуясь полнотой, попытаемся доказать, что хотя бы в одной из этих множеств попадает шарик.
    Если $\exists f_0 \in H$ и $\exists N_0$ и $r_0 > 0$ так, что $B_{r_0}(f_0)=\{f \in H \mid \|f - f_0\| \le r_0\} \subset \Gamma_{N_0}$, тогда $\|A_n f\| \le N_0$
    если $\forall f\; \|f - f_0\| \le r_0$.
    Если взять $\|w\| = 1$ и рассмотреть $\|A_n w\|$, то $f_0 + r_0 w \in B_{r_0}(f_0)$ (к $f_0$ прибавили вектор единичной длины, умноженный на радиус шара
    и остались в нём), тогда, учитывая $\|A_n f_0\| \le R_0$
    $$
    \frac{1}{r_0}\|A_n(f_0 + r_0 w) - A_n f_0\| \le \frac{1}{r_0}(N_0 + \|A_n f_0\|)\le \frac{N_0 + R_0}{r_0}
    $$
    для любого вектора на единичной сфере.
    Следовательно 
    $$
    \|A_n\| \le \frac{N_0 + R_0}{r_0}
    $$
    Если удастся в $\Gamma_N$ впихнуть шарик, то можно оценить любой элемент сферы.
    Пусть в любое $\Gamma_N$ нельзя впихнуть никакой шар положительного радиуса.
    Тогда рассмотрим шар в центре 0 радиуса 1 и рассмотрим $\Gamma_1$, в который нельзя впихнуть какой-либо шар.
    Рассмотрим разность этого открытого шара и $\Gamma_1$.
    Получим открытое множество, в нём любая точка входит вместе с окрестностью, а значит можно выбрать шар радиуса $r_1 < \frac12$, непересекающий $\Gamma_1$.
    Рассмотрим теперь этот шар и множество $\Gamma_2$, в которое также нельзя запихнуть ни один шар.
    Опять смотрим их разность (открытое множество) и выбираем шар $r_2 < \left(\frac12\right)^2$ и так далее.
    Получаем последовательность вложенных шаров $B_{r_1}(f_1) \supset B_{r_2}(f_2)$ и $\forall k\; B_{r_k} \cap \Gamma_k = \varnothing$, а радиусы стремяться к нулю.
    Центры этих шаров образуют фундаментальную последовательность
    $$
    \|f_k - f_{k + p}\| \le \frac{1}{2^k} \le \varepsilon\; \forall k \ge k(\varepsilon)
    $$
    Пользуяся полнотой получаем, что ряд $f_k \to f_*$. сходится в $H$.
    Получаем парадокс: $f_{k+p} \in B_{r_k}(f_k)$, при $p \to \infty\; f_{k+p} \to f_* \Rightarrow  f_* \in B_{r_k}(f_k)\; \forall k$.
    Но каждый такой шар построен так, что $B_{r_k} \cap \Gamma_k = \varnothing$, следовательно $f_* \notin \Gamma_k\; \forall k$, но всё пространство есть объединение
    в том числе и $f_*$, получили противоречие с условием.

    Следствие шага 1: Если $\sup\limits_{n \in N}\|A_n\| = \infty$, $A_n$ --- линейный непрерывный оператор, то 
    $$\exists f_* \in H\colon
    \sup \limits_{n \in \mathbb N}\|A_n f_*\| = \infty
    $$

    Шаг второй. Если есть поточечная сходимость из полного пространства, то тогда поточечный предел --- непрерывный ограниченный оператор.
    $A_n f \to T f\; \forall f$ в $\varepsilon$, следовательно $\{A_n f\}$ ограниченна в $\varepsilon\; \forall f\; \Rightarrow$ по шагу один $\sup \limits_{n \in
    \mathbb N}\|A_n\| < +\infty$.
    Тогда 
    $$
    \|Tf + A_n f - A_n f\| \le \|Tf - A_n f\| + \|A_n f\|
    $$
     где $\|Tf - A_n f\| \le 1$ при $n \ge N(f)$ и $\|A_n f\| \le \|A_n\| \|f\|$. 
     Пусть $\|f\| = 1$ ($f$ с
    единичной сферы, а $\|A_n\| \le R$).
    Следовательно $\|Tf\| \le 1 + R\; \forall \|f\| = 1$ и значит $\|T\| \le 1 + R$.
    Поэтому поточечный предел последовательности будет непрерывным.
    Во-вторых, $\forall \|f\| = 1$, $\forall \varepsilon > 0\; \exists N(\varepsilon, f)\colon \forall n \ge N(\varepsilon, f)\; \|A_n f - Tf\| \le \varepsilon$
    $\Rightarrow$
    $$
    \|Tf\| \le \varepsilon + \|A_n\|\|f\|
    $$
    Нижний предел последовательности с добавкой превосходит $\|A_n\|$, начиная с некоторого $n$:
    $$
    \forall \varepsilon > 0\; \forall M(\varepsilon) > 0\; \exists n \ge M(\varepsilon)\; \varliminf\limits_{k \to \infty}\|A_k\| + \varepsilon \ge \|A_n\|
    $$
    Если возмём $M = N(\varepsilon, f)$, то 
    $$
    \varepsilon + \|A_n\| \|f\| \le \varliminf\limits_{k \to \infty}\|A_k\| + 2 \varepsilon
    $$
    Отсюда взяв супремум по $f$ получаем, что 
    $$
    \|T\| \le \varliminf\limits_{k \to \infty}\|A_k\| + 2 \varepsilon\quad \varepsilon \to +0
    $$
    Доказательство закончено.
\end{Proof}


\begin{Prim}
    $H$ --- гильбертово пространство, $\{e_n\}_{n = 1}^{\infty}$ --- ортонормированный базис в $H$.
    $\forall f \in H$
    $$
     f = \sum\limits_{k = 1}^{\infty}(f, e_k)e_k
    $$ 
    это ни что иное $P_k(f)$ --- ортопроектор на линейную оболочку $e_k$.
    Очевидно $\|P_k\| = 1$.
    Возникает ряд из проекторов 
    $$
    I = \sum\limits_{k = 1}^{\infty}P_k
    $$
    где $If = f$ --- тождественый оператор из $H$ в $H$, $\|I\| = 1$.
    Это обозначение символизирует собой ряд Фурье:
    $$
    f = \sum\limits_{k = 1}^{\infty}P_k(f)
    $$
    Это справедливо для любого $f$, следовательно получается, что 
    $$
    S_n = \sum\limits_{k = 1}^{n}P_k \to I
    $$ поточечно, потому что
    $$
    S_n(f) = \sum\limits_{k = 1}^{n}(f, e_k)e_k \to f \quad \text{в }H \text{ при } n \to \infty
    $$
    А сходимости по норме здесь нет.
    Действительно 
    $$
    \|I - \sum\limits_{k = 1}^{n}P_k\| \ge \|(I - \sum\limits_{k = 1}^{n}P_k)e_{n + 1}\|
    $$
    где $e_{n+  1}$ с единичной сферы, а супремум по единичной сфере даёт норму.
    Но $Ie_{n + 1} = e_{n + 1}$, а $P_k e_{n + 1} = 0$ при $k < n + 1$.
    Получаем 
    $$
    \|I - \sum\limits_{k = 1}^{n}P_k\| = \|e_n\| = 1\quad \forall n
    $$
    То есть никакой сходимости по норме нет.
    Но поточечная есть, она из неё следует факт, что последовательность частичных сумм $S_n$ ограничена.
    Оценивать норму суммой норм проекторов плохо, так как оценка стремится к бесконечности.
    Лучше воспользоваться теоремой Банаха-Штейнгаусса: $S_n$ по норме меньше некого числа $R$.
\end{Prim}

\section*{Спектр и резольвентное множество линейного непрерывного оператора в гильбертовом пространстве $A : H \mapsto H$.}

Введём оператор $A_\lambda = A - \lambda I$, $\lambda \in \mathbb C$.
Резольвентное множество 
$$
\rho(A) = \{\lambda \in \mathbb C \mid \exists (A_\lambda)^{-1} : H \mapsto H\}
$$

\begin{Zam}
    Пусть $\varepsilon_1, \varepsilon_2$ --- два евклидовых пространства и линейный оператор $T : \varepsilon_1 \mapsto \varepsilon_2$.
    Обратный оператор отображает образ $T^{-1} : \Im T \mapsto \varepsilon_1 \Leftrightarrow \ker T = \{0\}$.
    Тогда существует обратный $T^{-1}:\varepsilon_2 \rho \varepsilon_1\; \Leftrightarrow\; \ker T = \{0\}$ и $\Im T = \varepsilon_2$.
\end{Zam}

Тогда можно переписать определение резольвентного множества как 
$$
\rho(A) = \{\lambda \in \mathbb C \mid \ker A_\lambda = 0\text{ и }\Im A_\lambda = H \}
$$
Более того в гильбертовом пространстве обратный оператор автоматически непрерывен.

\begin{Zam}[Теорема Банаха об обратном операторе]
    $T:H_1 \mapsto H_2$ --- линейный непрерывный оператор, где $H_{1,2}$ --- гильбертовы пространства.
    Тогда $\exists! T^{-1} : H_2 \mapsto H_1$ линейный и непрерывный если, и только если $\ker T = 0$ и $\Im T = H_2$.
\end{Zam}

    $\lambda \in \rho(A) \Leftrightarrow \exists(A_\lambda)^{-1} : H \mapsto H$ --- непрерывный оператор.
    $(I - \mu A) = -\mu A_{\frac{1}{\mu}}$, то 
    $$
    \frac{1}{\mu} \in \rho(A) \Leftrightarrow \exists (I - \mu A)^{-1} = -\frac{1}{\mu}(A_{\frac{1}{\mu}})
    $$
    где обозначим $(I - \mu A)^{-1} = \rho_A(\mu)$ --- резольвента.
\begin{Opr}
    Спектр оператора $\sigma(A) = \mathbb C \setminus \rho(A)$. Пусть $\lambda \in \sigma(A)$.
    Это эквивалентно одному из двух случаев:
    либо $\ker A_\lambda \ne \{0\}$ --- точечный спектр, состоящий из собственных значений ($\sigma_P(A)$) (точечный спектр может образовывать множество мощности континум),
    либо $\ker A_\lambda = \{0\}$, но $\Im A_\lambda \ne H$ --- непрерывный спектр ($\sigma_C(A)$) (непрерывный спектр может быть из отдельных точек).
    В случае непрерывного спектра обратный оператор существует $A_\lambda^{-1} : \Im A_\lambda \mapsto H$ и даже может оказаться непрерывным, если (и только если) образ его
    замкнут.
\end{Opr}

\begin{Prim}[когда обратный оператор непрерывный есть, а точка в спектре]
    Пусть $H$ с ортонормированным базисом $\{e_k\}$.
    $$
    f = \sum \limits_{k = 1}^{\infty} \alpha_k(f) e_k
    $$
    где $\alpha_k(f)=(f, e_k)$.
    Пусть оператор $A$ производит сдвиг: $Af = \sum \limits_{k = 1}^{\infty} \alpha_k(f) e_{k+1}$.
    Естественно 
    $$
    \|f\| = \|Af\| = \sqrt{\sum \limits_{k = 1}^{\infty} |\alpha_k(f)|^2}
    $$
    что получается из равенства Парсеваля.
    Для этого оператора
    $\ker A = \{0\}$, $\Im A = (\Lin e_1)^{\perp}$.
    Обратный оператор
    $A^{-1} : (\Lin e_1)^{\perp} \mapsto H$. Пусть $g \in (\Lin e_1)^{\perp}$, тогда
    $$
    g = \sum \limits_{k = 2}^{\infty}\underbrace{\beta_k}_{(g, e_k)}e_k
    $$ 
    отображается в 
    $$
    \sum\limits_{k = 1}^{\infty} \beta_{k + 1}e_k
    $$
    действием оператора $A^{-1}$.
    Получается $A^{-1}g = f \Leftrightarrow Af = g$, естественным образом $\|A^{-1}g\| = \|g\|\; \forall g \in (\Lin e_1)^{\perp}$, следовательно $\|A^{-1}\| = 1$.
\end{Prim}

\begin{Theor}[фон Неймана]
    Если $T : H \mapsto H$ --- линейный и непрерывный так, что ряд 
    $$
    \sum_{n = 0}^\infty \|T^n\|
    $$
    сходится, тогда
    $$
    S_N = \sum_{n = 0}^N T^n
    $$
    сходится по операторной норме
    к некоторому оператору $S$ --- линейный непрерывный оператор
    $$
    S = \sum_{n = 0}^\infty T^n
    $$
    причём $S = (I - T)^{-1}$.
\end{Theor}
\begin{Proof}
    Убедимся, что $S_N$ обладает сходимостью.
    $S_N f$ --- фундаментальная последовательность в $H\; \forall f \in H$, так как,
    учитывая $\|T^n\| \to 0$ 
    $$
    \|S_N f - S_{N+P} f\| = \|\sum_{n = N + 1}^{N + P} T^n f\|
    \le \sum_{n = N + 1}^{N + P}\|T^n\|\|f\| 
    $$
    $\Rightarrow$ $\|S_N f - S_{N+P} f\| \to 0\; N \to \infty$ равномерно по $P$.
    Раз фундаментальна в гильбертовом, значит сходится 
    $$
    S_N f \to Sf
    $$
    Тогда 
    $$
    \|S_f\| = \lim \limits_{N \to \infty}\|S_N f\| \le \lim \limits_{N \to \infty} \sum_{k = 1}^N \|T\|^k\|f\| \le \sum_{k = 1}^{\infty} \|T^k\| \|f\|
    $$
    Следовательно 
    $$
    \|S\| \le \sum \limits_{k = 1}^{\infty}\|T^k\|
    $$ 
    отсюда $S$ --- линейный непрерывный оператор.

    Теперь покажем, что он обратный.
    Надо доказать, что 
    $$
    \left\{
        \begin{aligned}
            &(I - T)S = I \Rightarrow \Im(I - T) = H\\
            &S(I - T) = I \Rightarrow \ker(I - T) = 0
        \end{aligned}
    \right\} \Rightarrow \exists (I - T)^{-1} : H \mapsto H
    $$
    Рассмотрим $\forall f \in H$ действие оператора $(I - T)Sf$, где $I - T$ --- непрерывный
    \begin{multline*}
    (I - T)Sf = (I - T)(\lim \limits_{N \to \infty} S_N f) = \\
    = \lim \limits_{N \to \infty}((I - T)S_N f) 
    = \lim \limits_{N \to \infty}((I - T) \times \\
    \times \sum\limits_{n = 0}^N T^k f) = \lim \limits_{N \to \infty}(f - T^{N+1}f)
    \end{multline*}
    Но
    $$
    \|T^{N+1}f\| \le \|T^{N+1}\|\|f\| \to 0
    $$
    Отсюда 
    $$
    (I - T)S = I
    $$
    Рассмотрим теперь действие оператора $S(I - T)f$
    $$
    S(I - T)f = \lim \limits_{N \to \infty}(S_N(I - T)f) = \lim \limits_{N \to \infty}((I - T^{N+1})f) = f
    $$
    Таким образом теорема фон Неймана доказана.
\end{Proof}
\newtheorem{Sled}{Следствия}
\begin{Sled}
    Пусть $A : H \mapsto H$ --- линейный и непрерывный, тогда
    \begin{enumerate}
        \item{$\forall \lambda \in \mathbb C$ такой, что $|\lambda| > \|A\| \Rightarrow  \lambda \in \rho(A)$.
            При этом 
            $$
            (A_\lambda)^{-1} = -\sum \limits_{n = 0}^{\infty}\frac{A^k}{\lambda^{k+1}}
            $$
            причём ряд сходиться по операторной норме.}
        \item{$\rho(A)$ --- открыто в $\mathbb C$}
        \item{Функция от $\lambda\; (A_\lambda)^{-1}$ непрерывна на $\rho(A)$ по операторной норме}
        \item{$\forall \lambda \in \rho(A)\; \exists \lim\limits_{\text{по операторной норме}}\frac{(A_{\lambda + \Delta \lambda})^{-1} - (A_\lambda)^{-1}}{\Delta\lambda} =
            ((A_\lambda)^{-1})^2$}
    \end{enumerate}
\end{Sled}
\end{document}
