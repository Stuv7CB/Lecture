%Надо добавить примеры и упражнения
\documentclass[14pt]{extarticle}
\usepackage{../preamble}
\begin{document}
\section*{Спектр и резольвентное множество линейного непрерывного оператора в гильбертовом пространстве $A : H \mapsto H$.}
Пусть $A : H \mapsto H$ --- линейный и непрерывный оператор.
Тождественный оператор $I : H \mapsto H$ ($If = f\;\forall f \in H$).
Далее $\forall \lambda \in \mathbb C\;A_\lambda = A - \lambda I$.
Резольвентное множество
$$
\rho(A) = \{ \lambda \in \mathbb C \mid \exists (A_\lambda)^{-1} : H \mapsto H \}
$$
где $\Alo$ --- линейный и непрерывный и по теореме Банаха об обратном операторе
$$
\rho(A) = \{ \lambda \in \mathbb C \mid \ker A_\lambda = 0, \Im A_\lambda = H \}
$$
Тогда $\sigma(A) = \mathbb C \setminus \rho(A)$ --- спектр, 
$$
\sigma(A) = \left \{ \lambda \in \mathbb C \mid 
\left [
    \begin{aligned}
        &\ker A_\lambda \ne 0\\
        &\left \{
            \begin{aligned}
                &{\ker \Al = 0}\\
                &{\Im \Al \ne H}
            \end{aligned}
            \right.
    \end{aligned}
\right.\right\}
$$
Точечный спектр оператора $A;$ это $\lambda \in \sigma_p(A)= \left \{ \lambda \in \mathbb C \mid \ker A_\lambda \ne 0 \right\}$ называют собственными значениями $A$.
Непрерывный спектр в свою очередь 
$$\sigma_C(A) =  \left \{ \lambda \in \mathbb C \mid 
\left\{
    \begin{aligned}
                &{\ker \Al = 0}\\
                &{\Im \Al \ne H}
    \end{aligned}
\right.\right \}
$$
Очевидно $\sigma(A) = \sigma_p(A) \cup \sigma_C(A)$.
Резольвента $R_A(\lambda) = \Rez : H \mapsto H$, также используется определение
$\forall \lambda \ne 0\colon \dfrac{1}{\lambda} \in \rho(A)\;-\dfrac{1}{\lambda}(A - \dfrac{I}{\lambda})^{-1} = -\dfrac{1}{\lambda}(A_{\dfrac1{\lambda}})^{-1}$

\begin{Theor}
    Пусть $A : H \mapsto H$ --- линейный и непрерывный, тогда
    \begin{enumerate}
        \item{$\forall \lambda \in \mathbb C$ такой, что $|\lambda| > \|A\| \Rightarrow \lambda \in \rho(A)$.
            При этом $(A_\lambda)^{-1} = -\sum \limits_{n = 0}^{\infty}\dfrac{A^k}{\lambda^{k+1}}$, причём ряд сходиться по операторной норме.}
        \item{$\rho(A)$ --- открыто в $\mathbb C$ (очевидно следует, что спект --- замкнутое множество, а так же $\sigma(A) \subset \{ \lambda \in \mathbb C \mid |\lambda
            \le \|A\|\}$, поэтому спектр --- компакт).}
        \item{Функция от $\lambda\;(A_\lambda)^{-1}$ непрерывна на $\rho(A)$ по операторной норме}
        \item{$\forall \lambda \in \rho(A)\;\exists \lim\limits_{\text{по операторной норме}}\dfrac{(A_{\lambda + \Delta \lambda})^{-1} - (A_\lambda)^{-1}}{\Delta\lambda} =
            ((A_\lambda)^{-1})^2$}
    \end{enumerate}
\end{Theor}
\begin{Proof}
    \begin{enumerate}
        \item{$|\lambda| > \|A\|$, то
        $$
        \Al = -\lambda(I - \dfrac{A}{\lambda}
        $$
        где $\dfrac{A}{\lambda}$  обозначим $T$, $\|T\| = \dfrac{\|A\|}{|\lambda|}<1$
        По теореме Неймана $\exists$ оператор непрерывный в $H$
        $$
        \Alo = -\dfrac{1}{\lambda}(I - \dfrac{A}{\lambda})^{-1} =
            -\sum_{n=0}^\infty \dfrac{A^n}{\lambda^{n+1}}
        $$
        ряд сходится по операторной норме.}
        \item{Так как существует $\exists \Alo$ линейный и непрерывный.
        $\forall \lambda \in \rho(A)$ рассмотрим 
        $$
        A_{\lambda + \Delta\lambda} = \Al - \Delta\lambda I = \Al(I - 
            \Delta\lambda\Alo)
            $$.
            Обозначим $T = \Delta\lambda\Alo$, тогда
            $$
            \exists \Delta\lambda\colon \|T\|=|\Delta\lambda|\|\Alo\| < 1
            $$
            Тогда 
            $$
            \Delta\lambda < \dfrac{1}{\|\Alo\|}
            $$
             следовательно $\exists (A_{\lambda+\Delta\lambda})^{-1} : H \mapsto H$ линейный и непрерывный, следовательно
            $\lambda + \Delta\lambda \in \rho(A)$.}
        \item{$\lambda \in \rho(A)\colon \lambda \to \Alo$.
        Рассмотрим
            $$
            (A_{\lambda + \Delta\lambda})^{-1} - \Alo = (I - \Delta\lambda\Alo)\Alo - \Alo
            $$
            расскладываем по Нейману 
            $$
            A_{\lambda + \Delta\lambda})^{-1} - \Alo = \sum_{n=1}^\infty
            (\Delta\lambda)^n(\Alo)^{n+1}
            $$
            Учтём, что $\|(\Alo)^{n+1}\| \le \|\Alo\|^{n+1}$.
            Отсюда 
            $$
            \|(A_{\lambda + \Delta\lambda})^{-1} - \Alo\| \le \sum_{n=1}^\infty |\Delta\lambda|^n\|(\Alo)^{n+1}\| \le
            |\Delta\lambda| \dfrac{\|\Alo\|^2}{1-|\Delta\lambda\|\Alo\|}
            $$
            где $|\Delta\lambda| \dfrac{\|\Alo\|^2}{1-|\Delta\lambda\|\Alo\|} \to 0\; \Delta\lambda \to 0$.
            Следовательно непрерывен по операторной норме}
        \item{$\lambda \in \rho(A)\;|\Delta\lambda| < \dfrac{1}{\|\Alo\|}$, тогда
        получим тождество Гильберта 
        $$
        (A_{\lambda + \Delta\lambda})^{-1} - \Alo =
            \Alo(\Al - A_{\lambda + \Delta\lambda})(A_{\lambda + \Delta\lambda})^{-1} 
            $$
            где очевидным образом $\Al - A_{\lambda + \Delta\lambda} = \Delta\lambda I$.
            По этому 
            $$
            (A_{\lambda + \Delta\lambda})^{-1} - \Alo = \Delta\lambda \Alo (A_{\lambda + \Delta \lambda})^{-1}
            $$
            Рассмотрим теперь предел
            \begin{multline*}
            \lim \limits_{\Delta\lambda \to 0 \text{ по операторной норме}} \dfrac{(A_{\lambda + \Delta\lambda})^{-1} - \Alo}{\Delta\lambda} = \\
            = \lim \limits_{\Delta\lambda \to 0 \text{ по операторной норме}}  \Alo (A_{\lambda + \Delta\lambda})^{-1}
            \end{multline*}
            А по п.3 $(A_{\lambda + \Delta\lambda})^{-1} \to \Alo$ поэтому
            $$
             \lim \limits_{\Delta\lambda \to 0 \text{ по операторной норме}} \dfrac{(A_{\lambda + \Delta\lambda})^{-1} - \Alo}{\Delta\lambda} = \Alo \Alo
             $$
            }
    \end{enumerate}
\end{Proof}
\begin{MathCl}{Следствие}
    $\sigma(A) \ne \varnothing$, иначе говоря спектр --- непустой компакт в $\mathbb C$. Определим спектральный радиус $r(A) = \max |\lambda|$.
    Тогда $r(A) = \lim \limits_{n \to \infty} \sqrt[n]{\|A\|^n} \le \|A\|$.
\end{MathCl}
\begin{Proof}
    1) Если вдруг спектр пуст $\sigma(A) = \varnothing$, тогда $\forall f, g \in H\;\forall \lambda \in  \rho(A) = \mathbb C$.
    Рассмотрим $F(\lambda) = (\Alo f, g)$ при $|\lambda| > \|A\|$
    $$
    F(\lambda) = (\Alo f, g) = (-\sum_{n = 0}^{\infty}
    \dfrac{A^n f}{\lambda^{n + 1}}, g)
    $$
    где сумма сходится в $H$, а скалярное произведение непрерывно $H$.
    $$
    F(\lambda) = -\sum_{n = 0}^{\infty} \dfrac{(A^n f, g)}{\lambda^{n + 1}} = O(\dfrac{1}{\lambda}) \quad \lambda \to \infty
    $$
    но в силу п. 4 $F(\lambda)$ регулярная.
    Действительно, $\forall \lambda \in \mathbb C, \forall \Delta\lambda \ne 0$
    $$
    \lim\limits_{\Delta\lambda \to 0} \dfrac{F(\lambda + \Delta\lambda) - F(\lambda)}
    {\Delta\lambda} = \lim\limits_{\Delta\lambda \to 0} (\dfrac{(A_{\lambda + \Delta\lambda})^{-1}f - \Alo f}{\Delta\lambda}, g)
    $$
    где первая компонента скалярного произведения сходится по операторной норме к $(\Alo)^2f$.
    По определению $F'(\lambda) = \lim\limits_{\Delta\lambda \to 0} \dots = ((\Alo)^2 f, g)$ --- непрерывно в $\mathbb C$.
    Следовательно $F(\lambda)$ --- целая функция, далее $F(\lambda) \to 0\;\lambda \to \infty$, тогда по теореме Лиувиля из ТФКП $F(\lambda) \equiv 0$,
    следовательно $\ker \Alo = H$, с другой стороны $\ker \Alo = 0$ так как $\Alo$ имеет обратный $\Al$ на $H$, следовательно $H = 0$, получили противоречие.
    
    2) По определению $r(A) = \max\limits_{\lambda \in \sigma(A)} |\lambda|$ существует в $\mathbb R$.
    \textbf{Шаг 1}.
    Докажем $\forall \lambda \in \sigma(A) \Rightarrow \lambda^n \in \sigma(A^n)$. 
    Если вдруг это не так, тогда $\lambda^n \in \rho(A^n) \Rightarrow \exists ((A^n)_{\lambda^n})^{-1} : H \mapsto H$ линейный непрерывный оператор. 
    Обозначим его $B$. 
    Тогда
    $$
    (A^n - \lambda^n I) = (A - \lambda I)(A^{n - 1} + \lambda A^{n - 2} + \dots + \lambda^{n - 2} A + \lambda^{n - 1} I)
    $$
    Обозначим скобку как $C : H \mapsto H$.
    $$
    B = C(A - \lambda I)
    $$
    Также
    $$
    A^n - \lambda^n I = \Al C
    $$
    следовательно 
    $$
    (A^n - \lambda^n I) B = I = \Al C B \Rightarrow \Im \Al = H
    $$
    Далее 
    $$
    \Al CBf = f \in H\quad \forall f \in H
    $$
    Следовательно
    $$
    A^n - \lambda^n I = C\Al
    $$
    тогда
    $$
    B(A^n - \lambda^n I) = I = B C \Al \Rightarrow \ker \Al = 0
    $$
    Пусть $f \in \ker \Al \Rightarrow f = B C \Al f = B C(0) = 0$.
    Из $\Im \Al = H$ и $\ker \Al = 0$ следует по определению, что $\lambda \in \rho(A)$.
    Получили противоречие условию, что $\lambda$ в спектре.
    Отсюда $|\lambda^n| \le \|A^n\|$ так как $\rho(A^n) \supset \{\mu \in \mathbb C \mid |\mu| > \|A^n\|\}$.
    Очевидно тогда, что $|\lambda| \le \sqrt[n]{\|A^n\|}$.
    Следовательно
    $$
    |\lambda| \le \varliminf \sqrt[n]{\|A^n\|} \Rightarrow r(A) \le \varliminf \sqrt[n]{\|A^n\|}
    $$.
    \textbf{Шаг 2}.
    Утверждаем, что $\varlimsup \sqrt[n]{\|A\|^n} \le r(A)$.
    Для $\forall f, g \in H$ рассмотрим 
    $$
    F(\lambda) = (\Alo f, g)
    $$
    $F(\lambda) \to 0, \lambda \to \infty$, регулярная во внешности круга $|\lambda| < r(A)$.
    $|\lambda| > r(A) \Rightarrow \lambda \in \rho(A)$. Так как $|\lambda| > \|A\| \ge r(A)$, то по теореме Неймана
    $$
    F(\lambda) = - \sum_0^\infty \dfrac{(A^n f, g)}{\lambda^{n + 1}}
    $$
    С другой стороны
    $$F(\lambda) = \sum_0^\infty \dfrac{c_n}{\lambda^{n + 1}}\quad |\lambda|>r(A)
    $$
    $\Rightarrow$ по теореме единственности разложения в ряд Лорана
    $c_n = - (A^n f, g)\;\forall n \in \mathbb N \cup \{0\}$.
    Таким образом $\forall |\lambda| > r(A)$ получаем $\dfrac{A^n f, g}{\lambda^n}$ --- член сходящегося ряда, следовательно
    $$
    \dfrac{(A^n f, g)}{\lambda^n} \to 0\quad n \to \infty
    $$
    что равносильно $\forall g \in H\;(g, \dfrac{A^n f}{\lambda^n}) \to 0, n \to \infty$.
    $\Phi_n : H \to \mathbb C$ линейный и непрерывный оператор, где $\Phi_n(g) = \dfrac{A^n f}{\lambda^n}$.
    Норма оператора
    $\|\Phi_n\| = \dfrac {\|A^n f\|}{|\lambda^n|}$ сходится к 0 поточечно на $H$.
    Отсюда по теореме Банаха~--- Штейнгаусса $\|\Phi_n\|$ --- ограниченная числа последовательность. $\forall f \in H\;\exists M_f > 0\colon \|\dfrac{A_n f}{\lambda^n}\|
    \le M_f \Rightarrow \{\dfrac{A_n}{\lambda_n}\}$ --- поточечно сходящаяся на $H$ ограниченная последовательность операторов, следовательно
    по теореме Банаха~--- Штейнгаусса $\|\dfrac{A^n}{\lambda^n}\|$ ограниченная числовая последовательность.
    Следовательно $\exists M>0\colon \|\dfrac{A^n}{\lambda^n}\| \le M \Rightarrow \forall |\lambda| > r(A)\;\sqrt[n]{\|A^n\|} \le \sqrt[n]{M}|\lambda|$.
    $\varlimsup\limits_{n \to \infty} \sqrt[n]{\|A^n\|} \le |\lambda| \to r(A) + 0$.
    Получаем $\varlimsup\sqrt[n]{\|A^n\|} \le r(A) \le \varliminf\sqrt[n]{\|A^n\|}$, что и требовалось доказать.
\end{Proof}

\begin{Opr}
    $A : H \mapsto H$ линейно непрерывный оператор, то $A^* : H \mapsto H$ называется сопряжённым к $A$ (эрмитов оператор),
    если $(Af, g) = (f, A^* g)\;\forall f, g \in H$.
\end{Opr}

\begin{Utv}
    $\forall A : H \mapsto H$ линейный непрерывный оператор $\exists! A^* = T\colon (Af, g) = (f, Tg)\;\forall f, g \in H$ и $\|T\| = \|A\|$.
\end{Utv}
\begin{Proof}
    $\forall g \in H$ рассмотрим $f \in H\;f \to (Af, g) = \Phi_g(f)$.
    $\Phi_g : H \to \mathbb C$ линейный и непрерывный.
    Тогда по теореме Рисса~--- Фреше $\exists! h \in H\colon \Phi_g(f) = (f, h) \Rightarrow \exists! T : H \mapsto H\colon \forall g \in H\;Tg = h_g$ и
    $$
    (Af, g) = (f, h_g) = (f, Tg)
    $$
    Далее 
    \begin{multline*}
    (Af, \alpha_1 g_1 + \alpha_2 g_2) = (f, T(\alpha_1 g_1 + \alpha_2 g_2)) =\\ 
    = \overline{\alpha}_1 (Af, g_1) + \overline{\alpha}_2 (Af, g2) =\\
    = (f, \alpha_1 Tg_1) + (f, \alpha_2 Tg_2)\quad \forall f \in H
    \end{multline*}
    Следовательно $T(\alpha_1 g_1 + \alpha_2 g_2) = \alpha_1 T(g_1) + \alpha_2 T(g_2)$, $T$ --- линейный оператор.
    
 $$
 \|Tg\| = \sup\limits_{\|f\| = 1}|(f, Tg)| = \sup\limits_{\|f\| = 1}|(Af, g)| \le \sup\limits_{\|f\| = 1}\|Af\|\|g\| \le \|A\|\|g\|
 $$
    С другой стороны 
    $$
    \|Af\| = \sup\limits_{\|g\| = 1}|(Af, g)| = \sup\limits_{\|g\| = 1}|(f, Tg)| \le \sup\limits_{\|g\| = 1} \|f\| \|Tg\| \le \|f\| \|T\|
    $$.
    Получаем, что $T$ --- линейный и непрерывный оператор и $\|T\| = \|A\|$.
\end{Proof}
\begin{Upr}
    $A^{**} = A\;\forall A : H \mapsto H$.
\end{Upr}

\begin{Theor}[Фредгольма]
    Пусть $A : H \mapsto H$ линейный и непрерывный оператор. Тогда $\ker A = (\Im A^*)^\perp$ и $(\Im A)^\perp = \ker A^*$.
\end{Theor}
\begin{Proof}
Пусть $f \in \ker A\;\Leftrightarrow\;Af = 0\;\Leftrightarrow\;\forall g \in H\;(Af, g) = 0\;\Leftrightarrow\;(f, A^* g) = 0$.
$\forall h \in \Im A^*\;(f, h) = 0\;\Leftrightarrow\;f \in (\Im A^*)^\perp$. Второе утверждение доказывается аналогично.
\end{Proof}

\begin{Utv}
$L \subset H$ --- подпространство, тогда $L \subseteq L^{\perp\perp} = \overline{L}$.
\end{Utv}

\begin{Proof}
    $L^{\perp\perp} \supset L$.
    С другой стороны из непрерывности скалярного произведения в $H$ имеем
    $L^{\perp\perp} \subset \overline{L}$.
    Рассмотрим $\forall f \in L^{\perp\perp}$, тогда $f \in \overline{L}$ и 
    $$
    \overline{L} \oplus \overline{L}^{\perp} = H
    $$ по теореме Рисса об ортогональном дополнении.
    Покажем, что $\overline{L}^{\perp} \subset L^{\perp}$.
    Пусть последовательность $h_n \in L$ такая, что $h_n \to \Phi \in \overline{L}$.
    Пусть $g \in L^{\perp}$, тогда $(h_n, g) = 0$, $h_n \to \Phi$, поэтому $(\Phi, g) = 0\;\forall \Phi \in \overline{L}$ по непрерывности скалярного произведения.
    Получаем, что
    $$
    \overline{L} \oplus L^{\perp} = H
    $$
    Теперь разложим $f = f_1 + f_2$, $f_1 \in \overline{L}$, $f_2 \in L^{\perp}$ и $f$ перпендикулярен $f_2$.
    $$
    0 = (f, f_2) = (f_1, f_2) + \|f_2\|^2
    $$
    следовательно $\|f_2\| = 0$, следовательно $f = f_1 \in \overline{L} \Rightarrow L^{\perp\perp} \supset \overline{L}$.
\end{Proof}

\begin{MathCl}{Следствие}
    $\overline{\Im A} = (\ker A^*)^\perp$
\end{MathCl}

\begin{Prim}
	Рассмотрим оператор $A : L_2(G) \mapsto L_2(G)$, действующий следующим
	образом:
	$$
	(Af)(x) = \int \limits_G K(t, x) f(t) dt
	$$
	где $K(t, x) \in L_2(G \times G)$.
	Рассмотрим скалярное произведение $(Af, g)$
	$$
	(Af, g) = \int \limits_G dx \int \limits_G dt K(t, x) f(t) \overline{g(x)}=
	\int \limits_G dt f(t) \int \limits_G dx \overline{K(t, x)} g(x)
	$$
	где $\int \limits_G dx \overline{K(t, x)} g(x) = (A^* g)(t)$.
	
	Расмотрим частный случай.
	Пусть $A:L_2[0,1] \mapsto L_2[0,1]$,
	$$
	(A f)(x) = \int \limits_0^x f(t) dt
	$$
	Тогда сопряжённый оператор будет иметь вид:
	$$
	(A^*g)(t) = \int \limits_t^1 g(x) dx
	$$
	Образ этого оператора $\Im A^* \supset \{h\in C^1[0,1], h(1) = 0\}$ --- всюду
	плотно в $L_2[0,1]$.
	Очевидно, что $\overline{Im A^*}^\perp = H^\perp = 0$
\end{Prim}

Введём уравнение Фредгольма.
$(I - \lambda A) u = f\;f\in H\;\lambda \in \mathbb C \setminus\{0\}$.
Пусть $\dfrac{1}{\lambda} \in \rho(A) \Rightarrow \exists! u = (I - \lambda A)^{-1}f$ 
и $\overline{\Im(I - \lambda A)} = (\ker(I - \lambda A)^*)^\perp$
\end{document}
